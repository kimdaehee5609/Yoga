%	-------------------------------------------------------------------------------
% 
%		요가
%
%
%
%
%
% 		요가 지도법
%
%		요가대회
%		
%		2020년 7월 9일 작성
%
%	-------------------------------------------------------------------------------

	\documentclass[12pt, a4paper, oneside]{book}
%	\documentclass[12pt, a4paper, landscape, oneside]{book}

		% --------------------------------- 페이지 스타일 지정
		\usepackage{geometry}
%		\geometry{landscape=true	}
		\geometry{top 		=10em}
		\geometry{bottom		=10em}
		\geometry{left		=8em}
		\geometry{right		=8em}
		\geometry{headheight	=4em} % 머리말 설치 높이
		\geometry{headsep		=2em} % 머리말의 본문과의 띠우기 크기
		\geometry{footskip		=4em} % 꼬리말의 본문과의 띠우기 크기
% 		\geometry{showframe}
	
%		paperwidth 	= left + width + right (1)
%		paperheight 	= top + height + bottom (2)
%		width 		= textwidth (+ marginparsep + marginparwidth) (3)
%		height 		= textheight (+ headheight + headsep + footskip) (4)



		%	===================================================================
		%	package
		%	===================================================================
%			\usepackage[hangul]{kotex}				% 한글 사용
			\usepackage{kotex}					% 한글 사용
			\usepackage[unicode]{hyperref}			% 한글 하이퍼링크 사용

		% ------------------------------ 수학 수식
			\usepackage{amssymb,amsfonts,amsmath}	% 수학 수식 사용
			\usepackage{mathtools}				% amsmath 확장판

			\usepackage{scrextend}				% 
		

		% ------------------------------ LIST
			\usepackage{enumerate}			%
			\usepackage{enumitem}			%
			\usepackage{tablists}				%	수학문제의 보기 등을 표현하는데 사용
										%	tabenum


		% ------------------------------ table 
			\usepackage{longtable}			%
			\usepackage{tabularx}			%
			\usepackage{tabu}				%




		% ------------------------------ 
			\usepackage{setspace}			%
			\usepackage{booktabs}		% table
			\usepackage{color}			%
			\usepackage{multirow}			%
			\usepackage{boxedminipage}	% 미니 페이지
			\usepackage[pdftex]{graphicx}	% 그림 사용
			\usepackage[final]{pdfpages}		% pdf 사용
			\usepackage{framed}			% pdf 사용

			
			\usepackage{fix-cm}	
			\usepackage[english]{babel}
	
		%	=======================================================================================
		% 	tikz package
		% 	
		% 	--------------------------------- 	
			\usepackage{tikz}%
			\usetikzlibrary{arrows,positioning,shapes}
			\usetikzlibrary{mindmap}			
			

		% --------------------------------- 	page
			\usepackage{afterpage}		% 다음페이지가 나온면 어떻게 하라는 명령 정의 패키지
%			\usepackage{fullpage}			% 잘못 사용하면 다 흐트러짐 주의해서 사용
%			\usepackage{pdflscape}		% 
			\usepackage{lscape}			%	 


			\usepackage{blindtext}
	
		% --------------------------------- font 사용
			\usepackage{pifont}				%
			\usepackage{textcomp}
			\usepackage{gensymb}
			\usepackage{marvosym}



		% Package --------------------------------- 

			\usepackage{tablists}				%


		% Package --------------------------------- 
			\usepackage[framemethod=TikZ]{mdframed}				% md framed package
			\usepackage{smartdiagram}								% smart diagram package



		% Package ---------------------------------    연습문제 

			\usepackage{exsheets}				%

			\SetupExSheets{solution/print=true}
			\SetupExSheets{question/type=exam}
			\SetupExSheets[points]{name=point,name-plural=points}


		% --------------------------------- 페이지 스타일 지정

		\usepackage[Sonny]		{fncychap}

			\makeatletter
			\ChNameVar	{\Large\bf}
			\ChNumVar	{\Huge\bf}
			\ChTitleVar		{\Large\bf}
			\ChRuleWidth	{0.5pt}
			\makeatother

%		\usepackage[Lenny]		{fncychap}
%		\usepackage[Glenn]		{fncychap}
%		\usepackage[Conny]		{fncychap}
%		\usepackage[Rejne]		{fncychap}
%		\usepackage[Bjarne]	{fncychap}
%		\usepackage[Bjornstrup]{fncychap}

		\usepackage{fancyhdr}
		\pagestyle{fancy}
		\fancyhead{} % clear all fields
		\fancyhead[LO]{\footnotesize \leftmark}
		\fancyhead[RE]{\footnotesize \leftmark}
		\fancyfoot{} % clear all fields
		\fancyfoot[LE,RO]{\large \thepage}
		%\fancyfoot[CO,CE]{\empty}
		\renewcommand{\headrulewidth}{1.0pt}
		\renewcommand{\footrulewidth}{0.4pt}
	
	
	
		%	--------------------------------------------------------------------------------------- 
		% 	tritlesec package
		% 	
		% 	
		% 	------------------------------------------------------------------ section 스타일 지정
	
			\usepackage{titlesec}
		
		% 	----------------------------------------------------------------- section 글자 모양 설정
			\titleformat*{\section}					{\large\bfseries}
			\titleformat*{\subsection}				{\normalsize\bfseries}
			\titleformat*{\subsubsection}			{\normalsize\bfseries}
			\titleformat*{\paragraph}				{\normalsize\bfseries}
			\titleformat*{\subparagraph}				{\normalsize\bfseries}
	
		% 	----------------------------------------------------------------- section 번호 설정
			\renewcommand{\thepart}				{\arabic{part}.}
			\renewcommand{\thesection}				{\arabic{section}.}
			\renewcommand{\thesubsection}			{\thesection\arabic{subsection}.}
			\renewcommand{\thesubsubsection}		{\thesubsection\arabic{subsubsection}}
			\renewcommand\theparagraph 			{$\blacksquare$ \hspace{3pt}}

		% 	----------------------------------------------------------------- section 페이지 나누기 설정
			\let\stdsection\section
			\renewcommand\section{\newpage\stdsection}



		%	--------------------------------------------------------------------------------------- 
		% 	\titlespacing*{commandi} {left} {before-sep} {after-sep} [right-sep]		
		% 	left
		%	before-sep		:  수직 전 간격
		% 	after-sep	 	:  수직으로 후 간격
		%	right-sep

			\titlespacing*{\section} 			{0pt}{1.0em}{1.0em}
			\titlespacing*{\subsection}	  		{0ex}{1.0em}{1.0em}
			\titlespacing*{\subsubsection}		{0ex}{1.0em}{1.0em}
			\titlespacing*{\paragraph}			{0em}{1.5em}{1.0em}
			\titlespacing*{\subparagraph}		{4em}{1.0em}{1.0em}
	
		%	\titlespacing*{\section} 			{0pt}{0.0\baselineskip}{0.0\baselineskip}
		%	\titlespacing*{\subsection}	  		{0ex}{0.0\baselineskip}{0.0\baselineskip}
		%	\titlespacing*{\subsubsection}		{6ex}{0.0\baselineskip}{0.0\baselineskip}
		%	\titlespacing*{\paragraph}			{6pt}{0.0\baselineskip}{0.0\baselineskip}
	

		% --------------------------------- recommend		섹션별 페이지 상단 여백
		\newcommand{\SectionMargin}			{\newpage  \null \vskip 2cm}
		\newcommand{\SubSectionMargin}		{\newpage  \null \vskip 2cm}
		\newcommand{\SubSubSectionMargin}		{\newpage  \null \vskip 2cm}


		%	--------------------------------------------------------------------------------------- 
		% 	toc 설정  - table of contents
		% 	
		% 	
		% 	----------------------------------------------------------------  문서 기본 사항 설정
			\setcounter{secnumdepth}{4} 		% 문단 번호 깊이
			\setcounter{tocdepth}{2} 			% 문단 번호 깊이 - 목차 출력시 출력 범위

			\setlength{\parindent}{0cm} 		% 문서 들여 쓰기를 하지 않는다.


		%	--------------------------------------------------------------------------------------- 
		% 	mini toc 설정
		% 	
		% 	
		% 	--------------------------------------------------------- 장의 목차  minitoc package
			\usepackage{minitoc}

			\setcounter{minitocdepth}{1}    	%  Show until subsubsections in minitoc
%			\setlength{\mtcindent}{12pt} 	% default 24pt
			\setlength{\mtcindent}{24pt} 	% default 24pt

		% 	--------------------------------------------------------- part toc
		%	\setcounter{parttocdepth}{2} 	%  default
			\setcounter{parttocdepth}{0}
		%	\setlength{\ptcindent}{0em}		%  default  목차 내용 들여 쓰기
			\setlength{\ptcindent}{0em}         


		% 	--------------------------------------------------------- section toc

			\renewcommand{\ptcfont}{\normalsize\rm} 		%  default
			\renewcommand{\ptcCfont}{\normalsize\bf} 	%  default
			\renewcommand{\ptcSfont}{\normalsize\rm} 	%  default


		%	=======================================================================================
		% 	tocloft package
		% 	
		% 	------------------------------------------ 목차의 목차 번호와 목차 사이의 간격 조정
			\usepackage{tocloft}

		% 	------------------------------------------ 목차의 내어쓰기 즉 왼쪽 마진 설정
			\setlength{\cftsecindent}{2em}			%  section

		% 	------------------------------------------ 목차의 목차 번호와 목차 사이의 간격 조정
			\setlength{\cftsecnumwidth}{2em}		%  section





		%	=======================================================================================
		% 	flowchart  package
		% 	
		% 	------------------------------------------ 목차의 목차 번호와 목차 사이의 간격 조정
			\usepackage{flowchart}
			\usetikzlibrary{arrows}



		%	=======================================================================================
		% 	줄 간격 설정
		% 	
		% 	
		% 	--------------------------------- 	줄간격 설정
			\doublespace
%			\onehalfspace
%			\singlespace
		
		

	% 	============================================================================== itemi Global setting

	
		%	-------------------------------------------------------------------------------
		%		Vertical spacing
		%	-------------------------------------------------------------------------------
			\setlist[itemize]{topsep=0.0em}			% 상단의 여유치
			\setlist[itemize]{partopsep=0.0em}			% 
			\setlist[itemize]{parsep=0.0em}			% 
%			\setlist[itemize]{itemsep=0.0em}			% 
			\setlist[itemize]{noitemsep}				% 
			
		%	-------------------------------------------------------------------------------
		%		Horizontal spacing
		%	-------------------------------------------------------------------------------
			\setlist[itemize]{labelwidth=1em}			%  라벨의 표시 폭
			\setlist[itemize]{leftmargin=8em}			%  본문 까지의 왼쪽 여백  - 4em
			\setlist[itemize]{labelsep=3em} 			%  본문에서 라벨까지의 거리 -  3em
			\setlist[itemize]{rightmargin=0em}			% 오른쪽 여백  - 4em
			\setlist[itemize]{itemindent=0em} 			% 점 내민 거리 label sep 과 같은면 점위치 까지 내민다
			\setlist[itemize]{listparindent=3em}		% 본문 드려쓰기 간격
	
	
			\setlist[itemize]{ topsep=0.0em, 			%  상단의 여유치
						partopsep=0.0em, 		%  
						parsep=0.0em, 
						itemsep=0.0em, 
						labelwidth=1em, 
						leftmargin=2.5em,
						labelsep=2em,			%  본문에서 라벨 까지의 거리
						rightmargin=0em,		% 오른쪽 여백  - 4em
						itemindent=0em, 		% 점 내민 거리 label sep 과 같은면 점위치 까지 내민다
						listparindent=0em}		% 본문 드려쓰기 간격
	
%			\begin{itemize}
	
		%	-------------------------------------------------------------------------------
		%		Label
		%	-------------------------------------------------------------------------------
			\renewcommand{\labelitemi}{$\bullet$}
			\renewcommand{\labelitemii}{$\bullet$}
%			\renewcommand{\labelitemii}{$\cdot$}
			\renewcommand{\labelitemiii}{$\diamond$}
			\renewcommand{\labelitemiv}{$\ast$}		
	
%			\renewcommand{\labelitemi}{$\blacksquare$}   	% 사각형 - 찬것
%			\renewcommand\labelitemii{$\square$}		% 사각형 - 빈것	
			






% ------------------------------------------------------------------------------
% Begin document (Content goes below)
% ------------------------------------------------------------------------------
	\begin{document}
	




			\title{요가}
			\author{김대희}
			\date{2018년 10월 7일}
			\maketitle


			\dominitoc
			\doparttoc[0]			

			\tableofcontents 		% 목차 출력
			\cleardoublepage
			\listoffigures 			% 그림 목차 출력
			\cleardoublepage
			\listoftables 			% 표 목차 출력





		\mdfdefinestyle	{con_specification} {
						outerlinewidth		=1pt			,%
						innerlinewidth		=2pt			,%
						outerlinecolor		=blue!70!black	,%
						innerlinecolor		=white 			,%
						roundcorner			=4pt			,%
						skipabove			=1em 			,%
						skipbelow			=1em 			,%
						leftmargin			=0em			,%
						rightmargin			=0em			,%
						innertopmargin		=2em 			,%
						innerbottommargin 	=2em 			,%
						innerleftmargin		=1em 			,%
						innerrightmargin		=1em 			,%
						backgroundcolor		=gray!4			,%
						frametitlerule		=true 			,%
						frametitlerulecolor	=white			,%
						frametitlebackgroundcolor=black		,%
						frametitleaboveskip=1em 			,%
						frametitlebelowskip=1em 			,%
						frametitlefontcolor=white 			,%
						}


	




%	================================================================== Part			요가 에세이
%	\addtocontents{toc}{\protect\newpage}
	\part{요가 에세이}
	\noptcrule
	\parttoc				


	\chapter{요가 에세이}
	\minitoc% Creating an actual minitoc



	% -----------------------------------------------------------------------------
	%
	% -----------------------------------------------------------------------------
	\section{요가 에세이}



	% -----------------------------------------------------------------------------
	%
	% -----------------------------------------------------------------------------
	\section{아쉬탕가 요가}



	% -----------------------------------------------------------------------------
	%
	% -----------------------------------------------------------------------------
	\section{빈야사 요가}



	% -----------------------------------------------------------------------------
	%
	% -----------------------------------------------------------------------------
	\section{요가 - 반다}







%	================================================================== Part			요가 개요
%	\addtocontents{toc}{\protect\newpage}
	\part{요가 개요}
	\noptcrule
	\parttoc				


\chapter{요가 개요}
\minitoc% Creating an actual minitoc

% -----------------------------------------------------------------------------
%
%
%
% -----------------------------------------------------------------------------
\newpage
\section{요가의 발생}


% -----------------------------------------------------------------------------
%
%
%
% -----------------------------------------------------------------------------
\newpage
\section{요가의 역사와 유래}



\subsection{요가의 사상}





% -----------------------------------------------------------------------------
%
%
%
% -----------------------------------------------------------------------------
\chapter{요가의 문헌 및 유물}
\minitoc% Creating an actual minitoc


\section{요가의 문헌}



\section{요가의 유물}


% -----------------------------------------------------------------------------
%
%
%
% -----------------------------------------------------------------------------
\chapter{요가의 경전}
\minitoc% Creating an actual minitoc

\section{요가의 경전}

\subsection{요가의 경전}




			\begin{itemize}[topsep=0.0em, parsep=0.0em, itemsep=0em, leftmargin=12.0em, labelwidth=3em, labelsep=3em] 
			\item 리그 베다
			\item 마하 바라타
			\item 우파니 샤드
			\item 요가 수트라
			\item 바가바드기타
			\end{itemize}







\section{리그 베다}















\section{마하바라타}

\subsection{개요} 
인도 고대의 서사시.

'바라타족의 전쟁을 읊은 대사시’란 뜻으로 오랜 세월에 걸쳐 구전되다가 오는 사이에 4세기경에 지금의 형태를 갖추게 된 것으로 여겨진다. 
18편 10만 슬로카로 구성되었는데, 1슬로카는 2행이므로약 20만 행에 달한다고 할 수 있다.

바라타족의 사촌 간인 판다바 형제와 카우라바 형제 간의 갈등과 기원전 4세기에서 기원전 1세기 사이에 실제로 있었을 것이라 추정되는 18일간에 걸친 쿠룩셰트라 전투를 그렸다. 그러나 이러한 주요 줄거리는 마하바라타의 전체 분량의 약 5분의 1에 지나지 않으며 중간중간 고대의 전설, 윤리, 종교, 법, 당시의 생활상에 이르기까지 다양한 이야기가 포함되어 있어 "모든 것은 마하바라타에 있나니, 마하바라타에 없는 것은 이 세상에 없다"는 말이 있을 정도. 
고대 인도를 연구하는 데 빼놓을 수 없는 작품이다. 
라마야나와 더불어 인도 전역과 인도네시아 등지에서 높은 인기를 누려왔으며, 이들 나라에서는 만화나 애니메이션, 영화, 드라마 등으로 많이 제작되었다. 
영국 연출가 피터 브룩에 의해 연극으로 각색되어 서구 여러 나라에서 호평을 받기도 했다. 
중국의 삼국지연의, 서구의 일리아스에 비견될 정도의 작품이라고 할 수 있으며, 시 안에서 다루는 범위는 이들 작품보다 훨씬 광범위하다. 

아바타의 감독으로 유명한 제임스 카메룬 역시 이 작품을 좋아하고 있으며, 아바타 3부작이 끝나면 마하바라타를 영화로 만들고 싶다는 의사를 밝히기도 했고, 마하바라타를 영화로 만들 인도 영화사가 있다면 3D 영화 기술을 제공할 용의가 있다고 언급하기도 했다.

한편 작중 엔진, 미사일, 로켓 등으로 추정되는 무기나 도구들이 나오는데(판다바 형제가 탈출할 때 배에 달린 '엔진'의 '시동'을 걸었다는 표현이라든가..), 이 때문에 초고대의 실전된 과학에 대한 떡밥이 끊이지 않고 있다. 초고대문명설을 믿는 자들이 '고대에도 핵무기와 핵전쟁이 있었다'라고 주장하면서 이 신화를 인용하기도 하지만, 물론 마하바라타는 어디까지나 신화적인 서사시이므로 당시의 생활 습관이나 종교관 등을 제외하면 사료적 가치는 사실상 없다.

인도 외의 나라에서는 단순히 서사시로만 알려져 있지만, 엄연히 힌두교의 경전 중 하나이기 때문에 그 점에 유의하는 것이 좋다. 
현대 힌두교에서 최고로 치는 경전 바가바드 기타가 바로 마하바라타의 일부라는 점을 유념. 
이 경전의 문구는 현대에서도 자주 인용되어 서양인에게도 유명하다. 
원자폭탄 실험에서 오펜하이머가 말했던 "나는 죽음이요,세상의 파괴자가 되었도다."가 바로 바가바드 기타에서 크리슈나가 한 말이다.





\subsection{상세} 

\paragraph{}
선을 상징하는 판다바 형제가 오랜 고난과 결전 끝에 악을 상징하는 카우라바 형제들을 쳐부순다라고 요약할 수 있지만 양쪽을 지원하는 수많은 영웅들이 각자의 이야기를 담고 있어서 군상극으로도 볼 수 있다. 무엇보다 주인공이라 할 수 있는 판다바 형제들도 전투에서 승리하기 위해 여러 가지 편법과 속임수를 쓰는 등의 모순을 내포하고 있으며, 악역인 카우라바 형제들 역시 자기 입장이 있고 나름대로의 신념을 관철하고 있어 단순히 권선징악적인 내용을 담았다고는 볼 수 없다.

\paragraph{}
판다바 5형제 중에서도 비중이 높은 쿤티 소생의 3형제인 유디슈티라, 비마, 아르주나 셋의 관계는 삼국지의 유비, 장비, 관우와 많이 닮아있다. 이들은 각각 자신의 특기와 능력치를 가지고 있으며 그중 아르주나가 자타공인 제일가는 영웅이긴 하지만 유디슈티라 또한 창의 달인이자 뛰어난 전차 기수이며 비마는 힘에서 따를 자가 없는 자로 묘사된다.

\paragraph{}
전투가 이어지는 후반부 전개는 현대의 소설에도 떨어지지 않을 정도이며 고대 서사시라는 게 믿기지 않을 정도로 개성 강한 캐릭터들과 임팩트넘치는 사건들이 등장하니, 문학작품으로서도 한번쯤 읽어볼 만 하다.

\paragraph{}
현대의 관점으로 보면 신들의 선택을 받아 초인의 경지에 오른 판다바 형제들보단, 일단은 카우라바 측이 정부군이고 판다바 측이 반란군이기 때문에, 은혜나 또는 우정 때문에, 혹은 속임수에 넘어가서 어쩔 수 없이 싸움에 임하는 다크 히어로들이 포진한 카우라바 형제 측 진영 쪽이 오히려 선한 주인공으로 보인다는 감상도 있다.

\paragraph{}
일반적으로 카우라바를 악역으로, 판다바를 선역으로 묘사하기는 하지만 읽다보면 작중에서도 은근히 판다바를 까기도 하고 카우라바 측의 인물의 미덕도 묘사되기 때문에 결국 판다바가 이긴다고는 해도 전형적인 권선징악물로 보긴 조금 어렵다. 
인물묘사도 비교적 입체적. 대표적으로 카우라바 측의 영웅인 카르나는 부탁을 절대 거절하지 않고 남을 생각해주는 선량한 인물이나 신분을 속였다는 이유로 저주를 받고 이것이 하나의 요인으로 작용해 사망한다. 단, 그렇다고 카르나가 마냥 고결한 것은 아니고 나쁜 짓은 충분히 했다. 한편 판다바 형제 중 비마는 작중 내내 하층민을 무시하고, 술을 마시고, 여기저기서 온갖 사고와 깽판을 부리고 다니는데도 끝까지 영웅으로 추앙된다. 때문에 힌두교도가 아닌 제3자의 입장으로 보면 두 편의 선악에 대해 명확히 판정을 내리기 쉽지 않다...고는 해도 결국 대체로 먼저 시비를 걸거나 원인을 제공하는 쪽은 카우라바인 것은 명확하다. 판다바는 어느 정도 양보도 하고 원수인 그들에게 은혜도 베풀지만 카우라바는 그런 점도 없기도 하고. 카우라바가 동원하는 수단 또한 악랄한 것이 많다. 결국 판다바가 결함 있는 선역, 카우라바 쪽이 이해의 여지는 있는 악역 정도랄까.

\paragraph{}
비슈누의 8번째 화신(化神)인 크리슈나도 여기에 등장한다. 쿠루크셰트라 전투 전날 크리슈나가 아르주나에게 가르침을 주는 노래인 바가바드 기타는 힌두교 사상 최고의 평가를 받는 경전들 중 하나이다. 그러나 그 이외의 부분에서는 주인공들에게 거짓말을 하게 하거나 법도에 어긋난 책략을 쓰게 하거나 하는 등, 뭐랄까 타락시키는 느낌의 캐릭터. 이는 이 전쟁이 '크샤트리아를 몰락시키는 전쟁'으로 예정되었으며 또한 마침 이때가 깔리 유가(일종의 말세)로 들어서는 시기라서 예전의 법도 등이 지켜지지 않는 게 당연한 일이 되기 때문으로 해석할 수 있다. 
사실 다른 신화에서 나오는 크리슈나와 비교하면 상당히 이질적인 느낌이 드는 편이기는 하다. 이 때문에 분석도 꽤 활발하다.


\section{우파니샤드}

\section{요가 수트라}

요가수트라의 요가는 "정신의 활동이 멈춤"과 정신의 활동을 머추게 하기"를 의미한다.





\section{바가바드기타}





\subsection{하타요가 경전}


% -----------------------------------------------------------------------------
%
%
%
% -----------------------------------------------------------------------------
\newpage
\section{요가 책}


	\subsection{힌두 탄트라 입문}

			윤기봉, 김재천 편역


	\subsection{아사나 프라나야 무드라 반다}

Asana Pranayama Mudra Bandha
아사나 프라나야 무드라 반다
위대한 요가 스승 쉬바난다의 제자 사띠 야안다가 쓴 책




	\subsection{요가 마카란다 (yoga makaranda)}


% -----------------------------------------------------------------------------
%
%
%
% -----------------------------------------------------------------------------
\newpage
\section{(내 몸에 맞는) 요가교정 A.P.C : 체형을 알면 교정이 보인다}

\paragraph{제 목}
	(내 몸에 맞는) 요가교정 A.P.C : 체형을 알면 교정이 보인다

\paragraph{저자} 정두화 지음 \\
저자 정두화는 전직 약사인 바유는 약뿐만 아니라 삶의 문제를 해결하는 처방에 대한 다양한 모색과 실험 끝에 1996년부터 요가여행을 시작했고 1999년부터는 매일 연습하는 삶을 선택하였다. 2001년부터 요가의 뿌리를 찾아 인도를 여행하다가 마침내 2005-2007년 인도 마이솔에서 구루지, 쉬리 파타비 조이스를 통해 아쉬탕가 요가의 전통과 연결되었다. 2008년 북경에서 존 스콧 선생님을 만나 빈야사의 참의미에 눈뜨고 [아쉬탕가 요가]를 번역 출간 하였고, 2009년 뉴질랜드 넬슨의 스틸포인트 요가센터에서 아쉬탕가 요가 지도자 과정을 졸업하였다. 
2007년부터 고향인 부산에서 정통 마이솔 수업을 열고 2010년 해운대 요가 VnA를 설립하면서 아쉬탕가 빈야사 요가 전통의 과학적인 현대 해석을 위해 노력하고 있다. 2012년부터 요가쿨라에서, 2014년부터 존 스콧 차이나 요가 지도자 과정에서 요가 해부학 강의를 담당하고 있으며, 요가 해부학 워크샵의 연구 주제들을 10년간의 소그룹 수업의 체험에 비추어 자세 타입에 따른 요가 교정의 이론과 방법을 정립하고 있다.

\paragraph{출판사}
/ 요가저널코리아
줄거리	불균형 패턴을 정확하게 읽어 체형에 맞는 교정이 이루어질 수 있도록 돕는『내몸에 맞는 요가교정 A.P.C』. 이 책은 특정한 요가 자세가 안되는 것이 평상시 자세 불균형이 요가 자세를... [더보기]
자료실	중앙도서관 - 종합자료실(3층) [512.57-418]	001
대출 상태	현재 대출가능한 책입니다.


\paragraph{책소개}


불균형 패턴을 정확하게 읽어 체형에 맞는 교정이 이루어질 수 있도록 돕는『내몸에 맞는 요가교정 A.P.C』. 이 책은 특정한 요가 자세가 안되는 것이 평상시 자세 불균형이 요가 자세를 통해 드러나는 것이라고 말하며, 수련시 잘되지 않는 요가자세의 올바른 교정방법을 제시하여 균형회복을 돕는다. 따라서 먼저, 불균형을 어떻게 관찰할 것인가를 먼저 다루고 불균형 패턴에 따라 체계적으로 나눌 수 있는 체형에 대해 살펴본다. 마지막으로 체형에 따른 교정의 원리와 구체적인 방법들을 제시한다.


\paragraph{책 목차}

머리말 
- 불균형 관찰에 따른 체형 교정 
- 요가의 길 
- 특정 자세나 특정 방향이 안된다면? 
- 요가 자세에서 나타나는 불균형 패턴 
- 같은 자세에 대한 교정방법이 상반될 수 있는 이유는? 

1부 관찰 

1장 요가자세를 통한 불균형 관찰 방법 

$-$ 요가 자세의 기본패턴은 전굴과 후굴 
$-$ 관절운동의 기본패턴은 굴곡과 신전 
$-$ 균형자세와 불균형자세패턴 
$-$ 노선 불균형과 구간 불균형 
$-$ 단일구간과 복합구간자세 
$-$ 복합구간 전굴자세 관찰 
$-$ 복합구간 후굴자세 관찰 

2장 알게 되면 보인다 
$-$ 척추를 보는 포인트: 만곡과 1,2차 커브 
$-$ 척추커브와 골반경사의 관계 알아차리기 
$-$ Q1골반경사와 고관절 굴곡/신전의 차이 
$-$ 그리고 골반경사 불균형의 의미는? 
$-$ 척추 골반의 균형을 좌우하는4가지 코어근육 
$-$ 요추 골반 리듬 
$-$ 구간불균형 = 척추 골반 근육 불균형 연결하기 

2부 체형 

3장 4가지 체형 (자세 타입) 

$-$ A타입과 P타입 구별하기 
$-$ A타입과 C타입 구별하기 
$-$ 4가지 자세타입 
$-$ 3가지 주요패턴 
$-$ 척추골반 불균형과 하지의 관절운동 패턴 

4장 직립자세 불균형모델 

$-$ 직립자세 딜레마 : 불균형의 시초 
$-$ 이다, 핑갈라 에너지 채널 
$-$ 자세불균형과 호흡패턴 

3부 교정 
$-$ 타입에 따른 교정의 특정 의도 
$-$ 타입에 따른 교정전략 
$-$ Q2: 균형과 불균형의 기준은 무엇인가? 
$-$ Q3: 유연성을 기르는 것이 균형회복에 어떤 도움이 되는가? 

5장 P타입 교정 

$-$ P타입 교정의 의도 
$-$ P타입 전굴자세의 교정 
$-$ Q4무릎을 굽히면 후면노선 3번구간의 늘림을 방해하지 않나? 
$-$ P타입 후굴자세의 교정 
$-$ P타입 측굴 자세의 교정 

6장 A타입 교정 

$-$ A타입 교정의도 
$-$ A타입 전굴자세 교정 
$-$ A타입 후굴자세 교정 
$-$ A타입 측굴 자세교정 

7장 C타입 교정 

$-$ C타입 교정의도 
$-$ Q5 왜 C타입 교정에만 1차 의도와 2차 의도를 나누었는가? 
$-$ 나비자세 A와 B의 차이 
$-$ 거북이 / 잠자는 거북이자세 
$-$ Q6 전굴자세를 할 때 고관절과 요추굴곡의 적절한 비율은? 
$-$ Q7 요가수련을 하다보면 통증과 만나는 경우가 많다. 통증을 어떻게 보아야 하고 통증을 다루는 기본원칙은 무엇인가? 

8장 사지의 정렬 

$-$ 태양맞이 시퀀스 (수리야 나마스까라) 
$-$ 골반경사와 하지 회전패턴의 연결 
$-$ 고관절 회전 불균형 관찰방법 
$-$ 고관절과 슬관절의 내회전과 외회전의 차이 
$-$ 아래로 얼굴 개자세 하지교정 
$-$ 위로 얼굴 개자세 하지 교정 
$-$ Q8 위로 얼굴 개자세를 할때 대둔근을 수축해야 하는가, 이완해야 하는가? 
$-$ Q9 상지의 안정성을 높이는 암밸런스 자세들 중에서 가장 중요한 자세들은? 
$-$ 견갑골과 어깨관절 불균형 
$-$ 아래로 얼굴 개자세 상지교정 
$-$ 특정의도 상지 교정자세들 

부록 요가 자세 체형비교 방법 

$-$ 4가지 타입의 15가지 요가자세 비교 
$-$ 아사나 인덱스 
$-$ 감사의 말

% -----------------------------------------------------------------------------
%
%
%
% -----------------------------------------------------------------------------
\chapter{요가의 기초}
\minitoc% Creating an actual minitoc


\section{요가의 본질}

\section{요가의 의미}

\section{자기초월의 단계 - 수행자}

\section{인도하는 빛 - 스승}


\section{자아를 넘어서는 배움 - 수행}


\section{새로운 정체성을 위해 태어남 - 입문}


\section{미치광이 같은 지혜와 미치광이 같은 숙련자들 }





\section{요가의 정의}


		\begin{itemize}[topsep=0.0em, parsep=0.0em, itemsep=0em, leftmargin=6.0em, labelwidth=3em, labelsep=3em] 
			\item 	요가의 정의를 알아봅시다.
			\item 	요가는 마음이 대상과 결합된 상태로 자신의 주의력을 모아주고 집중시키며 그것을 사용하고 응용하게 해줍니다.
			\item 	마음을 조절하고 마음의 움직임을 억제해서 인간 본래의 고요한 마음으로 돌아가는 상태라고  할 수 있습니다. 
					자신의 마음을 컨트롤하는 방법을 배우고 아사나와 호흡으로 집중함으로써 마음을 컨트롤하는 것입니다. 
					인간의 마음이 육체를 잘 통제하여 바른 길로 갈 수 있게 하는 것이 요가입니다.
		\end{itemize}

			스스로 새로운 육체를 만들어가는것


\section{요가의 효과}


\paragraph{효과}
요가를 함으로써 얻어지는 효과는 무엇일까요?
우리는 요가의 효과를 정신적인 효과와 육체적인 효과로 나누어 생각해 볼 수 있습니다.


\paragraph{}정신적인 효과는

1. 스트레스와 불안감을 해소시킨다.

2. 긍정적인 사고와 자신감을 길러준다.

3. 조절능력을 키워주고 통제력을 향상시킨다.

4. 긴장감을눅러뜨려 평온해진다.

5. 깨달음을 얻을 수 있다.



\paragraph{}육제적인 효과는 

1. 몸의 발란스를 맞추준다.

2. 혈압과 뇌파를 안정시킨다.

3. 독소와 노폐물을 제거한다.

4, 수련을 통해 세포를 활성화 시킨다.

5. 산소소비량과 탄산가스 배출량을 줄여준다.



이처럼 요가는 건강한 육체뿐 아니라 건강한 마음을 갖을 수 있도록 도와주는 운동이라고 할 수 있습니다.



\paragraph{}아쉬탕가 요가 8단계를 살펴볼께요.

1. 야마 금지계로 살생이나 거짓말, 도둑질, 성욕, 탐욕 등 스스로 노력하여 금해야 할 사회적 도덕적 계율을 말합니다.

2. 니야마 권고계는 청결, 만족, 고행, 학습, 신에 기원등 권장하는 계율을 말합니다.

3. 아사나는 동작을 의미합니다.

4. 숨을 제어하는 숨제어는 프라나야마라고 합니다.

5. 감각을 제어하여 주의력을 모으는 단계를 프라티아하라 라고 합니다.

6. 집중을 의미하는 다라나 단계가 있습니다. 

7. 집중이 이어지는 명상은 디아나 입니다.

8. 환희 평화,자유를 의미하는 사마디 즉 삼매단계입니다.





%	==========================================================================
	\chapter{요가의 종류와 분류}
	\minitoc

% -----------------------------------------------------------------------------
%
%
%
% -----------------------------------------------------------------------------
\newpage
\section{요가의 종류와 분류}



\section{요가의 종류}



\paragraph{요가의 종류는? }
요가의 어원은 고대 인도어 유즈(Yuj)에서 파생된 것. ‘결합한다’, ‘단련한다’는 의미로 인도에서 시작돼 현재는 호흡과 몸의 균형을 잡아주는 운동 중 하나로 자리매김했다. 고대 인도에서부터 전해졌기 때문에 이론이나 그에 따른 수행법이 다양하지만 크게 하타, 라자, 박티, 카르마, 만트라 요가 등으로 나뉜다. 이 중 우리나라에 주로 보급된 요가는 육체적, 생리적인 면을 중시하는 ‘하타 요가’와 체위 요가라 불리는 ‘카르마 요가’다. 

\paragraph{하타(Hatha) 요가 }
가장 오래된 요가로 생리적으로 몸가짐을 다스리고 숨쉬기를 조절하여 육체적 잠재력을 깨우는 요가다. 전 세계에 가장 활발하게 보급된 요가로 수많은 스포츠와 무술, 체조, 무용, 등 나라마다 다양하게 응용되고 있다. 최근에 유행하는 핫요가나 비크람요가 등이 하타요가의 일종이라고 볼 수 있다. 

\paragraph{라자(Raja)요가 }
라자요가는 요가의 왕이라는 뜻을 갖고 있을 정도로 모든 요가의 궁극적인 의미를 담고 있다. 명상을 통하여 마음의 평화를 찾고 지혜를 얻으며 해탈의 경지를 추구하는 명상 요가이다. 

\paragraph{카르마(Karma) 요가 }
‘카르마’는 행동을 뜻하며 체위요가로도 불린다. 사색, 명상, 지식 등 심리요가에 대한 보완책으로 나온 요가이며 바른 앎보다 바른 행동으로 이기적이 아닌 이타적인 행동을 강조하는 요가다. 마음을 비우고 모든 것을 포기하는 것만이 큰 성취를 이룰 수 있는 것이 아니기 때문에 행동에 맞게 자신에게 주워진 상황으로부터 자유로워지는 요가다. 

\paragraph{박티(Bhakti)요가 }
‘박티’는 ‘참여한다’, ‘함께한다’는 뜻을 갖고 있다. 신성과 스승에 대한 헌신과 봉사와 사랑을 의미한다. 이 사랑과 헌신이 자신과 주변을 변화시켜 에너지를 갖게 한다. 타인을 위해 희생하는 기쁨을 즐기며, 용서하고 만족하여 참회를 통해 보다 높은 곳으로 나아가는 요가다. 

\paragraph{}
이 밖에도 소리로 수련하는 만트라요가, 몸과 마음을 하나의 원동력으로 수련하는 탄트라 요가, 신경력을 개발하는 쿤달리니요가 등이 있다.


\paragraph{}
자신이 추구하는 성향과 몸에 맞는 요가를 수행하면 그 효과는 더욱 극대화 될 수 있다. 덧붙여, 심신이 허약한 사람은 하타요가, 조용한 성향을 가진 사람은 라자요가를 추천한다.

그 정도로 요가를 하는 남자는 생소하고 어색하다. 대부분의 남자들도 빨래판 같은 근육을 갖기 위해 헬스장에서 무거운 덤벨을 들어올리고 짐승남처럼 땀을 흘리는 운동을 주로 했다. 하지만 현재 우리나라에서 가장 많이 하고 있는 히타요가의 창시자는 바로 남자. 원래 고대 요가는 전사들을 위해 고안된 운동이란 기록이 있어 당시 여자들은 접할 수도 없었다고 한다.

요가는 남자에게도 아주 유용한 운동이란 소리다. 특히, 대부분의 남자들은 근육위주의 운동을 많이 하기 때문에 유연성이 떨어지는 편이다. 또, 여자와 달리 근육이 조밀하고 단단해서 뭉치기도 쉽다. 요가는 그런 뭉친 근육을 풀어주고 신체를 강화시켜준다. 그 동안 요가를 배우고 싶어도 여자들의 시선이 의식되고 부끄러웠다면 이제 당당하게 배우자. 특히, 요가 자세 중 메뚜기 자세는 신장기능과 정력강화에 매우 효과적이다. 



\section{요가의 분류}




\section{요가 ‘제대로’ 잘하는 방법 }



 
\paragraph{1. 오버하지 말자 }
다이어트를 결심한 오늘, 선생님 따라 똑같이 하려다 괜한 욕심이 화를 부를 수 있다. 옛말에 다들 잘 아는 속담이 있다. ‘뱁새가 황새 따라 가려다 가랑이 찢어진다’했다. 요가동작은 서커스처럼 묘기를 부리는 게 아니다. 과유불급이라 했다. 어떤 운동이나 무리해서 하다 보면 근육이나 관절을 다치기 쉬우니까. 요가는 자신의 몸의 유연성을 알고 그 생리에 맞춰 적당하게 몸을 이완 수축시켜 수련하는 것이다. 남들이 하지 못하는 어려운 동작을 어설프게라도 따라 하는 게 맞다는 착각은 금물. 자칫, 근육이나 관절을 지나치게 이완시켜 간과 신장이 상해 건강을 해칠수 도 있다.

\paragraph{2. 내 몸을 알자 }
내 몸이 어떤지는 내가 더 잘 안다. 당장 윗몸 일으키기와 상체 굽히기만 해봐도 유연성을 알 수 있다. 앞서 언급한대로 요가 동작마다 하나하나 우리 몸에 미치는 영향은 다르다. 그러므로 성별, 나이, 습관, 병리적 현상 등을 고려해서 요가 수련 법을 선택해야 한다. 살을 빼야 하는 사람이 비장이나 위장을 강화는 ‘활 체위’를 하면 오히려 입맛이 돌아 살이 더 찔 수도 있다. 그렇기 때문에 사전에 체질에 맞는 요가 동작과 그에 맞는 호흡 명상법을 알아야 한다.

\paragraph{3. 빼먹지 말자 }
요가는 최소 3개월이상 해야 그 효과를 확인할 수 있다. 단순히 똥배를 없애는 운동이 아니기 때문에 처음 한 두 달은 요가를 내 몸에 익히고 적응시키는 기간이기 필요하다. 그러니 살이 안 빠진다고 조바심내지 말자. 3개월 이후부터는 그 동안 익힌 요가 동작이 자연스러워지고 근육들이 반복적으로 자극됨으로써 쓸데없는 군살은 없어지고 몸매가 다듬어진다.

\paragraph{4. 글로 배우지 말자. }
요가를 오랫동안 수련한 사람이 아니라면 처음부터 책이나 비디오만 보고 혼자서 요가를 하면 자신이 제대로 동작을 따라 하고 있는지 알 수 없다. 게다가 혼자서 하기 때문에 여러 사람과 운동을 함께 할 때보다 동기부여나 운동에 대한 자극이 적다. 또, 강제성이 없기 때문에 긴장이 풀려 집중하기도 어렵다. 스스로 운동 컨트롤이 힘든 사람이라면 당연한 얘기지만 전문가한테 지도를 받는 게 좋다.

\paragraph{}
그 밖에 식사 후 바로 요가를 하면 좋지 않다. 배가 부른 상태에서는 혈액순환이 잘 되지 않기 때문에 위장에 무리를 줄 수 있고, 배가 너무 고프면 피곤해질 수 있다. 식사 후 2시간 정도가 적당하다.




%	==========================================================================
	\chapter{아쉬탕가 요가}
	\minitoc

% -----------------------------------------------------------------------------
%
% -----------------------------------------------------------------------------
	\section{아쉬탕가 요가}


	\subsection{아쉬탕가 요가}

%		\begin{itemize}[label=\arabic*), topsep=0.0em, itemsep=0.0em ]
		\begin{itemize}
			\item 	아쉬탕가요가란 무엇인가?
			\item 	아쉬탕가 요가를 이해하는데 있어서 가장 먼저 우리가 먼저 해야할 것이 바로 요가철학에에서의 아쉬탕가 요가, 동작(아사나) 스타일에서의 아쉬탕가 요가를 분리해서 이해해야 한다는 것이다. 이 두개를 분리하지 않으면 평생가도 이해할 수 없게 된다.
실제로 많은 웹사이트들이 동작(아사나)에서의 아쉬탕가 빈야사 요가를 설명하면서 내용은 요가 철학에서의 아쉬탕가 요가 이야기를 담고 있다. 이것은 마치 라면에 대해서 설명하고자 하면서 밀가루란 무엇인가를 설명하는 것과 같다.

			\item 	요가를 공부하는 사람이 일어야 할 필독서가 세가지 있다.
바가바드 기타, 하타요가 프라디피카, 파탄잘리 수트라 이다.
			\item 	파탄잘리 수트라를 따르는 요가를 라자요가라 한다.
			\item 	이 파탄잘리 수트라는 우리가 어떻게 요가를 수련해 나가랴 하는지 8단계로 나누어 설명한 책인데, 요가철학 이야기와 요가에서의 8 단계에 관한 설명이 있을뿐 어떻게 동작(아사나)을 수련하고 어떻게 호흡하고 구체적인 내용은 없다.
바로 이 8단계 요가를 아쉬당카 요가라 부른다. (아쉬타=숫자 8을 의미한다)

		\end{itemize}



	\subsection{아쉬탕가 빈야사 요가}

		\begin{itemize}
			\item 	아쉬탕가 빈야사 요가는 파타비 조이스에 의해서 만들어진 하나의 요가수타일이다.

			\item 	아쉬탕가 빈야사 요가는 순서가 존재한다.
선생님이 원하는대로 수련하는게 아니라 정해진 순서에 맟추어 매일 반복하게 된다.
그 순서가 총 6개의 시리즈로 되어 있으며 첫번째의 시리즈를 마쳐야 두번 째 시리즈로 넘어갈 수 있게 되어있다.

1973년 낸시 길고프와 데이비드 월리엄스가 인도여행 중 파타비 조이스의 아들 만주 조이스를 만나게 된다.
만주 조이스는 요가 홍보 투어 중이었고 만주 조이스가 요가동작을 시연하는 것을 두 사람은 보게 된다.
만주 조이스는 이것을 배우고 시파어하게 된 두 사람을 자신의 아버지 파타비 조이스에게 데려간다.
그들은 파타비 조이스의 제자가 되었고, 그 외에 데이비드 스완슨 등이 1970년대부터 아쉬탕가 빈야사 요가를 배우게 된다.
그들에 의해 아쉬탕가 빈야사 요가는 전세계로 퍼져나갔다.

		\end{itemize}


	\subsection{잘못 알려진 아쉬탕가 빈야사 요가 상식들}

아쉬탕가 빈야사 요가의 가장 큰 특징은 반다, 우짜이 호흡, 드리시티이다.

반다

우자이 호흡


%	==========================================================================
	\chapter{인사이트 요가}
	\minitoc

% -----------------------------------------------------------------------------
%
% -----------------------------------------------------------------------------
	\section{인사이트요가}

	사라 파워스 지음\\
	서 영조 올김\\
	풀 그릴리 서문\\
	문지영 감수\\

		\begin{flushright}
		전통적인 요가에 중국 의학의 경락 이론\\
		불교의 마음챙김 명상을 결합한 요가 수련법, 인사이트 요가! \\
		몸과 정신의 균형있고 조화로운 수련을 통해 삶의 에너지를 깨운다.
		\end{flushright}

1. 요가란 무었인가?
2. 내가 걸어온 요가여정
3. 경락 이론
4. 인양요가 수련 시작하기
5 인요가와 장기의 건강
6. 신장과 방광
7. 신장과 방광의 건강을 위한 인요가
8. 간과 쓸개
9. 간과 쓸개의 건강을 위한 인요가
10. 비장과 위
11. 비장과 위의 건강을 위한 인요가
12. 폐와 대장
13. 심장과 소장
14. 폐 심장 대장 소장의 건강을 위한 인요가
15. 인요가와 마음 작용
17. 인요가 수련과 균형을 이루기 위한 양요가 프로그램
18. 프라나야마
19. 불교와 마음팽김
20. 마음챙김 명상 수련
요가 프로그램구성에 대한 제안
요가 자세 찾아보기


	\subsection{1. 요가란 무었인가?}
	\subsection{2. 내가 걸어온 요가여정}
	\subsection{3. 경락 이론}
	\subsection{4. 인양요가 수련 시작하기}
	\subsection{5 인요가와 장기의 건강}
	\subsection{6. 신장과 방광}
	\subsection{7. 신장과 방광의 건강을 위한 인요가}
	\subsection{8. 간과 쓸개}
	\subsection{9. 간과 쓸개의 건강을 위한 인요가}
	\subsection{10. 비장과 위}
	\subsection{11. 비장과 위의 건강을 위한 인요가}
	\subsection{12. 폐와 대장}
	\subsection{13. 심장과 소장}
	\subsection{14. 폐 심장 대장 소장의 건강을 위한 인요가}
	\subsection{15. 인요가와 마음 작용}
	\subsection{17. 인요가 수련과 균형을 이루기 위한 양요가 프로그램}
	\subsection{18. 프라나야마}
	\subsection{19. 불교와 마음팽김}
	\subsection{20. 마음챙김 명상 수련}
	\subsection{요가 프로그램구성에 대한 제안}
	\subsection{요가 자세 찾아보기}




%	================================================================== chapter
	\chapter{치유 요가}
	\minitoc	


%	================================================================== chapter
	\chapter{요가의 수련 8단계}
	\minitoc	

% -----------------------------------------------------------------------------
%
% -----------------------------------------------------------------------------
	\section{요가의 수련 8단계}

	\subsection{요가의 수련 8단계}

	\section{쿰바카, 반다, 나다, 차크라}

	\section{프라나야마 (호흡법)}






% -----------------------------------------------------------------------------
%
%
%
% -----------------------------------------------------------------------------




%	================================================================== Part			아사나
	\addtocontents{toc}{\protect\newpage}
	\part{아사나}
		\noptcrule
		\parttoc				



% ========================================================================================= chapter 		호흡
	\chapter{호흡}


	% ----------------------------------------------------------------------------- 	호흡법
	%
	% -----------------------------------------------------------------------------
		\section{호흡법}

		\subsection{근육 운동 시 }
일반적으로 힘을 줄때 숨을 내 쉰다.
예를 들어 무거운 중량을 밀어 올리거나 권투선수가 펀치를 뻗을때 숨을 뱉으면서 동작을 해야 큰 힘이 생긴다.
더불어 운동 중에 기합을 넣으면 중추신겨을 자극하여 큰 힘을 발휘할 수 있다


		\subsection{스트레칭 시 }
몸을 늘릴 때 숨을 내쉰다.
숨을 내쉬면서 스트레칭을 해야 몸이 이완되는 효과를 얻을 수 있다.




	% ----------------------------------------------------------------------------- 	우짜이 호흡	
	%
	% -----------------------------------------------------------------------------
		\newpage
		\section{우짜이 호흡}


		먼저 편하게 앉아서 코로 천천히 숨을 들이마신다.
숨을 내쉴 때는 자연스럽게 입을 벌려 "하아 " 소리 내면 밲는다.
이 방법을 몇 차례 연습해서 편해지면 입이 아닌 코로 숨을 내쉰다.
들이 마시는 숨과 내쉬는 숨의 길이는 같고, 얕은 숨소리가 동일하게 나도록 호흡한다.

이 호흡법은 마음에 안정을 주어 요가 동작을 더욱 견고하게 만든다.



	% ----------------------------------------------------------------------------- 
	%
	% ----------------------------------------------------------------------------- 요가반다
		\section{요가 반다}

요가에서 반다는 인체의 어떤 조직이나 부분을 수축시키거나 조이는 자세이다





		\tikz	[		mindmap,
						align=flush center, 
						every node/.style=concept,
						concept color=black,
						grow cyclic,
						concept/.append style={fill={none}},
						level 1/.append style={level distance=4.5cm,sibling angle=45 },
						level 2/.append style={level distance=4.0cm,sibling angle=00 },
						level 3/.append style={level distance=6.0cm,sibling angle=00 }
					]
			\node 	[concept] 		{반다 (잠금)}  
			[clockwise from=-45]
			child		{node[concept] {항문 }
					child	[grow=down]	{node[concept] {물라 }}
					}  
			child		{node[concept] {배꼽}
					child	[grow=down]	{node[concept] {우다아나 }}
					} 
			child		{node[concept] {목}
					child	[grow=down]	{node[concept] {잘란다라 }}
					}  ;

\paragraph{물라 반다} 항문과 음낭 사이를 조인다

\paragraph{우티아나 반다} 복부의 아랫쪽을 수축시켜 조여준다.

\paragraph{잘란다라  반다}
턱이 들리지 않도록 쇄골쪽으로 끌어당긴다.





	% ----------------------------------------------------------------------------- 
	%
	% ----------------------------------------------------------------------------- 요가우디야나반다
		\section{요가 우디야나 반다}



		\tikz {\draw[line width=0.8mm] rectangle node{1} (2,2) }	\tikz {\draw[line width=0.4mm] rectangle node{반드시 공복에 실시 } (12,2) }\\
		\tikz {\draw[line width=0.8mm] rectangle node{2} (2,2) }	\tikz {\draw[line width=0.4mm] rectangle node{숨을 내쉰후에 개시해야 한다  } (12,2) } \\
		\tikz {\draw[line width=0.8mm] rectangle node{} (2,2) }	\tikz {\draw[line width=0.4mm] rectangle node{숨을 들이쉬기전에 시작해선 안된다   } (12,2) } \\
		\tikz {\draw[line width=0.8mm] rectangle node{3} (2,2) }	\tikz {\draw[line width=0.4mm] rectangle node{선자세에서 우디야나 반다를 수련할 것을 권한다} (12,2) } 	\\


''우디야나 반다''를 유지하는 동안 ''잘란다라 반다''를 취해도 좋다.





% ========================================================================================= chapter
	\chapter{아사나 일반}
	\minitoc


	% -----------------------------------------------------------------------------
	%
	% -----------------------------------------------------------------------------
		\section{아사나 일반}

괄약근 조으고
배집어 넣고
하늘에서 누가 당긴다.




	% -----------------------------------------------------------------------------
	%
	% -----------------------------------------------------------------------------
		\section{아사나}



아사나는 요가의 체위를 말한다. 
인도어의 원뜻은 "앉는다"는 뜻이다. 
좌법이라고도 한다. 
요가 수트라의 8단계 중 3단계 수행법이다. 
수십가지의 아사나가 있다.

\paragraph{} 아사나는 요가 좌법과 체위들이며, 바른 몸과 명상을 위한 기초가 되는 자세들이다.
아사나에는 의미 있다.
심리적이고 영적인 의미가 있는 것이다.
수행자는 이제 자세를 상징으로 보고 느끼면서 그로부터 분명한 의미를 발견하게 될 것이다.

\paragraph{} 육체는 욕망을 추구하고 의지를 행사하기 위한 도구이다. 하타요가의 수련과 아사나들의 상징적 의미에 대한 명상을 통해 육체를 제어할 수 있게 되면 결국 감정과 지성도 제어할 수 있게 될 것이다.
이로써 아사나는 심신의 건강, 무병, 쾌적함을 준다.





% -----------------------------------------------------------------------------
%
%
%
% -----------------------------------------------------------------------------
\newpage
\section{아사나의 구분 및 분류 }

아사나들은 그 이름이 동물,식물, 새, 구조물 등의 그룹으로  나뉜다.
어떤 아사나의 이름은 그 아사나의 상징적인 원리를  찾기 위한 출발점이다.


	\href{http://yogaman.co.kr/wordpress/yoga/yogaasana_name_search.php}{요가 아사나의 구분}


	%	------------------------------------------------------------------------------  table

			\begin{table} [h]
	
			\caption{요가 아사나}  
			\label{tab:title} 
	
			\begin{center}
			\tabulinesep=0.4em
			\begin{tabu} to 0.8\linewidth { X[r] X[l] X[c]  }
			\tabucline [1pt,] {-}
			한글 명칭 & 명칭 	&비고\\
			\tabucline [0.1pt,] {-}
산 자세&&\\
나무 자세&&\\
\tabucline [1pt,] {-}
서서 상체 숙이기 자세&&\\
서서 다리벌려 상체 숙이기 자세&&\\

\tabucline [1pt,] {-}
의자 자세&&\\
전사 자세Ⅰ&&\\
전사 자세Ⅱ&&\\
전사 자세Ⅲ&&\\

춤의 여왕 자세&&\\

독수리 자세&&\\
반달 자세&아르다 챤드라사나&\\

낙타 자세&&\\
토끼 자세&&\\
고양이 자세A&&\\
고양이 자세B&&\\
뱀 자세&&\\
악어 자세&&\\

메뚜기 자세 &살람바사나&\\

스핑크스 자세&나라비랄라사나&\\

상체 뒤로 젖히는 자세&&\\
지마 자세&&\\
십자가 자세&&\\
엎드려서 고개 숙인 견 자세&&\\

			\tabucline [0.1pt,] {-}
			\end{tabu} 
			\end{center}
			\end{table}






\section{아쉬탕가요가  ASHTANGA PRIMARY SERIES }


ASHTANGA PRIMARY SERIES A1 POSTER


	% -----------------------------------------------------------------------------
	%
	% -----------------------------------------------------------------------------
		\section{스쿼트}

% ========================================================================================= chapter 서서하는아사나 
	\chapter{ 서서하는 아사나 : STANDING POSES 아사나}
	\newpage
	\minitoc




		\section{서서하는 아사나}
산 자세
나무 자세
서서 상체 숙이기 자세
서서 다리벌려 상체 숙이기 자세
의자 자세
전사 자세Ⅰ
전사 자세Ⅱ
전사 자세Ⅲ
춤의 여왕 자세
독수리 자세

		\section{산 자세}
		\section{나무 자세 : 브르크사사나}
		\section{서서 상체 숙이기 자세}
		\section{서서 다리 벌려 상체 숙이기 자세}
		\section{의자 자세}

		\section{전사 자세Ⅰ : 바라바드라사나}
		\section{전사 자세Ⅱ : }
		\section{전사 자세Ⅲ}

		\section{춤의 여왕 자세}
		\section{독수리 자세}

		\section{한다리 엄지발가락 잡고 옆으로 들어올리기 자세 B: 우티사 하스타 파당거스트하사나 B}




	\section{PRASARITA PADOTTANASANA}

	Prasarita Padottanasana
	
	쁘라사리따 빠돋따나사나
	
	프리사리타 파도타나사나
	
	
	
	효과
	
	다리의 오금근과 외전근이 발달
	
	머리, 목, 몸통, 뇌하수체, 송과선, 갑상선, 부갑상선, 흉선에 혈액순환을 향상
	
	발목, 손목 관절과 다리를 강화, 넓적다리의 지방을 감소
	
	복부 기관과 골반 기관, 생식선의 울혈을 감소, 소화력이 증진(소장과 대장에 아주 좋은 자세)
	
	폐의 분비물(담)을 배출
	
	육체적 정신적 피로를 감소
	
	시르사사나가 되지 않는 사람이 대신할 수 있는 자세
	
	
	
	넓적다리 비만을 치유해주고
	
	집중력 부족을 개선해주며
	
	약한 기억력 또한 좋게 만들어 준다
	
	골반 기관의 질환(방광, 자궁, 전립선), 여성의 생리 질환, 생식선 질환을 개선해준다
	
	불면증, 편두통, 정신적 스트레스 또한 완화
	
	
	
	이제 사진으로 자세를 볼까요~?!
	
	

% ========================================================================================= chapter
	\chapter{앉아서 하는 아사나 : SEATED POSES 아사나}
	\newpage
	\minitoc


	% -----------------------------------------------------------------------------
	%
	% -----------------------------------------------------------------------------
		\section{앉아서 하는 아사나 }

	앉아서 상체 숙이기 자세
	한 다리 구부리고 머리와 무릎 대는 자세
	한 다리를 바깥쪽으로 구부려어 상체 숙이기 자세
	마리챠 자세Ⅰ
	마리챠 자세Ⅱ
	마리챠 자세Ⅲ
	반가부좌로 상체 숙이기 자세
	상체를 측면으로 회전하여 기울이는 자세
	영웅 자세
	소머리 자세

	% -----------------------------------------------------------------------------
	%
	% ----------------------------------------------------------------------------- 타나사나산자세
		\section{타다사나 - 산자세}


	% -----------------------------------------------------------------------------
	%
	% -----------------------------------------------------------------------------
		\section{	앉아서 상체 숙이기 자세}
		\section{	한 다리 구부리고 머리와 무릎 대는 자세}
		\section{	한 다리를 바깥쪽으로 구부려어 상체 숙이기 자세}
		\section{	마리챠 자세Ⅰ}
		\section{	마리챠 자세Ⅱ}
		\section{	마리챠 자세Ⅲ}
		\section{	반가부좌로 상체 숙이기 자세}
		\section{	상체를 측면으로 회전하여 기울이는 자세}
		\section{	영웅 자세}
		\section{	소머리 자세}


% ========================================================================================= chapter
	\chapter{FLOOR POSES 아사나}
	\newpage
	\minitoc

	낙타 자세
	토끼 자세
	고양이 자세A
	고양이 자세B
	뱀 자세

	상체 뒤로 젖히는 자세
	지마 자세
	십자가 자세
	악어 자세
	엎드려서 고개 숙인 견 자세

	% -----------------------------------------------------------------------------
	%
	% ----------------------------------------------------------------------------- 나비자세
		\section{요가무드라}
		\section{나비 자세}


	% -----------------------------------------------------------------------------
	%
	% ----------------------------------------------------------------------------- 개구리자세
		\section{ - 개구리 자세 }


	% -----------------------------------------------------------------------------
	%
	% ----------------------------------------------------------------------------- 고양이자세
		\section{ - 고양이 자세 }
	
	
	% -----------------------------------------------------------------------------
	%
	% ----------------------------------------------------------------------------- 코브라자세
		\section{ - 코브라자세 }
	
	% -----------------------------------------------------------------------------
	%
	% -----------------------------------------------------------------------------
		\section{ -  메뚜기 자세}
	
	% -----------------------------------------------------------------------------
	%
	% -----------------------------------------------------------------------------
		\section{ - 물고기 자세}
	
	% -----------------------------------------------------------------------------
	%	낙타
	% -----------------------------------------------------------------------------
		\section{  낙타 자세 : 우스트라 아사나}
	


	% -----------------------------------------------------------------------------
	%	거북이
	% -----------------------------------------------------------------------------
		\section{ 거북이 자세 : 꾸르마 아사나 }

		\section{ 잠자는 거북이 자세 : 숲타 꾸르마 아사나 }


	% -----------------------------------------------------------------------------
	%	개
	% -----------------------------------------------------------------------------
		\section{ 위로 얼굴 개자세 : 우르드바 무카 스바나 아사나 }



% ========================================================================================= chapter
	\chapter{PRONE SUPINE POSES 아사나}
	\newpage
	\minitoc


	활 자세
	메뚜기 자세
	바람빼기 자세
	누워서 엄지발가락 잡고 다리 들어올리는 자세
	누워서 비둘기 자세


		\section{ 	활 자세}
		\section{ 	메뚜기 자세}
		\section{ 	바람빼기 자세}
		\section{ 	누워서 엄지발가락 잡고 다리 들어올리는 자세}
		\section{ 	누워서 비둘기 자세}

% ========================================================================================= chapter
	\chapter{영웅 자세 : 비라 아사나}
	\newpage
	\minitoc


	누운 영웅 자세


	% -----------------------------------------------------------------------------
	%
	% -----------------------------------------------------------------------------
	\section{비라사나 - 앉은 영웅자세}


	% -----------------------------------------------------------------------------
	%
	% -----------------------------------------------------------------------------
	\section{ - 영웅 자세1 }
	
	% -----------------------------------------------------------------------------
	%
	% -----------------------------------------------------------------------------
	\section{ - 영웅 자세2}
	




% ========================================================================================= chapter
	\chapter{ARM BALANCING INVERSIONS POSES 아사나}
	\newpage
	\minitoc


	까마귀 자세
	두루미 자세
	지팡이 자세
	토운단야 자세
	개똥벌레 자세
	앞뒤로 흔들기 자세
	어깨를 감싸안아 버티는 자세
	코운단야 자세
	바시슈타 자세
	비쉬바미트라 자세
	...

% ========================================================================================= chapter
	\chapter{RELAXATION POSES 아사나}
	\newpage
	\minitoc


	시체 자세
	누운 나비 자세
	어린이 자세

	\section{시체 자세}
	\section{	누운 나비 자세}
	\section{	어린이 자세}


% -----------------------------------------------------------------------------
%
% -----------------------------------------------------------------------------
	\section{파드마사나 - 연꽃자세}



% -----------------------------------------------------------------------------
%
% -----------------------------------------------------------------------------
	\section{단다아사나}


\subsection{사진}  단다 danda 는 막대기 또는 장대를 의미한다.


\subsection{방법} 아사나는 요가 좌법과 체위들이며, 바른 몸과 명상을 위한 기초가 되는 자세들이다.
	\paragraph{방법} 아사나는 요가 좌법과 체위들이며, 바른 몸과 명상을 위한 기초가 되는 자세들이다.
	\paragraph{효과} 아사나는 요가 좌법과 체위들이며, 바른 몸과 명상을 위한 기초가 되는 자세들이다.


아사나는 요가 좌법과 체위들이며, 바른 몸과 명상을 위한 기초가 되는 자세들이다.



단다아사나란 산스크리트어 단다(Danda; 지팡이)와 아사나(Asana; 자세·체위)가 합쳐진, 말 그대로 「지팡이 자세」다. 이는 앉아서 수련하는 모든 자세의 준비자세이다. 

\subsection{효과}

척추를 펴고 자세를 바로 잡아줌으로써 측만증, 요추, 경추 등  디스크 이상을 예방해준다. 
또 가슴이 펴지면서 호흡이 안정되며, 장기의 울혈(鬱血; 혈관의 일부에 정맥성 혈액의 양이 증가되어 있는 상태. 정맥의 협착(狹窄)이나 폐색(閉塞)이 원인이 됨)을 제거해 소화를 돕고, 마음을 편하게 한다. 
자율신경 조절력을 향상시키는 데도 도움이 된다.
단 요추나 경추디스크가 이미 있는 사람의 경우 척추에 무리를 줄 수 있으므로 이 자세는 금하도록 한다.


\subsection{따라하기}
① 머리와 목을 곧게 하고 등을 편다. 가슴을 펴고, 어깨는 열고, 복부는 
내밀지 않고 바르게 앉는다. 엉덩이뼈가 편안함을 유지하도록 양쪽 엉덩이 
근육을 손으로 당겨 바깥쪽으로 보내서 앉는다.
② 두 다리를 죽 펴고 발목을 수직으로 당긴 상태에서 엄지발가락, 발목, 
무릎, 허벅지(대퇴부)안쪽을 최대한 바짝 붙이고, 오금(무릎의 구부러지는 
부분 바로 뒤쪽)도 가능한한 바닥에 바짝 붙이도록 노력한다. 
③ 양 손의 손가락을 가지런히 모아서 손끝을 발끝쪽으로 향하게 한 상태
에서 손바닥을 엉덩이쪽에 둔다. 이는 지팡이자세에서 손바닥이나 팔의 힘
을 사용하지 않게 하기 위해서다.
④ 등은 곧게 펴고 턱을 당겨 가슴에 붙이고 호흡을 한다. 시선은 정면이
나 엄지발가락쪽을 향하고, 얼굴과 눈의 긴장을 푼채 편안한 상태를 유지하
려고 한다.
⑤ 의식은 자극부위를 관찰하면서 집중을 하여 1~2분 정도 이 자세를 유지
한다.
⑥ 수련이 끝나면 앉은 자세에서 다리를 45°정도 벌려서 긴장을 풀고 손
은 엉덩이 뒤쪽 바닥에 두고 쉰다.

\subsection{일상에서의 응용}
발목이 굳은 사람이나 오금이 경직돼서 자세가 잘 되지 않는 사람은 발바닥을 벽에 붙이고, 등이 굽은 사람은 등과 엉덩이를 벽에 붙여서 자세를 취한다.
또 앉아서 보내는 시간이 많은 사람(학생, 직장인)은 이 자세를 응용해서 엉덩이를 의자 깊숙이 앉고 등을 곧게 세우는 습관을 들이면 좋다. 








% -----------------------------------------------------------------------------
%
%
%
% -----------------------------------------------------------------------------
\section{사바사나 - 열반자세}



% -----------------------------------------------------------------------------
%
%
%
% -----------------------------------------------------------------------------
\section{요가무드라 - 어깨 수정 아사나}



% -----------------------------------------------------------------------------
%
%
%
% -----------------------------------------------------------------------------
\section{요가무드라 - 앉은 산자세}

% -----------------------------------------------------------------------------
%
%
%
% -----------------------------------------------------------------------------
\section{요가무드라 - 소머리 자세}


% -----------------------------------------------------------------------------
%
%
%
% -----------------------------------------------------------------------------
\section{요가무드라 - 현자세}


% -----------------------------------------------------------------------------
%
%
%
% -----------------------------------------------------------------------------
	\section{요가무드라 - 앉은서 좌우 기울기}


% -----------------------------------------------------------------------------
%
%
%
% -----------------------------------------------------------------------------
	\section{요가무드라 - 전굴 자세}


% -----------------------------------------------------------------------------
%
%
%
% -----------------------------------------------------------------------------
	\section{자루 시르사아사나 - }

% -----------------------------------------------------------------------------
%
%
%
% -----------------------------------------------------------------------------
	\section{트리앙카 무카이카파타 파스치모타아사나 - }

% -----------------------------------------------------------------------------
%
%
%
% -----------------------------------------------------------------------------
	\section{아르다 받다 파드마 파스치모타나아사나 - }

% -----------------------------------------------------------------------------
%
%
%
% -----------------------------------------------------------------------------
	\section{누워서 굽히는 자세}


% -----------------------------------------------------------------------------
%
%
%
% -----------------------------------------------------------------------------
	\section{ - 서서 굽히는 자세}


% -----------------------------------------------------------------------------
%
%
%
% -----------------------------------------------------------------------------
	\section{파르스보타나 아사나 - }

% -----------------------------------------------------------------------------
%
%
%
% -----------------------------------------------------------------------------
	\section{ - 견상 자세 }


%	================================================================== chapter
	\chapter{삼각자세}
	\newpage
	\minitoc

		% -----------------------------------------------------------------------------
		%
		% -----------------------------------------------------------------------------
			\section{ - 삼각자세 }

			우티타 트리코나 아사나


		우티타 : 쭉 뻗은
		트리코나 : 삼각형



		% -----------------------------------------------------------------------------
		%
		% -----------------------------------------------------------------------------
			\section{ - 기운 삼각 자세 }
		
		% -----------------------------------------------------------------------------
		%
		% -----------------------------------------------------------------------------
			\section{ 역 삼각 자세 }
		
		
		% -----------------------------------------------------------------------------
		%
		% -----------------------------------------------------------------------------
			\section{  비튼 삼각자세 }


% -----------------------------------------------------------------------------
%
%
%
% -----------------------------------------------------------------------------
	\section{ - 척추 좌우 비틀기}



% -----------------------------------------------------------------------------
%
%
%
% -----------------------------------------------------------------------------
	\section{ - 누워서 다리 좌우 기울기}



% -----------------------------------------------------------------------------
%
%
%
% -----------------------------------------------------------------------------
	\section{비라드바자 아사나 - }

% -----------------------------------------------------------------------------
%
%
%
% -----------------------------------------------------------------------------
	\section{ - 서서 좌우 기울기 }

% -----------------------------------------------------------------------------
%
%
%
% -----------------------------------------------------------------------------
	\section{ - 낚시 자세 }

% -----------------------------------------------------------------------------
%
%
%
% -----------------------------------------------------------------------------
	\section{파리가아사나 - }



% -----------------------------------------------------------------------------
%
%
%
% -----------------------------------------------------------------------------
	\section{ - 활자세}

% -----------------------------------------------------------------------------
%
%
%
% -----------------------------------------------------------------------------
	\section{ - 누워서 다리들어올리기}

% -----------------------------------------------------------------------------
%
%
%
% -----------------------------------------------------------------------------
	\section{ - 한쪽 무릎 감아올리기}

% -----------------------------------------------------------------------------
%
%
%
% -----------------------------------------------------------------------------
	\section{ - 보트자세}

% -----------------------------------------------------------------------------
%
%
%
% -----------------------------------------------------------------------------
	\section{우바야 파당구 쉬타 아사나 - }

% -----------------------------------------------------------------------------
%
%
%
% -----------------------------------------------------------------------------
	\section{ - 어깨서기}


% -----------------------------------------------------------------------------
%
%
%
% -----------------------------------------------------------------------------
	\section{ - 쟁기자세}


% ========================================================================================= chapter
	\chapter{물구나무 서기 : 헤드 스탠드 }
	\newpage
	\minitoc


% -----------------------------------------------------------------------------
%
% -----------------------------------------------------------------------------
	\section{ 물구나무서기 }


% -----------------------------------------------------------------------------
%
% -----------------------------------------------------------------------------
	\section{ 머리를 받침 물구나무 서기 - 살람바 시르사아사나}


\paragraph{단계별 수련법}


			\begin{itemize}[topsep=0.0em, parsep=0.0em, itemsep=0em, leftmargin=6.0em, labelwidth=3em, labelsep=3em] 
			\item [1.] 머리와 팔뚝 아래를 받칠 접은 담요나 스티키 매트를 준비한다.
					바닥에 무릎을 꿁는다.
					손가락을 깍지 끼고 바닥에 팔뚝을 내려놓는다.
					팔꿈치는 어깨너비로 벌린다.
					상완을 살짝 바같족으로 돌리고, 손목 안쪽은 바닥을 힘껏 누른다.
					정수리를 바닥에 댄다.
					이 자세가 처음이라면 깍지 낀 손안에 뒤통수를 대자.
					상급자는 깍지를 풀고 활짝 편 손바닥에 뒤통수를 댄다.
			\item [2.] 
					숨을 들이쉬며 바닥에세 무릎을 든다. 팔굼치 근처로 양발을 조심스럽게 이동시킨다.
					뒷꿈치는 든다.
					넓적다리 상단을 힘껏들어서 뒤집어지 V자를 만든다.
					등에 견갑골을 붙이고, 상체 앞쪽을 최대한 늘려 준다.
					그러면 어깨의 무게가 목과 머리에 부담을 주는 걸 방지할 수 있다.
			\item [3.] 
					숨을 내쉬며 바닥에서 발을 든다.
					양발을 동시에 들어야 한다.
					무릎을 살짝 굽혀서 바닥에서 가볍게 점프를 해도 좋다. 
					다리가 바닥과 수직을 이루면 골반 뒤쪽에 꼬리뼈를 단단히 붙인다.
					넓적다리 상단을 안으로 살짝 돌리고, 천장을 향해 뒤꿈치를 힘껏 민다(무릅을 굽히고 있으면 펴자).
					발바닥 중앙이 골반 중앙, 정수리와 동일선상에 놓여야 한다.
			\item [4.] 
					팔 바깥쪽을 안으로 밀고, 손가락의 힘을 푼다.
					견갑골로 계속 등을 누르고, 견갑공을 펼쳐서 꼬리뼈를 향해 당긴다.
					양쪽 팔뚝에 체중을 고르게 분산하다.
					뒤꿈치를 향해 꼬리뼈를 계속 드는 것도 중요하다.
					다리 뒤쪽을 완전히 폈므연 그 상태를 유지하면서 엄지발가락을 위로 밀자.
					다리 안쪽이 바깥쪽 보다 살짝 길어야 한다.

			\item [5.] 
					초보자는 10초간 머무른다.
					점차 5`10초씩 수련 시간을 늘려나가서 3분까지 버텨 보자.
					그리고 1~2주 동안 매일 3분씩 수현하면 자세가 편한해질 적이다.
					그러면 다시 5~10초씩 술현ㅇ 시간을 늘려 나가서 5분까지 버텨 보자.
					숨을 내쉬면 내려온다.
					양발이 바닥에 동시에 닿아야 한다.





			\end{itemize}


\paragraph{산스크리트어 명칭}

			살람바 사르사아사나
			삼람바=받치다\\
			사르사=머리\\





\paragraph{자세 난이도}
1

\paragraph{다음증상이 있으면 수련하지 않는게 좋다}

			\begin{itemize}[topsep=0.0em, parsep=0.0em, itemsep=0em, leftmargin=6.0em, labelwidth=3em, labelsep=3em] 
			\item
					등부상
			\item
					두통
			\item
					심장질환
			\item
					월경
			\item
					목 부상
			\item
					저혈압이 있으면 이 자세를 시컨스 맨 처름에 배치하지 말자.
			\item
					임산부 이 자세에 숙달된 사람은 출산 전 까지 이 자세를 수련해도 좋다.
					하지만 임신 후에 이 자세를 처음 배우는 건 좋지 않다.
			\item
					시르사아사나는 중상급자를 위한 자세다.
					경험이 부족하거나 숙력된 지도자의 도움을 받을 수 없다면 수련하지 말자.
			\item
					시르사아사나을 실시한 후에 사방가아사나을 수련해야 한다는 지도자도 있고, 그와 반대의 주장을 하는 지도자도 있다.
					여기선 전자를 따랐다.
			\end{itemize}

\paragraph{자세 수정하기 소품 활용하기}

			처음엔 균형 잡기가 힘든 자세이다.
			벽에 몸을 기대고 실시해 보자.
			깍지 낀 손의 손등을 벽에 대자
			가능하다면 방구석에서 실시하는게 좋다.
			양쪽 벽이 어깨와 골반, 뒤꿈치 바깥족에ㅐ 닿도록 말이다.



\paragraph{자세에 깊이 더하기}

			손목 안쪽의 위치를  체크하자. 
			바깥쪽으로 내려가서 팔뚝 바깥쪽으로 체중이 쏠리는 경향이 있다.
			뒤통수 반대쪽으로 새끼 손가락을 당기고, 손목 안쪽을 바닥과 수직으로 만들자.
			상완 바깥쪽을 안으로 미는 동시에 손목으로 바닥을 힘껏 누르자



\paragraph{준비 자세}

			\begin{itemize}[topsep=0.0em, parsep=0.0em, itemsep=0em, leftmargin=6.0em, labelwidth=3em, labelsep=3em] 
			\item 아도 무카 스바나아사나
			\item 살람바 사르방가아사나
			\item 욷타나아사나
			\item 비라아사나
			\end{itemize}



\paragraph{이어서 하면 좋은 자세}


\paragraph{초보자를 위한 팁}

\paragraph{효능}

\paragraph{파트너와 함께 하기}


\paragraph{변형 동작}
















\paragraph{출처}

	\href{http://www.yogajournal.kr/news/articleView.html?idxno=188}
		{요가저널}






	% -----------------------------------------------------------------------------
	%
	% -----------------------------------------------------------------------------
		\section{헤드 스텐드 기본자세}

	% -----------------------------------------------------------------------------
	%
	% -----------------------------------------------------------------------------
		\subsection{손모양}
		\subsection{머리 위치 }
		\subsection{어깨, 등 모양}
		\subsection{다리 모양}
		\subsection{허리 모양}
		\subsection{호흡법}
		\subsection{유지 시간}

% ========================================================================================= chapter
	\chapter{자세}
	\newpage
	\minitoc

% -----------------------------------------------------------------------------
%
% -----------------------------------------------------------------------------
	\section{ - 다리자세}

% -----------------------------------------------------------------------------
%
% -----------------------------------------------------------------------------
	\section{ - 나무 자세}

% -----------------------------------------------------------------------------
%
% -----------------------------------------------------------------------------
	\section{ - 독수리 자세}

% -----------------------------------------------------------------------------
%
% -----------------------------------------------------------------------------
	\section{ - 반달 자세}

% -----------------------------------------------------------------------------
%
% -----------------------------------------------------------------------------
	\section{아난타아사나 - }


% -----------------------------------------------------------------------------
%
% -----------------------------------------------------------------------------
	\section{ - 영웅자세 3}

% -----------------------------------------------------------------------------
%
% -----------------------------------------------------------------------------
	\section{ - 영웅자세3}



% -----------------------------------------------------------------------------
%
% -----------------------------------------------------------------------------
	\section{욷 카타아사나 - }

% -----------------------------------------------------------------------------
%
% -----------------------------------------------------------------------------
	\section{푸르보타나아사나 - }


% -----------------------------------------------------------------------------
%
% -----------------------------------------------------------------------------
	\section{숩타비라아사나 - }



% -----------------------------------------------------------------------------
%
% -----------------------------------------------------------------------------
	\section{프리사리타 파도타나아사나 - }

% -----------------------------------------------------------------------------
%
% -----------------------------------------------------------------------------
	\section{ - 박쥐 자세 }


% -----------------------------------------------------------------------------
%
% -----------------------------------------------------------------------------
	\section{ - 두루미 자세}


% -----------------------------------------------------------------------------
%
% -----------------------------------------------------------------------------
	\section{ - 역 활자세 }

% -----------------------------------------------------------------------------
%
% -----------------------------------------------------------------------------
	\section{자타라 파리바르타나아사나 - }















% ========================================================================================= chapter  태양경배자세
	\chapter{태양경배 자세}
	\newpage
	\minitoc

	% -----------------------------------------------------------------------------
	%
	%
	%
	% -----------------------------------------------------------------------------
	\newpage
	\section{태양경배 - 수리야나 마스카라}


		\subsection{태양경배 순서}


			\paragraph{태양경배 순서}
			\begin{itemize}[topsep=0.0em, parsep=0.0em, itemsep=0em, leftmargin=12.0em, labelwidth=3em, labelsep=3em] 
			\item [1.] 	따다 아사나
			\item [2.] 	우드르바 하스타 아사나
			\item [3.] 	웃따나 아사나
			\item [4.] 	아르다 웃따나 아사나
			\item [5.] 	차뚜랑가 단다 아사나
			\item [6.] 	우르드바 우카 스바나 아사나
			\item [7.] 	아도무카 스바나 아사나
			\item [8.] 	아르다 웃따나 아사나
			\item [9.] 	웃따나 아사나
			\item [10.] 	우드르바 하스타 아사나
			\item [11.] 	따다 아사나
			\end{itemize}

		\subsection{태양경배 관련 사이트}
		
			\href{https://www.youtube.com/watch?v=SpqKCQZQBcc}{태양경배자세A}

			\href{https://www.youtube.com/watch?v=CL3czAIUDFY}{태양경배자세A}


		\clearpage
			\subsection{태양경배 관련 그림}

			\begin{figure}
			\centering
			\includegraphics[width=1\textwidth]{./fig/472300819.pdf}
			\end{figure}
			\clearpage



% -----------------------------------------------------------------------------
%
% -----------------------------------------------------------------------------
	\section{달 경배 - 찬드라나마스카라(Moon salutation)}



찬드라나마스카라(Moon salutation)을 들어보셨나요?  달 경배자세 요가 종류 입니다.

수리야나마스카라의 반대인 달경배자세, 찬드라나마스카라가 있습니다.

수리야나마스카라만큼 대중적으로 알려진 동작은 아니지만 Moon salution 또는 Moon flow라고도 불리기도 하는 찬드라나마스카라 !

그동안 태양의 열기를 듬뿍 느낄 수 있는 수리야나마스카라를 즐기셨다면 차분하고 고요한 달의 기운으로 평화와 완전한 릴렉스를 가져다주는 Moon flow ~

찬드라나마스카라를 접해보세요~!

신성한 여성 에너지의 힘을 느낄 수 있는 움직이는 명상.

찬드라 나마스카라!

많은 나라의 문화에서 달은 여성적인 힘을 가진다고 묘사됩니다.

하타요가에서도 이러한 달의 에너지가 우리 안에 내제되어있다고 설명합니다.

태양의 에너지가 따뜻하고 활동적이고 외향적이라면 달 에너지는 차갑고 수용적이며 내향적입니다.

요즘 우리는 뜨꺼운 열기나 남성적인 파워와 추진, 스피드 등이 강조되는 분위기 속에 살고 있습니다.

이 때문에 내면적인 통찰보다는 외부적인 성취만을 중시하기 쉽습니다.

이러한 경향은 우리가 주로 하는 요가 아사나들을 보았을때도 느낄 수 있습니다.

하타요가의 목적은 우리 몸안의 에너지 균형을 맞추는 것이지만 지금 보통 우리가 하는 요가 수련은 수리야나마스카라와 같이 활발하게 몸을 데우는 동작에 더 치중하는 것이 사실이랍니다.

그런데 우리몸은 분출형인 에너지와 차분한 에너지, 둘다 중요합니다.

요가인들이라면 16가지 스텝으로 이루어진 찬드라나마스카라 즉 Moon salutation을 하게 되면 이러한 달의 에너지를 깨울 수 있습니다.


찬드라나마스카라를 할 때에는 달과 신성한 여성의 힘을 찬양하는 마음으로 합니다.

찬드라나마스카라는 이미 느끼셨듯이 파워풀하고 분출적인 움직임이 아닌 내부를 향한 느리고 부드러운 움직임 입니다.

여성들의 경우 임신이나 생리 기간에 찬드라나마스카라를 하면 특히 편안함을 느낀다고 합니다.

찬드라나마스카라를 하는 중에는 느리고 차분한 움직임을 느끼며 부드럽고 깊은 호흡을 합니다.

그리고 수리야나마스카라와 달리 우짜이 호흡을 하지는 않습니다.




% -----------------------------------------------------------------------------
%
% -----------------------------------------------------------------------------
	\section{라자요가 8계단의 첫 단추, ‘야마’는 그저 낡은 도덕률인가? - 오마이뉴스 블로그}


% http://blog.ohmynews.com/feminif/tag/%EB%9D%BC%EC%9E%90%EC%9A%94%EA%B0%808%EB%8B%A8%EA%B3%84

  
현재까지 광범위하게 힘을 발휘하는 전통 요가의 두 가지가 바로 하타요가와 라자요가다. 이 둘 외에 전통 및 현대가 섞인 요가의 가짓수는 전에 소개한 글에서처럼 무수히 많다. 나는 이 가운데 ‘전통을 바탕으로 한 현대적 해석’을 지향하는 라자요가를 다행히 좋은 스승을 만나 3년째 꾸준히 배우고 있다. 심신의 통증 때문에 동작 위주의 요가를 시작한 것은 대략 10년 전이지만 교사, 혹은 안내자로서 집중 과정은 대학원 시절까지 포함 4,5년쯤 되어가는 듯하다.

전체 회원 수 약 6억명쯤이라는 국제요가연맹(IYF)의 라자요가, 인도에서 기원하긴 했지만 20세기 초부터 유럽에서 활발하게 움직였던 신지학의 흐름이 그 뿌리 지식인 IYF 라자요가의 길은 쉽게 말해 유럽식 요가라 할 수 있다. 서구의 과학적 접근 방식에 동양의 수행법을 결합한 형태다. 하지만 점점 시간이 흐를수록 그것을 둘로 갈랐다, 합했다 하는 것도 별 무의미하단 생각이 든다. 






▲신지학회 3대 회장이자 뉴에이지운동의 초석을 놓은 것으로 알려진 앨리스 베일리, 그녀는 《요가수트라 해설》을 비롯 수많은 영적 저서를 남겼다.






어느 세일즈맨 출신의 유럽인이 운전을 하다 유체이탈을 하고 가족과 함께 나들이 간 바닷가에서 신의 에너지와 교감하며, 직관의 힘만으로 에너지 센터의 크기 및 모양과 색깔을 투시한단 말인가. 또 어느 잘 자라고 배운 유럽인이 세상에 이제 꼭 필요한 세 권의 책이 있으니 이는 바로 <신약성경>, <바가바드 기타>, <요가수트라>다, 라 버젓이 말할 것인가.

위대한 요기 부처, 위대한 깨달은 자 솔로몬 왕, 신의 위대한 현현이자 역시 위대한 요기 예수...수많은 지상지하의 여신들, 위대한 어머니 원형 성모 마리아...이런 표현을 아무렇지 않게 쓰는 라자요가의 해설서들...도 이젠 점점 익숙해진다.




성부(聖父)는 정수리, 즉 백회 주변인 사하스라라 차크라에, 성자(聖子)는 미간 사이 아즈나 차크라에 현현하며, 성신(聖神)은 목부터 꼬리뼈까지 다섯 개의 차크라를 관장하니, 성(Saint) 삼위일체는 바로 우리 영, 혼, 신의 삼체를 말한다, 해석하며 전 지구적 종교의 통합을 추구하는 이 요가의 주장은 암만 봐도 재미있고 독특한 것 같다. 무엇보다 내게 잘 맞는다. 21세기는 영성계에 있어 가장 중요한 화두요 키워드가 ‘통합’이라는 말에도 잘 부합되고 말이다.




나아가 생식기, 호르몬 중심 생물학적 성차는 인간의 전 생애 속에서 발정기의 포유류들처럼 짧은 순간이며, 음양 에너지의 차이가 곧 세상만물을 이루고 여신과 남신 역시 그때그때 프라나 운행(에너지조절)에 따라 갈라졌다 마침내 한 몸으로 통합되니 이것이 바로 조화와 균형의 상태이다. 이제 이 통합의 상태가 각 개체에 일시적 상태가 아니라 구조화되어 영원히 지속되게 되면 이를 다른 말로 해탈, 열반, 목샤의 성취라 한다. 그렇게 되면 우리는 다시는 여와 남의 몸으로 분리되어 이 물질계로 돌아오고 싶지 않으면 돌아올 일이 없다는 것...! 대신 영원히 죽지 않는 각 개체 속의 불멸의 의식이 자유롭게 제 할 일을 찾아 상위 우주 속에서 노닐며 즐거이 일하게 된다는 것...천사든 스승이든 전령사든 제자든 열망자든...온전히 변형된 존재의 형태로.

 

아무튼 지금부터 나는 이렇게 인도식도 유럽식도 한국식도 아닌 그저 위대한 신, 내면의 신, 자아와 참자아 사이의 결합을 말하는 라자요가 여덟 계단에 대해 있는 한도 안에서 찬찬히 내 나름의 체험을 섞어 소개하려 한다. 그 처음은 야마, 금계라고 불리는 것들이다.




야마 - 요기로서 금하는 다섯 가지 계율




야마, Yama는 발음 상 이런저런 쓰임새가 많다. 일본어로는 산(山),이란 뜻이고 역시 일본어 음독으로 머리 두(頭)자, 를 야마라고도 읽는다. 그래서 사실 야마, 하면 우리 귀에 가장 익은 말이 바로 흔히 ‘야마 돈다’ 일 것이다, ㅎㅎㅎ! 그런데 인도의 고대어인 산스크리트어로 ‘야마’는 쌍둥이를 이르며 8단계 요가란 뜻의 아슈탕가 요가(요즘 유행하는 파워요가 = 아슈탕가요가와는 다른 의미의 아슈탕가, 원래 8개의 가지란 뜻)에서 쓰이는 뜻으로는 제어, 즉 조절법이나 ‘컨트롤’을 의미한다. 자아와 욕망의 조절을 위한 5가지 계율이 그러므로 라자요가의 첫 번째 단추인 야마, 라 하겠다.




야마는 모두 5개의 금계라 하여 요가수련자로선 결코 행위로도, 말로도, 생각으로도 절대 하지 말아야 할 것들을 이른다.




1. 아힘사 - Non violence  - 폭력이나 살생을 행하지 않음

2. 사트야 - Truthfulness / Intuition   - 거짓을 말하지 않음 

3. 아스테야 - Not stealing - 투기, 시샘으로 훔치지 않음

4. 브라마차리야 - Chastity - 음행하지 않음

5. 아파리그라 - Non desire - 탐내지 않음 






▲아힘사를 무엇보다 가장 중요한 근본 사상으로 주장한 마하트마 간디





사실, 오래 전 10여년 동안 중고등학교의 도덕, 윤리, 상담교사던 내게 요가에서 다시 만난 금계, 곧 야마, 라는 것은 애증의 단어가 아닐 수 없다. 교사를 그만 두고 여성주의 저널 <이프>에 열중했던 시절엔 더욱 그러했고 35년 인생의 중반, 전환점을 맞이하여 가출 내지 출가를 결행하던 그때는 더더욱 그랬다.


규칙, 도덕, 윤리, 법...? 하하하, 그따위 건 개나 줘 버려라,의 심정이던 것...!




사실 별로 잃을 게 없는 자, 법 없이도 살아왔지만 늘 법에 의해 따귀 맞고 홀대 받는 자, 그들은 누구던가? 누가 누구를 위해 무엇을 지켜야 하나? 반면 국가와 제도와 법치주의와 그 근간을 이루는 윤리도덕의 입안자와 최대 수혜자는 누구인가, 그 답은 뻔했다.

힘을 가진 자, 뭔가를 이룬 자, 자신만의 역사를 쓰는 자, 곧 ‘히스’토리의 주인공들 아니던가? 그런데 왜 또 그것을 다 스스로 버릴 수밖에 없었고 잃거나 빼앗긴 사람들이 더 지켜야 하지? 왜 나만 자꾸 억울하게 남보다 ‘차카게’ 살아야 하는 걸까? 때문에 5개의 야마(요기로서 절대 하지 말아야 할 것들)와 5개의 니야마(요기로서 늘 꾸준히 행해야 할 것들)가 라자요가의 처음 1,2 단계 수행법이라는 것은 사실 가장 매력 없던 형식적 절차이자 문구였다, 적어도 내게는.




다시 가슴으로 받아들인 ‘올바른’ 삶의 의미




그런 어느 날, 스승 론으로부터 우리가 준수해야 할 야마란 외적 규율이 아니라 내면의 약속이라는 것, 행위의 결과로서만 사람을 꾸짖고 구속하고 단죄하는 율법,이 아니라 내 가슴의 소리를 무엇보다 직관을 사용, 정확하게 듣는 것, 이란 설명을 듣게 되었다. 또한 어떤 사람의 태생이 유난히 유하고 물러서 남의 마음을 조금이라도 상하게 되면 크게 후회하고 찜찜한 기분이 드는 것이 아니라, 거꾸로 남의 마음과 몸을 상하게 되면 자신의 에너지 흐름에 큰 장애물이 생겨 그런 불안함이 자꾸 생기는 것이란 이야기도 아울러 자꾸 듣게 되었다.




너무 평범하고 당연한 소리인지 모르겠지만 나 같은 삐딱이에겐 그것이 일단 도움이 되었다. 내가 못나서 뺏기고, 유약해서 우는 것이 아니라 도리어 내게 그런 짓을 한 사람들이 더 에너지가 꽉꽉 막힐 것이고, 나는 거꾸로 그런 짓을 하지 않음으로서 약한 것이 아니라 깨끗한 것이며, 못난 것이 아니라 강했던 거다, 어쩌면 이 또한 결국 자의적 해석이겠지만, 모든 생각과 말과 행위의 동기와 결과를 내적 에너지의 흐름과 차단의 관점으로 해석하게 되면 당했던 나보다 가해자던 이들이 더 잃는 게 많고, 앗기는 자보다 앗는 자들의 심신이 더 썩고 병들 수밖에 없다는 이치였다.

‘생명에게 나쁜 짓을 하면 그를 행하는 사람의 에너지가 막혀 도리어 더 큰 병이 나리라!’, 그러므로 말할 수 없이 잔학한 이들의 인과를 더 불쌍히 여겨주어라, 죄에 묶인 그들의 노예 같은 나날에 더 연민을 느껴라, 지금은 지는 자들이 결국 나중엔 이길 것이다......조금씩 수긍이 되었다, 그건 일종의 저주 같기도 하고 주문 같기도 했으니까.

그래도 함정에 빠지면 안 돼, 나는 계속 정신을 바짝 차렸다. 이 역시 숱하게 듣고 낙심한 패배자들의 논리, 눌림에 익숙한 자들의 자위가 아닐까, 의심하고 또 의심했다. 궁금하면 번쩍 손도 많이 들었고 지당하신 말씀들 듣다가 의심에 휩싸여 포기하고 꾸벅꾸벅 졸기도 많이 했다.




시간이 흐르고 1번 야마와 2번 니야마를 자주 건너뛰고는 3번 아사나부터 진짜 요가라는 생각만이 자꾸 들 무렵이었다. 스승 론은 자주 이렇게 말했다. 




- 나는 구루(어둠을 쫓는 자, 라는 뜻으로 요가에서 마스터 이상 더 큰 스승을 이름)가 아니라 당신들의 요가 친구일 뿐이다.

- 내게 절을 하지 말고 당신 자신 안의 신(God With In)에게 인사하라.

- 선생이란 다들 지독한 거짓말쟁이들이다, 각자 직접 경험하지 않은 것은 누가 뭐라든 믿지 마라.

- 인간은 누구도 타인을 조종(Control)하려 해선 안 된다. 나는 당신에게 아무 영향을 미치고 싶지 않다. 누구나 자신 안의 소리를 들어야 한다. 그것만이 진리다. 자기가 뭘 좀 안다고 누굴 조종하려 하면 그것은 사랑이 아니라 곧 죄다. 인간은 결국 자신만을 간신히 조절할 수 있을 뿐인데 그로부터 위대한 존재의 연금술이 일어난다.

- 번거로운 조직과 체계를 거부하고 그저 혼자 있으라. 진정한 관계는 그로부터 진짜 생기게 된다.

- 무엇이 되려 하지 말고 그저 존재하라. 

- 아무 것도 하지 않음으로 많은 것을 하라.




는 말들...그런 시간들이 쌓이면서 나는 야마에 대하여도 많은 의문이 풀리는 느낌이 들었다. 본질을 보기보다 결론을 먼저 익힌 것이 야마에 대한 내 오류라는 판단이 든 것이다. 즉 야마든 니야마든 시작은 우선 자신의 내면이다. 내재율을 따라 사는 삶은 아름답다. 그것을 이용하여 자신의 유익으로 삼고 힘없는 자를 혹독한 계명으로 다스리려 했던 것은 그것을 악용한 브라만이나 바리새인 같은 지배자들, 위선적 종교지도자의 잘못이다.

인간이 인간을 존중하고 스스로를 정화하고자 하는 길, 그에 이르는 야마는 그 자체 잘못이 없다. 그런 결론들...그래서 지킬 거면 아예 온전하게 깔끔하게 시원하게 완벽하게 지키기로 마음 먹었다. 물론 지금도 자주 걸려 넘어지지만. 그로 인해 남 눈치 볼 것도, 남을 판단하거나 정죄할 필요도 없다. 우선 내가 제일 먼저 바뀔 것이다. 그리고 계속 앞으로 나가보자. 계율을 바로 이해해야 동작도 건강도 따라온다니...그것이 다섯 개의 야마를 내가 처음 받아들인 방식이다.     



                                                                                                                   -무아 





% ========================================================================================= chapter
	\chapter{다리 일자 벌리기}
	\newpage
	\minitoc

	% -----------------------------------------------------------------------------
	%
	%
	%
	% -----------------------------------------------------------------------------
	\section{다리 일자 벌리기} 



% ========================================================================================= chapter
	\chapter{고관절}
	\newpage
	\minitoc

	% -----------------------------------------------------------------------------
	%
	%
	%
	% -----------------------------------------------------------------------------
	\section{고관절 열기 닫기} 








%	================================================================== Part			요가 교정서
	\addtocontents{toc}{\protect\newpage}
	\part{요가 교정서}
	\noptcrule
	\parttoc				

	% -----------------------------------------------------------------------------
	%
	% -----------------------------------------------------------------------------
	\section{요가 교정서}



% ========================================================================================= chapter 내 몸에 맞는 요가 교정 A.P.C
	\chapter{내 몸에 맞는 요가 교정 A.P.C}
	\minitoc% Creating an actual minitoc

	% -----------------------------------------------------------------------------
	%
	% -----------------------------------------------------------------------------
	\section{내 몸에 맞는 요가 교정 A.P.C}

	% -----------------------------------------------------------------------------
	%
	% -----------------------------------------------------------------------------
	\section{서문}

''자세, 자유 호흡, 시선''이 세가지는 구루지가 즐겨말하곤 했던 또 하나의 경우이다.
구루지가 학생들에게 주는 1\%의 이론은 단순해 보인다.
하지만 이것을 성취하기는 매우 어렵다.
여기서 구루지가 뜻하는 바는, 바른 자세가 균형 잡힌 호흡의 조건을 만들고, 균형잡힌 자세와 호흡, 이 두가지는 모든 있는 그대로 보고 가슴으로 느낄수 있는 편안한(수카 Sukha) '시선' (드리스티)을 위한 제반 조건들을 형성한다.




	\section{1부 관찰}
보는 자 :
마음이 고요해지면 있는 그대로 보게 된다.
그렇지 않을 때는 많은 것을 보아도
보고 싶은 것만 보게 된다.


	\section{2부 체형}
	\section{3부 교정}
	\section{부록 요가 자세 체형 비교 방법}




%	================================================================== Part			요가 지도법
	\addtocontents{toc}{\protect\newpage}
	\part{요가 지도법}
	\noptcrule
	\parttoc				

	% -----------------------------------------------------------------------------
	%
	% -----------------------------------------------------------------------------
	\section{요가 지도법}




%	================================================================== Part			요가대회
	\addtocontents{toc}{\protect\newpage}
	\part{요가 대회}
	\noptcrule
	\parttoc				

	% -----------------------------------------------------------------------------
	%
	% -----------------------------------------------------------------------------
	\section{요가 대회}










%	================================================================== Part			해부학
	\addtocontents{toc}{\protect\newpage}
	\part{해부학}
	\noptcrule
	\parttoc				



	% -----------------------------------------------------------------------------
	%
	% -----------------------------------------------------------------------------
	\chapter{인체의 구조}
	\minitoc% Creating an actual minitoc

	% -----------------------------------------------------------------------------
	%
	% -----------------------------------------------------------------------------
	\section{인체의 구조}


	% -----------------------------------------------------------------------------
	%
	% -----------------------------------------------------------------------------
	\chapter{골격계}
	\minitoc% Creating an actual minitoc

	% -----------------------------------------------------------------------------
	%
	% -----------------------------------------------------------------------------
	\newpage
	\section{골격계}
	
	\subsection{뼈의 기능}
	
	\subsection{뼈의 구조}
	
	\subsection{뼈의 분류}
	
	
	\subsection{관절}
	
	
	\subsection{척추와 골반}
	
	
	\subsection{견갑골}
	
	\paragraph{위치}
			견갑골은 어깨의 뒷면에 좌우대칭으로 제2~제7 늑골에 걸쳐 있습니다.

	\paragraph{구조}
			견갑골은 역삼각형 모양의 넓적한 뼈로 3개의 오목과 4개의 돌기로 이루어져 있습니다. 
			견갑골 주변으로 여러 근육과 관절이 위치하고 있는데 이로 인해 몸통의 뒤쪽과 팔을 연결하며, 몸통의 앞쪽은 빗장뼈와 연결되어 있습니다. 

	\paragraph{기능}
			견갑골은 어깨관절을 형성하는데, 어깨의 바깥쪽에는 쇄골과 연결되고 팔쪽으로는 위 팔뼈와 연결되어 있습니다. 
			견갑골은 빗장뼈와 같이 팔을 가슴쪽으로 안정시켜 몸통과 팔을 연결하고 힘과 움직임을 전달하는 역할을 합니다. 

	\paragraph{헬스팁}
			견갑골이 불안정하면 인접한 근육은 쉽게 피로해져 여러 가지 기능장애를 일으키며, 나아가 근육의 손상을 유발시킬 수 있습니다.
			
			그러므로 벽에 기대어 팔굽혀펴기, 손을 바닥에 대고 몸을 위로 일으키기, 엎드려 노젓는 자세하기 등의 견갑골 안정화운동이 필요합니다.



	% -----------------------------------------------------------------------------
	%
	% -----------------------------------------------------------------------------
		\section{척추}

경추, 흉추, 요추, 천추, 미추 등으로 구분 \\
가장 움직임이 많은 것은 경추 \\
가장 문제를 많이 일으키는 것은 요추 \\

\paragraph{척추}
척추는 모두 33개의 뼈로 구성되어 있으며, 주변의 근육과 각 뼈를 연결하는 수많은 인대에 의해서 그 기능이 유지된다. 

\paragraph{척추}
우리가 꼿꼿이 서 있다고 하여 척추의 배열 자체도 꼿꼿하지는 않다. 
척추는 목뼈인 경추로부터 앞으로 휘고 뒤로 휘는 전만과 후만을 교대로 가져, 완만한 S자 모양을 하고 있다. 
처음 출생 시에는 후만으로 구성되어 있다가, 머리를 들을 때쯤 경추의 전만이 형성되며, 앉아서 걷기 시작하는 때부터 요추의 전만이 형성된다. 
우리 조상들은 이러한 현상을 예전부터 알아서 백일과 돌이라 하여 축하해 주었다. 
이런 형상은 뼈 자체에 의하여 생기기도 하지만 각 뼈 사이에 있는 추간판(디스크)에 의하여 형성된다. 

\paragraph{척추의 구성}
척추는 경추 7개, 흉추 12개, 요추 5개, 천추 5개, 미추 4개로 구성되어 있다. 

\paragraph{경추}
이중 첫 번째 경추는 머리를 받치고 있어, 그리스 신화의 지구를 받치고 있는 아틀라스와 비슷하다고 하여 이름도 아틀라스라고 하며, 모양이 둥근 고리 모양으로 생겨 환추라 한다. 두 번째 경추는 첫 번째 경추의 몸체와 두 번째 경추의 몸체가 붙어 있는 모양을 하고 있으며, 몸에서 머리의 회전 운동의 축이 된다고 하여 축추라 부른다. 

가장 운동이 많은 척추 부위는 경추 부위이며, 이는 굴곡과 신전 측굴, 회전 운동을 하고 있으며, 정상적으로 340도의 운동을 할 수 있다. 이중에서도 환추와 축추에서 회전 운동의 50~70\%를 감당하고 있다. 

\paragraph{흉추}
흉추는 늑골에 의하여 가슴의 앞에 있는 흉골과 연결되어 있으며, 이로 인하여 흉추부에서는 거의 운동이 일어나지 않는다. 또한 운동이 거의 없는 관계로 흉추부에서 발생하는 질환은 그리 많지 않다. 뒤로는 어깨관절 움직임의 꼭지점 역할을 해주고 있다. 

\paragraph{요추}
요추부는 가장 문제를 많이 일으키는 부위로 질환도 많고, 다치기도 많이 하는 부위다. 요추는 다른 척추에 비하여 크기도 크고, 두껍기도 더욱 두껍다. 요추부는 경추에 비하여 운동 범위는 작으며, 전체적으로 약 240도의 운동 범위를 가지고 있다. 요추부에는 강한 근육이 부착되어 있어 힘을 낼 수 있으며, 주변부의 지지 없이 우리의 체중을 아래로 전달 할 수 있다. 

\paragraph{천추}
천추는 다섯 개의 뼈가 뭉쳐져 하나를 이루고 있으며, 위에서 전달된 하중을 골반으로 양분하는 역할을 하고 있다. 

\paragraph{미추}
미추는 3~4개의 뼈로 구성되어 있으며, 완전히 유합이 되어있는 경우도 있다. 평상시에는 운동이 없으며, 골반의 하부를 구성하는 근육이 붙는 위치이며, 여자의 경우 출산 시에 움직여서 산도를 넓힐 수 있다. 

\paragraph{}
척추는 몸을 받쳐 직립보행을 가능하게 할 뿐만 아니라, 뼈의 안쪽에 구멍이 있어 이 안으로 머리의 뇌로부터 연결되어 우리의 사지로 진행하는 척수 신경의 통로가 된다. 
외부의 충격으로부터 신경을 보호하는 역할을 하지만, 이것이 퇴행성 변화를 겪게 되면 골극이 형성되거나 주위의 인대가 두꺼워져 오히려 신경을 눌러 추간판 탈출증이나 척추관 협착증 등의 질환을 유발할 수 있다.	

\paragraph{추간판 탈출증}
\paragraph{척추관 협착증}


	% -----------------------------------------------------------------------------
	%
	% -----------------------------------------------------------------------------
	\section{고관절}

	% -----------------------------------------------------------------------------
	%
	% -----------------------------------------------------------------------------
	\section{대퇴골}


	% -----------------------------------------------------------------------------
	%
	% -----------------------------------------------------------------------------
	\section{견갑골}



	% -----------------------------------------------------------------------------
	%
	% -----------------------------------------------------------------------------
		\chapter{근육계}
		\minitoc% Creating an actual minitoc



	% -----------------------------------------------------------------------------
	%
	% -----------------------------------------------------------------------------
		\section{근육계}

우리 몸에는 약650개 이상의 근육이 존재한다.
광배근이나 승모근과 같이 커다란 근육도 있고, 소원근, 능현근과 같은 작은 근육도 있다.

근육의 특성과 발달 원리에 따라 우리는 평생 동안 근육을 무리하지 않게 꾸준히 사용해야 한다.
우리 몬에는 약 650개 정도의 근육이 있는데,

근육이라고 다 같은 근육이 아니다.
근육은 위치에 따라 골격 근육, 내장 근육, 심장근육으로 나뉘며, 
움직임을 조절할 수 있는지 여부에 따라 수의근과 불수의근으로 나눈다.

	\subsection{근육의 구분}

근육은 크게 적근과 백근으로 구분된다.
적근 : 지근이로 불리며, 오랜 시간 사용이 가능
백근 : 손근으로도 불리며, 짧은 시간만 사용 가능




	\subsection{근육의 특성}

		근육은 아래와 같은 특성을 가진다	

			\begin{enumerate}	
							[
							topsep=0.0em, 
							parsep=0.0em, 
							itemsep=0em, 
							leftmargin=12.0em, 
							labelwidth=3em, 
							labelsep=3em
							] 
			\item 1. 적당히 사용하면 증진한다.
			\item 2. 사용하지 않으면 쇠퇴한다.
			\item 3. 과다하게 사용하면 파괴된다.
			\item 4. 근육마다 회복의 속도가 다르다.
			\end{enumerate}



	\subsection{탄수화물의 섭취}
	
근육은 근육섬유가 여러 개 뭉쳐 있는 형태로 존재하며, 근육이 잘 사용되기 위해서는 운동 이후에 탄수화물을 섭취하여 에너지를  공급해줘야 한다.
만약 탄수화물의 섭취가 제대로 이루어지지 않는다면 근육에 저장되어 있던 탄수화물이 사용되면서 근육의 손실을 일으킨다.
그러므로 운동이후에는 탄수화물의 섭취가 필요한데, 최소 30분~40분 이내에는 섭취해야 근육 내에 있는 탄수화물의 손실을 막을 수 있다.
이 시간이 지나면 몸이 영양분을 끌어들이는 능력이 떨어지게 된다.

간혹 어떤 사람들은 탄수화물보다는 단백질의 섭취가 더 중요하다고 이야기하는 경우가 있다.
단백질이 근육을 만들기 위해 필요한 영양소임에는 분명하지만, 근육을 사용하는데 에는 탄수화물이 필요하다.

운동 후 30분 ~ 40분 이내가 영양분을 흡수할 수 있는 가장 좋은 상태이기 때문에 꼭 영양분을 섭취해야 한다.
그래서 운동선수들이나 전문적인 보디빌더 들은 이 시간대를 '황금시간'(기회의 창문)이라고 부른다.


탄수화물은 체지방을 연소하는데 장작과 같은 역할을 하기 때문에 운동에 있어 매우 중요한 요소이다.


운동을 통해 근육의 미세한 손상이 일어나면 근육에서 탄수화물을 요구하게 되는데, 
이때 제대로 섭취하지 않으면 근육의 손실이 일어나기 때문에 운동 후 30~40분 이내에 섭취를 해야 근육의 손실을 방지 할 수 있다.




	\subsection{혈당}
평소의 식사에 있어서 다당류(복합탄수화물)가 주가 되는 식품을 추천하는 이유는 소화와 흡수가 느리다 보니 혈당을 천천히 증가시키기 대문이다.
반대로 단당류의 섭취로 너무 빠르게 소화와 흡수가 이루어지면 혈당이 급격히 상승하여 당뇨병을 유발할 수 있으며, 간에서 지방합성이 증가할 수 있다.

열심히 운동을 했다면 양질의 응식을 먹어야 한다.
운동을 한 후에는 혈당이 떨어지고, 근육의 반복적인 수축작용으로 인해 손상된 근육 조직을 회복시켜야 하므로 운동을 마친 후는 꼭 음식을 섭취해야 한다.
운동 후에는 소화 흡수가 빠른  단당류의 탄수화물을 섭취하고, 과일음료나 바나나 같은 음식이 권장된다.


	\subsection{활성산소}

운동을 하면 활성산소가 발생한다.
활성산소는 쉽게 말해 몸을 산화시키는 나쁜 물질이다.
운동을 무리하게 너무 오래하게 되면 활성산소가 발생한다.
그렇기 때문에 활성산소를 억제하는 항산화 식품을 섭취할 필요가 있다.

대표적인 항산화 식물으로 토마토를 권장한다.
또한 달걀도 권장되는데, 달걀은 근육을 형성하는데 사용되는 필수 아미노산을 골고루 갖추고 있다는 점에서 완벽한 단백질 식품이다.
하지만 기름을 사용하여 조리한 프라이 형태로 먹으면 칼로리가 높아지므로 삶은 달걀이 이상적이다.



	
	\subsection{근육의 생리적 특성}
	
	
	\subsection{근육의 분류}
	
	
	\subsection{근육의 화학적 조성}
	
	\subsection{골근계의 기능}
	
	
	
	\subsection{다리의 근육}
	
	\subsection{대둔근}
	
	
	\subsection{요추근}
	
	
	\subsection{팔의 근육}

	% -----------------------------------------------------------------------------
	%
	% -----------------------------------------------------------------------------
		\section{요근}

	% -----------------------------------------------------------------------------
	%
	% -----------------------------------------------------------------------------
		\section{장유근}

	
	



	



	% -----------------------------------------------------------------------------
	%
	% -----------------------------------------------------------------------------
		\section{등 근육}
	
	% -----------------------------------------------------------------------------
	%
	% -----------------------------------------------------------------------------
		\section{광배근}

	% -----------------------------------------------------------------------------
	%
	% -----------------------------------------------------------------------------
		\section{승모근}


	% -----------------------------------------------------------------------------
	%
	% -----------------------------------------------------------------------------
		\section{척추 기립근}


	% -----------------------------------------------------------------------------
	%
	% -----------------------------------------------------------------------------
		\section{척추 기립근}


			\begin{itemize}		[
							topsep=0.0em, 
							parsep=0.0em, 
							itemsep=0em, 
%							leftmargin=12.0em, 
							leftmargin=6.0em, 
							labelwidth=3em, 
							labelsep=3em
							] 
			\item 등에 존재하는 근육. 영어로는 spinal erector muscle 라고 불린다. 천골(sacrum)과 장골(ilium)에서 시작해 척추와 늑골, 일부는 두개골에도 결합하는 근육이다. 가쪽으로 척주를 굽히거나 척추를 펴는것이 주요 기능.

			\item 글로벌코어: 척추기립근, 복직근, 외복사근, 내복사근, 둔근 등
			\item 로컬코어: 복횡근, 횡격막, 다열근, 골반저근

			\item 척추기립근은 글로벌코어에 해당한다. 
			\item 허리에 붙어있는 근육으로 복근의 길항근이다. 복근을 키우기위한 복근운동을 할 때 척추기립근이 약하면 허리에 손상을 입기 쉬우므로 복근을 키우기 위해서도 단련을 해야하며 겨우 그 정도의 기능만을 위한 게 아니라 코어 근육 중 하나로 실제 스포츠 퍼포먼스에서도 굉장히 중요한 근육이다. 물론 그것에도 당연히 유용하다. 하체와 함께 몸의 줄기이자 기반이 되는 근육이기에 거울로 안보인다고 등한시 하지말고 키우자.

			\item 운동으로 근육질인 사람중에서도 척추가 비틀어지거나 허리통증이 있는 사람들이 있는데 로컬코어에 문제가 있어서 그럴 확률이 높다. 한번 문제가 생긴 상태에서 기존의 운동을 아무리 열심히 해도 나아지지 않는 경우가 많다. 척추기립근은 큰 힘을 쓰는 근육이지 다열근 처럼 척추를 잡아주는 기능은 떨어진다. 다열근에 문제가 생겨서 다열근은 약해지고 척추기립근만 과대하게 커지는 사례도 있다. 사람의 몸은 항상 움직이기 전에 로컬코어가 먼저 활성화 되고 다른 근육들이 움직이는 것이 정상이다. 이런 패턴에 문제가 있는 사람은 헬스를 그만두고 재활부터 시작해야 안다친다.
			\end{itemize}






% -----------------------------------------------------------------------------
%
%
%
% -----------------------------------------------------------------------------
\newpage
\section{순환계}



% -----------------------------------------------------------------------------
%
%
%
% -----------------------------------------------------------------------------
\newpage
\section{호흡계}


% -----------------------------------------------------------------------------
%
%
%
% -----------------------------------------------------------------------------
\newpage
\section{소화계}

% -----------------------------------------------------------------------------
%
%
%
% -----------------------------------------------------------------------------
\newpage
\section{비뇨계}


% -----------------------------------------------------------------------------
%
%
%
% -----------------------------------------------------------------------------
\newpage
\section{내분비계}


% -----------------------------------------------------------------------------
%
%
%
% -----------------------------------------------------------------------------
\newpage
\section{신경계}


% -----------------------------------------------------------------------------
%
% -----------------------------------------------------------------------------
		\section{콜레스테롤}


% -----------------------------------------------------------------------------
%
% -----------------------------------------------------------------------------
		\section{호르몬}



% -----------------------------------------------------------------------------
%
% -----------------------------------------------------------------------------
		\section{호르몬 : 멜라토닌 호르몬}


열심히 운동을 한 이후에는 휴식을 취해야 하는데, 여러가지 휴식의 형태 중 최고는 수면이다.
수면을 취하면 멜라토닌 호르몬이 분출된다.
멜라토닌은 면역력을 높여 주고, 부종을 없애며, 인체의 회복에 많은 영향을 준다.




% -----------------------------------------------------------------------------
%
% -----------------------------------------------------------------------------
		\section{운동을 하면 피로감이 오는 이유}


		\begin{enumerate}				[topsep=0.0em, 
									parsep=0.0em, 
									itemsep=0em, 
									leftmargin=6.0em, 
									labelwidth=3em, 
									labelsep=3em] 
			\item 	1. 젓산의 축적
			\item 	2. 몸속 에너지의 고갈
			\item 	3. 신경계의 고갈
			\item 	4. 근육의 통증
		\end{enumerate}




%	================================================================== Part			산스크리트어
	\addtocontents{toc}{\protect\newpage}
	\part{산스크리트어}
	\noptcrule
	\parttoc				

% -----------------------------------------------------------------------------
%
%
%
% -----------------------------------------------------------------------------
\newpage
\section{산스크리트어}



\newpage
\section{구}

\paragraph{구루(Guru)}
요가의 구전된 전통에 따르면 구(go)라는 음절은 암혹을 의미 하고 루 (rU) 라는 음절은 그것을 몰아내는 사람을 의미한다. 그는 요가 수행법의 전부를 시도해보았고 영적인 통로를 통하여 생기는 모든 종류의 경험을 다려 어 녈 사람이 라 현혹스런 경험 으로부터 수행 자를 안내할 수 있고 수행자가 억류시키려는 에너지를 적절히 사용할 수 있게 돕는다. 외면적인 사람으로서 구루의 목적은 수행자를 내 부에 존재하는 수행자의 스승 즉 그 자신의 자아에 로 이끌어주는 것이다.

\paragraph{가르바 핀다 (Garbha-pinda)}
자궁내의 태아

\paragraph{강가 (Ganga)}
인도에서 가장 성스러운 강의 이름, 갠지스 강

\paragraph{고 (Go)}
암소

\paragraph{고라크사 (Goraksa)}
목동, 유명한 요기의 이름

\paragraph{고무카 (Gomukha)}
암소를 닮은 얼굴. 또한 일종의 악기의 이름으로 암소의 얼굴처럼 한쪽 끝은 좁고 다른 한쪽 끝은 넓다.

\paragraph{구나 (Guna)}
성질, 특질, 자연의 구성분이나 요소





\newpage
\section{아 }



\paragraph{아드바이타(Advaita)} : 「둘이 아니다」라는 뜻. 베단타 학파의 철학에서 사용되는 용어로써 이 학파에서는 아트만의 개인적인 속성을 없애고, 합일된 우주의 실체 (브라만)와 아트만을 동일시하며 또한 세상에서 존재하지 않는 것들을 흔히 경험해봄으로써 그것들을 아트만과 일치시키는 것을 강조한다. 상카라차리아의 브라마 수트라와 우파니샤드 철학에 대한 해설서에 보면 잘 묘사되어 있다.

\paragraph{아그니 (Agni)} :불의 뜻을 지닌 단어, 아그니는 베다의 정령으로서 신들의 전령이라고 하며 모든 바치는 공물들을 받아들여서 신의 명령에 의해 사용할 수 있게끔 변화시키는 역할을한다. 이런 측면에서 본다면 아그니는 그내면으로 부터 생성 되 어지 는 모든 변화되고 변형 하는 것들을 포함한 생명력의 상징이다.

\paragraph{아함카라(Ahamkara)} : 「나자신을만들어 주는것」 또는 에고(egO)이며 요가철학이나 상키야에서 사용되는 말로 마음의 기능을 뜻하며 이 아함카라 때문에 순수한 영혼 물질적인 것과 정신의 산물을 옳겨 식별하지 못하게 된다

\paragraph{아힘사(Ahimsa)} '다섯개의 도덕적인 금기사항중 첫번째써 야마스(yamas)라고도 불리는데, 이것은 요가의 8단계 중 첫번째 단계이기도 하다. 아힘사의 목적은 영적인 성장에 지장을 주는 행동을 될 수 있는 한 줄이려는 것이다. 아힘사는 글자 그대로 해석하자면 「살생을 하지않는것 또는 「해를 끼치지 않는 것」의 의미이며 생각과 언행에 있어 격함이 없는 것을 가리킨다. 이러한 언행을 함으로 모든 사물에 대한 가없는 사랑」을 수행할 수 있게 된다.

\paragraph{아즈나 차크라(Ajnachakra)} : 「통제력 중심부위」라는 의미 이며 의식의 중심이며 육체 신경계통의 신경망과 대부분 일치하며 위치는 양쪽 눈썹의 사이에 있다. 「통제력 중심 부위」라고불리는 이유는깨어 있는상태에서 마음의 근원이 되며, 이 아즈나 차크라가 개발되면 기타 다른 의식  센터는 이것의 통제력권에 들어오게 되기 때문이다. 아wm나 차크라는 마음의 주된 에너지를 통제하고 그 힘은 (OM)의 음절(bija)에 의해 일깨워진다. 쿤달리니 에너지 가 수슘나(sushumna)통로를 통하여 상승하게 되면 사마 디를 깨달을 수 있게 된다(차크라, 쿤달리니, 사비칼, 사마디, 수슘나 찹조).

\paragraph{아카샤(Akasha)} :산스크리트어로 공간 에너지의 원리라는 말이다. 허공이나 우리 주위의 대기는 물론 내적인 공간, 가슴의 깊은 골짜기 등으로 해석될 수 있다.

\paragraph{아나하타 차크라(Anahata chakra)} : 「충격받지 않는 중심 부위」라고해석이 되며, 이 중심 부위는 가슴 가운데 가슴뼈 바로 아래쪽에 위치하며 육체적으로 심장의 계통과 연관이 있다. 공기 에너지의 원리를통제하며 향냄새를 발산하는 것으로 상징되어진다. 또한 성기의 활동적인 감각과 감촉으로 느낄 수 있는 인식적 감각도 통제한다. 다윗의 별이란 것과 비숫하게 생긴 역삼각형과 겹친 삼갈청의 형상이 바로 차크라의 상징 이 다 색깔은 탁한 회색 이 나 초록색이라고 일컬어지고 있다. 아나하타 차크라의 에너지는 얌(yam)의 음절(bija)애 의해 일깨워진다. 이곳은 느낌과 감정의 중심부위 이기 때문에 수행자가 선택된 신성한 만  트라나 구루를 명상할때는 이곳에 의한 명상이 이루어진다. 이 곳은 또한 미묘한 소리의 진동 (nada)에 대한 집중의 센터이며, 알려지기로는 아트만이 존재한다는 곳이 바로 엄지 손가락만한 이곳을 이름하여 「가슴의 내밀한 동굴」이라고 한다.

\paragraph{아난다(Ananda)} :완벽한 즐거움. 브라만의 세 가지 측면중에 하나. 순수한 푼재인 사트(sat), 순수한 의식인 치트 (cit), 순수한 즐거움인 아난다(ananda).안타카라나(Antahkarana) :마음의 「내면적 기구」로써 마나스(manas) 즉 활동적인 마음, 부디 (buddhi) 이성적이  고 직관적인 지성, 치타(citta) 마음의 총합체이고 저장소이며 혹은 삼스카라(samskara)인 미세한 인상이며, 아함카라(ahamkara) 즉나를만드는에고의 동일화등이 구성되어진다.

\paragraph{아누(Anu)} '가장 작은 물질의 개별단위 원자

\paragraph{아파흐(Apah)} : 물. 액체성분의 에너지 원리.

\paragraph{아팍나(Apana)} :생명력 (prana)의 5가지 기능 중의 하나로써 배설기능을 호흡이나 콩팥,방광,결장,직장,배섞기 등을 통하여 수행한다.

아파리그라하(Aparigraha) :무소유 즉 요가철학에서 야마스리·불리는 여러 도덕적 금기사항 중에서 다섯번째의 것. 아파리그라하를지니면 우리의 소유욕을 감소시켜서 우리의 주위에 모든 것이 짐이라는 관념에서 벗어나 도구라는 관념으로 바뀌게 된다. 따라서 고런 것들을 필요로 하는 사람들에게 관대해지게 된다.

\paragraph{아르다 마첸드라사나(Ardha-Matsyendrasana)} 요가자세 의 하나로 척추를 측면으로 비트는 자세이다.

\paragraph{아사나(Asana)} :앉음, 위치, 자세; 라자 요가의 여덟 가지 중 네번째 것으로 편안하고 안정된 자세를 유지하는것에 중심을 둔것이다. 나중에 「자세의 체계」라는의미를 지닌 하타 요가라는 육체적인 과학의 요가로 발전되었다.

\paragraph{아수 (Asu)} : 프라나

\paragraph{아슈탕가 요가(Astanga Yoga)} :
파탄잘리의 요가수트라에 설명된 전통적인 라자 요가의 8단계 를 말한 「요가의 8가지 갈래」를 일컫는다. 이 여털 가지란 야마(yama) 도덕적 금기,니야마(niyama) 도덕적인 수행, 아사나(asana) 자세,프라나야마(pranayama) 프라나와 호흡의 통제, 프라트야하라(pratyahara) 금욕과 감각의 통제, 다라나 dharana) 집중, 드야나(dhyana) 명상, 사마디 (samadhi) 초의식의 명상상태를 말한다.

\paragraph{아스테야(Asteyal)}
훔치지 않음. 아스테야는 야마의 세번째 금기이다. 또한 돈을 함부로 쓰지 않으며 부정한 뇌물을 받지 말 것 등도 포함된다. 이것을 통해 수행자로부터 남의것에 대한 부정한 욕심을 정화시킨다.

\paragraph{아쉬타프라크리티(Ashtaprakriti)}
상키야 철학의 물질적 정신적 우주의 8가지 요소나 원리. 이 여덮 가지란 흙의 성질(땅),액체의 성질(물),뜨거움(불),기체의 성질(옹기), 공간, 마음, 부디 (이성적이고 직관적인 지성), 에고 (아함카라)이다.

\paragraph{아트만(Atman)}
베단타 철학에 따르자면 아트만이란 개인적인 자아를 말하며, 특히 우주적 자아요 절대적인 실체인 브라만과 합치되는 것을 가리킨다. 또 이 아트만은 어떤점에서는 요가나 상키야의 푸루샤, 즉 순수한 영혼과도 연관이 있다. 그순수한 영혼이라는것은 다섯 가지의 껍질 이란 물질적인 육체로써의 몸,생명력을 가진 육체,활동적 인 정신의 육체, 지식 의 육체 , 완벽한 즐거 움의 육체 를 말한다. 명상수행시 수행자는 점차적으로 이런 껍질을 뚫고 자신의 내적인 자아 아트만에 도달하여 사마디의 최고 높  경지를 달성하게 된다.

\paragraph{아트마삭티 (Atmashakti)}
자아의 힘과 기능(삭티 참조)


\newpage
\section{반}
································································································································································································································

\paragraph{반다(Bandha)}
속박,아비드야(avidya) 즉 무지의 소치로 인하여 영적인 자유를 얻지 못한상태. 반대로 영적인 자유를 얻는다는 것은 요가의 목적이기도 하다.

\paragraph{반다스(Bandhas)}
속박에 의한 감금, 프라나야마 수행을할 때 에너지 통로와 생명 에너지의 흐름을 연결시키려는 특정한 육체의 행동양식

\paragraph{박티요가(Bhakti Yoga)}
사람을 대할 때 자비를 베푸는 영적인 통로는 영성과 함께 개발된다. 박티 요가에사 기분에따라 마구잡이 로 흐트러지는 에 너지 는 수행 자의 영성 을 발   휘함으로써 모아지고 그럼으로써 수행자를 확실한 영적인 방향으로 밀어주는 원동력이 된다.

\paragraph{바스트리카 (Bhastrika)}
어원으로는 풀무 호흡을 통한 수련으로 복부근육과 횡격막이 풀무와 같은 기능을 한다. 즉 힘 있게 들이쉬 고 내 쉬는 것을 같은 길이 로 반복하는 것이다.

\paragraph{바바(Bhava)}
기분; 강한 감정을 느끼는 상태. 이 단어는 박티 요가에서 헌신의 에너지 통로를 통하여 느끼는 지고의  기쁨을 나타낼 때도 쓰인다.



\paragraph{부장가사나(Bhujangasana)}
	코부라 자세

\paragraph{빈두(Bindu)}
마음의 활동을 한 곳으로제한시켜 명상의 보다 높은 차원까지 관통해 들어가 수행자로 하여금 사마디  초의식상태 로 이르 하는 것. 이 기 술은 「돌파의 빈두」산스크리트어로 빈두 베다나라고 불린다.

\paragraph{브라마(Brahma)}
힌두교의 3대 신(채) 중에 첫번째이며 라자스(rajas :'에너지)의 원리를 그 특징으로 한다 따라서 그는 생식과 창조(praja-pati)의 신이다.

\paragraph{라흐마차리야(Brahmahcarya)}
「브라만속을 걷는다」는 의미. 야마스의 4번째 것이며 감각을 교묘하고 세심하게 발휘하여 에 너지 를 통제 하고 낭비를 억제시키는 것을 말할 때도 적용된다. 본래 에너지의 낭비는 성행위를 통해 가장 많이 발생하므로 성적인 독신을 의미할 때도 있다. 좀더 적절히 말하자면 모든 감각, 인식 , 환동을 절제 하는 것이 다.

\paragraph{브라만(Brahman)}
우주의 궁극적인 실체, 또는 우주적 자아로 말할 수 있으며, 베단타 철학에서는 사트(sat . 순수  존재, 진리, 실체) , 치트 (cit : 순수의 식) , 아난다 (ananda :  지고의 기쁨)로 표현된 것이다. 그러나 이것은 성질이나 속성을 초월해서 존재하는 것이며, 따라서 우파니샤드 철학에서는 「네티 네티」 (netineti)라 하여 이것도저것도 아닌 것이라는 자유분방한 표현으로 가르친다.

\paragraph{라마-렌드라(Brahma-randhra)}
브라만으로 통하는 문이나 구멍. 숨구멍에 있는 영적인 센터로 요가 수행자가 그의 물질적인 육체의 효용이라 했을 때 스스로 자신의 몸을 빠져나가는 데 사용된다.

\paragraph{부디 (Buddhi)}
분별력,직관적이거나 지적인 지성

································································································································································································································

\newpage
\section{차}

\paragraph{차크라(Chakra)}
「수레바퀴」라는 단어이며 라틴어 circus 와 영어 원(circle)으로 유래되었으며, 차크라는 의식의센터로, 척추를 따라 위치한 신경계통의 주신경계와 일치 한다. 각 센터는 특정한 에너지 원칙을 통제한다. 주에너지 중심 부위는 일곱개로써 각각의 중심 부위는 일곱개의 원리 즉 흙의 원리 (땅),액성의 원리 (물),열(불),기체 (공기),공간,마음,자아의 순수의식과 일치된다. 그리고 각 에너지의 원칙은 특정한 감각 즉 미묘한(물존적인 것이  아닌) 빛과 음성과 일치하며 (명상수행뿐만 아니고) 에너지 센터를 묘사하기 위한 기하학적 도형과도 일치한다. 또한 특정한 비자스(bijas :음절,혹은 소리의 진동)로서 만트라의 기 초를 이 루는 소리와도 일치한다 (이것을 계속 반복하면 특정한 에너지 센터를 일깨우는 데 도움이된다) .

\paragraph{차크라사나(chakrasana)}
수레바퀴 자세

\paragraph{치타(citta)}
잠재의식의 마음이 모인 장소로써 이곳에 감각을 통하여 얻어진 인상이 한데 모여지고 그 바닥으로부터 치밀어 올라와서 분별없는 생각과 대인관계의 꿀임 없는 흐 름을만들어낸다. 파탄잘리의 요가철학의 법칙을보면 "요가는 변화의 중단이고 치타의 변경이다(파탄잘리의 두번째 수트라) "

\paragraph{치타 브르티 (Chitta-vrtti)}
마음의 변화, 행동의 방향, 존재 양식, 상황 또는 정신 상태

\paragraph{챤드라 (Chandra)}
달

································································································································································································································

\newpage
\section{단}

\paragraph{단다 (Danda)}
막대기, 몸뚱이

\paragraph{데누 (Dhenu)}
암소

\paragraph{데바 (Deva)}
신

\paragraph{드르 (Dhr)}
잡다, 지지하다, 유지하다

\paragraph{드뷔 (Dwi)}
둘, 쌍

\paragraph{드뷔 파다 (Dwi-pada)}
두 발 혹은 양쪽 다리

\paragraph{드뷔 하스타 (Dwi-hasta)}
양손

\paragraph{다누라사나(Dhanurasana)}
활 자세

\paragraph{다라나(Dharana)}
집중 마음을 한 곳으로 모우는 과정으로 본래 마음의 성질은 한가지 대상에서 다른 대상으로 옳겨 다니 지만 이것은 자발적 으로 한 세상에 안정되게 집중시키   는 것. 요가 수트라에 나타난 요가의 8단계 중 6번째

\paragraph{드야나(Dhyana)}
명상. 마음이 모든 감각으로 부터 분리되고 집중이 되면 하나의 대상에 대한 안정되고 자연스런 집중의 흐름이 이루어진다.

\paragraph{디브야-착수(Divya-chaksu)}
「성스러운 눈」, 「제삼의 눈」양쪽 눈썹 사이 미간에 위치한 아즈나 차크라가 완전히 깨어났을 때 생기는 비상한 투찰력 (천리안)

\paragraph{디브야 드리쉬티 (Divya Drishti)}
성스러운 시야

································································································································································································································

\newpage
\section{할}

\paragraph{할라사나(Halasana)}
쟁기 자세

\paragraph{하타요가(Hatha Yoga)}
라자 요가에서 세번째 '단계 (a-sana)로부터 발달된 육체의 과학. 이 요가는 자세와 마음의 전화를 통해 수행자를더 높은요가수행에 대해 대비시키는 것이다. 또한 라자 요가에서 외적인 가지로 알려진 단계가 4개가 있는데 그것들을 야마, 니야마, 아사나, 프라나야마라 불린다. 때로는 하타 요가가 이 네 가지의 단계를 의미하기도 한다.

\paragraph{히란야가르바(Hiranyagarbha)}
문자대로 해석하면 「황금 자궁」 또는 「황금달걀」의 뜻. 우주의 마음으로써 최초의 그리고 진정한 은가의 스승

································································································································································································································

\newpage
\section{이}

\paragraph{이다(Ida)}
척추를 따라 흐르는 주에너지 통로 중의 하나. 왼쪽 콧구멍의 호흡을 통제하는 기능이 있고 자연의 여성적인 성질이나 음적인 것에 부합된다고 알려져 있으며, 마음의 직관,창조력,수동성,적막감,수면 등에 밀접하다. 하타라는단어에서 이것은 음절로 -' 다. 하타 요가의 목적은 핑 갈라와 이 다 또 하와 타의 두 통로를 중앙 의 에너지 통로인 수슘나에 집중시켜 마음이 조용하고 기쁘게 명상을 할 수 있는 상태로 만들어준다.

\paragraph{이쉬와라-프라니다나 (Ishwara-Pranidhana)}
요가명상체계의 도덕적 수행 중의 다섯번째 뜻은 「신에게 내맡긴다」라는 것으로 남에게 헌신하는 감각과 자신의 진정한 자아 자신을 내맡기는 것 두 가지의 뜻을 포함한다. 그것은 수행을 통하여 다른 사람에게서 자신의 자아를 발견 할수 있고 그 자아에 자신의 이기심을 기 할 수 있게 되는 것이다.

································································································································································································································

\newpage
\section{즈}

\paragraph{즈나나요가(Jnanayoga)}
「지식의 수행」 즈나나 요가는 수행자의 지성을 발전시키는데 이지적인 이성으로 부터시작해서 최후에는 그 목적을 치트(cit) 즉 순수의식으로 사는 직관에까지 발전시킨다.

\paragraph{즈나나 무드라 (Jnana-mudra)}
집게손가락의 끝이 엄지손가락의 끝과 맞닿아 있고 나머지 세 손가락은 뻗어 있는 모양의 수인. 이 수인은 지식jnana의 상징이다. 집게손가락은 개인의 영혼을 뜻하고, 엄지는 최상의 우주정신으로 이 두 손가락의 결합은 진정한 지식을 의미한다

\paragraph{즈나나 무르가 (Jnana-murga)}
인간을 자각하게 하는 깨달음으로 이끄는 지식의 길

\paragraph{자누 (Janu)}
무릎

\paragraph{자무나 (Jamuna)}
갠지스 강의 지류

\paragraph{자타라 (Jathara)}
복부, 위

\paragraph{자타라 파리바르타나 (Jathara-parivartana)}
복부를 앞뒤로 움직이게 하는 아사나

\paragraph{잘란다라 반다 (Jalandhara-bandha)}
Jalandhara는 목과 목구멍이 수축되어지고, 턱이 가슴뼈 상부의 쇄골 사이의 팬 부분(V자)에 놓여지는 자세이다.

\paragraph{지바 (Jiva)}
생명체, 창조물

\paragraph{지바나 묵타 (Jivana-mukta)}
지고의 영혼 즉 절대 신성과 합일됨으로써, 그의 생애 동안 해탈을 얻은 사람

\paragraph{자파(Japa)}
만트라를 정신적으로 되풀이하는 것, 이것을 통하여 이 만트라의 음절이 대표하고 있는 에너지의 진동 을 일깨운다

································································································································································································································

\newpage
\section{칸}

\paragraph{칸타(Kantha)}
「미세 육체」의 한 부분으로 후두부분과 일치하는 곳.

\paragraph{카이 발랴(Kaivalya)}
푸루샤 즉 순수영 혼을 프라크리 티 (물질적 성질)로 인해 생기와 그릇된 분별로부터 분리시키는것 그 과정 은 상키 야 수행 과 요가철학에서 영적 훈련의 목적 이 기도 하며 요가 수트라 수행의 마지막 장(효)의 주제이다.

\paragraph{카팔라바티 (Kapalabhati)}
「머리부분의 광채」 혹은 「앞이마에 보이는 괌채」라는 뜻. 호흡기술로써 복부근육과 횡경막이 바르고 힘차게 숨을 내쉬고 서서히 수동적으로 들이 쉬는 것이다.

\paragraph{카르마 요가(Karma Yoga)}
「활동의 수행을 통한 훈련」으로써 자아에 의한 행동이지 만 개 인적 인 욕망을 배재시키는것, 이것을 통해서 수행자는 점차적으로 잠재의식을 통해    얻은 상당량의 인상을 반감시킬 수 있다(여기서 인상이란 미래의 활동이나환생의 씨앗이 된다). 일상생활에 명상이 적용됨에 따라 수행자의 활동은 점차로 정화된다.

\paragraph{카르야 브라만(Karya-Brahman)}
비란야가르바와 일치되는 우주의 마음

\paragraph{쿰바카(Kumbhaka)}
상당히 진보된 프라나야마 수행법중 에서 숨쉬는 것을 미루거 나 보류하는 것, 이 러한 숨을 멈추는 수행법은 상당히 경험많고 훌륭한 스승의 지도하에 이루어져 야만 한다.

\paragraph{쿤달리니 (Kundalini)}
 「둘둘 말린 것」의 뜻을 지닌 단어. 집중된 개인의 기본적인 생명 에너지는 뱀이 또아리를 틀고 잠이 든 형 상이 가장 차원이 낮은 의 식의 중심 인 척 추 밑바닥에서 잠물 자는 것으로 상징된다. 앞서서 말한 요가 수행의 목적은 이러한 에너지를 일깨워 수슘나의 통로를 따라 연꽃으로 형상화되어져 있다. 에너지가 각 에너지 중심 위를 통과하여 위로 상승되어 갈 때마다 그것은 말 그대로활짝피게된다. 그리고점차로 수행자의 전체적인 존재가 변화하여 완벽하게 된다.

\paragraph{쿤달리니 요가(Kundaliniyoga)}
수행의 한 가지 체계로써 만트라와 얀트라(Yantra만트라 수행의 도형적인 측면), 특별한 자세 (무드라 : mudra)와 호흡수행을 통하여 숨어있는 쿤달리니의 힘을 일깨우고 상승시키는 것

································································································································································································································

\newpage
\section{라}

\paragraph{라야 요가(Laya Yoga)}
글자 그대로는 「흡수되어 녹아버리는 요가」라 한다. 명상수행의 한 가지 체계로써 수행을 통해 방대한 요소를 일깨워 그것을 좀더 나은 것들 속에 점진적으로 홉수시키는 것.

································································································································································································································

\newpage
\section{마}

\paragraph{마하트(Mahat)} :상키야철학에서 말하는 태초의 자연인 프라크리티에 의해 생기는 가장 첫번째의 실체. 이것은개인적인 부디에 대응하는 우주적인 부디라 할만글 것이다.

\paragraph{마나스(Manas)} :활동적인 마음. 막연한 의미로는 아타흐카라나와 동일한 의미로도 사용된다.

\paragraph{마나스파티(Manaspati)} .마음(manas)의 주인 혹은신. 마음을 통제하는 요소

\paragraph{마니푸라 차크라(Manipura charra)}'「보석의 도시」라는 뜻으로 배꼽 부분에 위치한 의식의 센저로 태양신경총과 연결되어 있다. of중심 부위는 열의 에너지 원칙을 지배하며 불로 상징된다. 또한 시각을 통제하며 창자를 통하여 배설의 활동적 감각을 관할한다. 이 힘 을 깨워 주는 음절 (bija) 은 람(ram) 이 다. 마니푸라 차크라의 도식은 위로 향한 삼각형이고 「붉은 색」으로 대표된다.

\paragraph{만트라(Mantra)} :음절의 조합, 또는 단어의 줄합으로써 특정한 에너지의 진동과 일치한다. 훌릅한 스승밑에서 수행을 시작한 수행자는 만트라를 명상의 대상으로 이용한다. 그리고 시란이 흐르면 이 만트라가 수행자의 명상을 더깊게 만들어준다. 이것은 구루가 전해주는 모든 가르침이 농축된 것이며, 수행자는 명상시에나 일상생활시에 이것에 대한 자팍 (반복수행)를 통해 야만 만트라의 가르침을 펼쳐 받을 수 있다 (그럴 때 라야만 만트라의 숨겨진 정신적이고 영적인 에너지가 방출된다).

\paragraph{만트라요가(Mantrayoga)} :수행의 한 가지 길로써 수행자의 잠재적인 영성을 일깨우기 위해 특정한 구문이나 단어의 음절(미묘한 소리로써 명상의 높은수준에 도달할 때의 진동음)을 명상의 대상으로 이용하는 것. 이 수행시 가장 필수적인 기술은 자파 즉 정신적인 반복이다.

\paragraph{마츠야사나(Matsyasana)} : 물고기 자세

\paragraph{마유라사나(Mayurasana)} .공작 자세

\paragraph{메루단다(Merudanda)} '메루산의 축 또는 메루의 극이란 뜻. 메루산은 지구의 축이 되는 산이고, 메루단다는 사람 의 중심축을 뜻하며, 수슘나가 올라오는 통로인 척추뼈를 의미한다. 이 통로를 따라 쿤달리니 에너지가 그야말로 산 정상을 오르듯이」 의식의 가장 높은 단계 로 상승한다.

\paragraph{물라다라 차크라(Muladhara hakra)} : 「근본적인 기초」 또 는 「바탕으로부터 지지해주는 중심」이란 뜻. 천골이나 골반과 동일한 의미로 척추뼈 아랫부분의 신경계통을 말한다. 이 장소는 활동하지 않는 생 명 에 러지의 근본인 쿤달리니 에너지가 맡치 뱀이 또아리를 틀고 잠을 자는 것으로 상징화되며 그것이 머무는 장소이다. 요가수행의 목적도 바로 이 에너지를 일깨워 상승시키고 정화시켜 궁극에는 깨달음을 얻을 수 있는 의식의 최고의 상태로까지 이르게 하는 것이다. 이 물라다라 차크라는 땅의 에너지 원리를 지배하며 지각적인 감각으로써는 후각을 통제하고, 활동 적인 감각으로써는 발의 감각을 통제한다. 도시화된 표현    으로는 정사각형 으로써 색 깔은 노랑이며 이 센터의 에너지는 르암(lam)이란 음절에 의해 일깨워진다. 종교적인 의식에서 이 에너지 원리의 상징으로 과일이나 향을 바친다.

\paragraph{마니푸라카 챠크라 (Manipuraka-chakra)} : 배꼽 부위에 위치한 신경총

\paragraph{마르가 (Marga)} : 길, 방법

\paragraph{마리치 (Marichi)} : Brahma의 아들 가운데 하나로, 현인이자 Kasyapa의 아버지였다.

\paragraph{마유라 (Mayura)} : 수공작

\paragraph{마첸드라 (Matsyendra)} : 하타 요가 창시자 중 하나

\paragraph{마하 (Maha)} : 강력한, 위대한, 고귀한

\paragraph{만다라 (Mandara)} : 신과 악마가 감로를 찾기 위해서 우주의 대양을 휘저을 때, 젖는 막대기로 사용한 산

\paragraph{만달라 (Mandala)} : 원. Rgveda의 모음(集), 장의 분류를 의미한다.

\paragraph{만두카 (Manduka)} : 개구리

\paragraph{맏스야 (Matsya)} : 물고기

\paragraph{말라 (Mala)} : 화환

\paragraph{메나카 (Menaka)} : 요정, Sakuntala의 어머니

\paragraph{목사 (Moksa)} : 깨달음, 윤회로부터 해탈

\paragraph{무다 (Mudha)} : 어리석고, 우둔한, 혼란스러운

\paragraph{무드라 (Mudra)} : 봉함, 봉합된 자세

\paragraph{묵카 (Mukha)} : 얼굴, 입

\paragraph{묵타 (Mukta)} : 해탈, 깨달음

································································································································································································································

\newpage
\section{나}

\paragraph{나디 (Nadi)} : 신체의 신경체계 와 평행 으로 이어진 미세한 육체의 통로를 비물질적인 생명 에너지가 흐르는 곳이다. 요가 교재에 의하면 약 7만 2,000에서 32만 5,000사이의 미세한 통로가 있다고 한다. 주된 세가지의 것이 이다,핑갈라,또 수슘나이고 이것들은 척추뼈를 따라 흐르며 콧구멍을 통하여 숨의 흐름을 조절해준다.

\paragraph{나디쇼다나(Nadishodhana)} : 「나디들을정화시킨다」는의미 일종의 호흡훈련으로 프라나야마의 좀더 높은 차원의 수행에 대비하여 나디를 정화시키는 것을 뜻한다. 또한 "통로정화" 또는 "호흡교체"라고도 알려져 있으며 내쉬는 숨과 들이 쉬는 숨의 간격 을 없애 고 리 듬을 평 온하고 천천히 함으로써 마음을 잠잠하게하고 호흡을 규칙적 으로 만드는 것이 그 목적이다.

\paragraph{니디드야사나(Nididhyasana)} :명상, 즉 드야나의 ·또 다른 용어로써 산스크리트어이다 즈나나 요가의 묵상이며 지적인 과정의 4단계 중 하나이다.

\paragraph{니르비칼파 사마디 (Nirvikalpasamadhi)} : 「구분이 없는 사마디」 또는 니르비자(nirbija) 「씨았이 었는」 사마디로써 만트라의 음절이나'명상의 대상이 더 이상 필요없는 최고    수준의 사마디이 다. 그 상태에서는 따는 자와 알아진 대상에 대한 구분이 없고, 오직 아는 과정만 있을 뿐이다. 즉 치트(cit) 브라만의 순수한 의식적 측면한 있을 뿐이다.

\paragraph{니야마(Niyama)} : 「관찰」 또는 「수행」 파탄잘리의 요가 수트라에 묘사된 라자 요가의 여턴 가지 중에 두번째 가지인 다섯 가지의 야마에서 도덕 적 안 금기 로 점 차적 으로 수행의 방해되는 습관이나 버룻을 줄여주는 것과 마찬가지 로 니야마의 다섯 가지는 자아의 깨달음을 촉진시키는 습관이나 버룻을 키워준다. 그 파섯가지는 다음과 같다.

   ① 몸과 마음의 청결(샤우차 shaucha)

   ② 스스로 족함 (산토샤 santosha)

   ③ 몸과 마음과 감각의 완벽한 기능을 위해 수행하는 것 (타파스 tapas)

   ④ 스스로 연구함(스바드야야 svadhyaya)

   ⑤ 자아를 위해 이기심을 버리는 것 (이쉬와라 프라니 다나 Ishwara pranidhana) 이다.

       더 자세한 설명은 각각의 단어를 찾아보면 된다.

\paragraph{나바 (Nava)} : 배, 보트

\paragraph{나울리 (Nauli)} : 복부 근육과 기관이 파도치는 것처럼 수평, 수직으로 움직이게 하는 과정

\paragraph{나크라 (Nakra)} : 악어

\paragraph{나타라쟈 (Nataraja)} : 춤의 신, 시바의 이름

\paragraph{니라람바 (Niralamba)} : 지지 없이

\paragraph{니로다 (Nirodha)} : 억제, 압박

·······················
·········································································································································································································

\newpage
\section{옴}

\paragraph{옴(Om)} '만트라들의 최고의 단계. 천상과 우주적인 소리의 원리이며 모든 언어의 모체라 일컬어지고 있파(입을 닫은 상태 에서 어떤 발음을 하더 라도 옴의 발음으로 시 작된다) .   이 옴(Om)의 세 가지 글자 즉 아우음(A. U. M)은 물질적 · 정신적 창조물(prakriti)의 세가지 특징 (gunas)뿐만아니라 모든 삼위일체를 대표한다. 또네번째의 소리냐지 않는 음절이 '있으니 이것은 순수영 혼의 네 번째 단계 로의 승화된투리야(turiya) 또는 사마디를 상정한다.

································································································································································································································

\paragraph{팍다(Pada)} :발 또는 부분 내지는 장(촐)의 뜻. 이 단어는 파탄잘리의 요가수트라에서 네 가지의 장에서 나온다. 「사마디 -락다」는 수행 방법의 장이며 「비 부티 -파다」는 성취   의 장이며 「카이 발랴 파다(Kaivalyapada),는독립의 장(이것은 물질세계로부터의 순수한 영흔을 의미한다)

\paragraph{파드마사나(Padmasana)} :연꽃 자세. 호흠이나 명상수행 자세 중의 하나

\paragraph{파라마트만(Paramatman)}:베단타(Vedanta)철학에서 나오는 궁극적인 자아로써 개인적인 자아의 제한된기능을 나타낸다.

\paragraph{파리치나나(Paricchinna)} .일상적이며, 제한된 다음의 기능이 제한되고, 중단한다는 뜻을 나타낸다.

\paragraph{파치모타나사나(Paschimottanasana)} '뒤로 뻗치는 자세, 머리서 부터 무릎까지 이어지는 자세로등의 근육과 뒷 다리 근육온 뻗쳐내는 자세

핑갈라(Ping히a) :나디 즉 척추 둥을 따라 평행하게 이어진 에너지 채널, 이 것은 오른쪽 콧구멍 을 통하여 호흡을 조절 해주고 이 통로가 활동적일 때는 인간의 성격이 이성적이   고 활동적 이며 정열적 으로 된다. 또한 몸속에 온기를 느낄 도 있게 된다. 이 프라나의 효과는 남자 다움과 밝음이 라 할 수 있으며, 왼쪽의 흐름의 특성은 어두움과 여자의 마음  이라 할 수 있다.

\paragraph{라크리티 (Prakriti)} : 「나아가게 하는 것」이란 뜻이다. 상키 야나 요가철학에서는 물질적 이고 정신적 인 요소를 순수한 영혼이 착각하게 만드는 것은 에고 또는 「자신을 만드는   자(ahamkara).라 한다. 요가의 목적은 이 프라크리티로 부터 푸루샤를 분리시키는 것 (kaivalya)으로 순수한 영혼이 스스로를 현혹됨이 없이 확인하는 것이다. 크라크리티에서는세가지의 속성 (gunas) 혹은 경향이 있다. 균형 또는 순수함(사트바 sattva)가 첫째이며, 에너지(라자스 rajas)이 두번째이며, 무기력 (tamas)이 세번째이다. 우주에 존재하는 모든것은 이 세 가지 성 질이 조합된 것이 다.

\paragraph{프라나(Prana)} :섬세하고, 비물질적인 형태로 존재하는 모든 생명에 내재된 생명력. 이 프라나는 「미세한 육체」를 구성하는 에너지 통로를 따라 흘러다닌다. 이러한.·흐름이 인체 에 독특하게 미치는 영 향은 특별히 이름지워졌고 알려 지기 로 다섯 (해 로는 열개의)겸의 프라나로 알려져 친다. 즉 프라나(prana) , 아파나(apana) ,사마나(samana) , 우다나(udana), 또브야나(vyana)둥이다. 여기서 프라나라는 에너지 흐름은 들이쉬는 숨을 규칙적으로 해주며 아파나 (apana)는 배설과 내쉬는 숨을,사마나 (samana)는 소화와 영양, 에너al의 분배를, 우다나(udan꿀는 프라나보다 좀 더 고조된 운동으로 기침, 재채기, 연동운동, 사망 둥에 관여한다. 브야나 (vyana)는 전체 골격, 근육,신경 조직, 혈액통제,긴장,이완에 광범위하게 관여한다.

\paragraph{프라나야마(Pranayama)} : 호흡을 점차로 깊게 하고 또한 통제할 수 있게 하기 위한 과학으로, 그 목적은 고도의 수행을 했을때 미세한 육체를 통하여 일어나는 프라나의 움직임을 통제할수 있는 통제력을 얻기 위함이다. 파탄자리에 의해 기술된 요가체계의 여덟 단계 중 네번째 단계

\paragraph{프라트야하라(Pratyahara)} :요가의 8가지 중 다섯번째 가지. 프라트야하라는 감각에서 물러나 내것을 통제하는 것이며 감각에 의해 마음이 흐트러지는 것을 막아준다.

\paragraph{프리티비 (Prithivi)} : 지 구

\paragraph{푸루샤(Purusha)} :상키야 철학에서 말하는 순수한 영혼으로 모든 것에 내재하는 인격. 그것은 정신적 ·물질적인 것 (prakriti)의 가까이에서 존재하며, 이 때문에 에고에 의해서 올바르지 못한 분별력을 갖기 쉽다. 요가의 궁극적인 목적은 이 두가지를 구분하여 순수한 영혼으로 분리하여 존재시키는데 있다(독짙된 존재 :카이발랴).

································································································································································································································

\newpage
\section{라}

\paragraph{라자 요가 (Raja-yoga)} : 내부의 적을 이겨 자신의 마음을 지배함으로써 이루는 지고의 우주 정신과의 합일

\paragraph{라자 카포타 (Raja-kapota)} : 비둘기 왕

\paragraph{라자식 아함카라(Rajasik Ahamkara)} :활동적인 에고. 라자스 또는 에너지의 성질 에고(e90)

\paragraph{라자요가(Rajayoga)} : 「왕도」를 뜻함. 라자요가는 요가철학의 전형적인 체계이고 파한팍리에 의해 요가 수트라에 성문화된 것이다. 또한 8개의 가지로 이루어진 요가(아쉬탕가 ashtanga) 라고도 불리는데, 그 이유는 8단계로 나누어지는 요가이기 때문이고 이중 몇몇은 신중을 기해 특정 분야별 수행 으로 나뉘어져 있다. 그 예 로 하타 요가 (정신 · 육체연구의 과학)는 자세의 분야인 세번째 가지에서 발전된 것이다. 라자 요가는 그 마지막에 네개의 가지에 의해 대 변된다. 프라트야하라인 감각의 통제, 드야나는 명상, 사마디 즉 초의식의 명상

\paragraph{레차카 (Rechaka)} : 날숨, 폐를 비움

\paragraph{리쉬 (Rishi)} :선지자, 특히 그 자신의 만트라가 나타내어진  선인들


\newpage
\section{사}

\paragraph{사드하나(Sadhana)} . 「성취」 또는 「만족」의 사드하나라는 말은 깨달음을 얻기 위한 수행의 특정한 길을 성실히 따르는 수행자에게 쓰이는 말이 다. 이것은 요가 수트라의 두번 째 장(촐)인 「사드하나 파다」의 주제이다.

\paragraph{사하스라라(Sahasrara)} : 「수천개의 잎을 가진 연꽃」으로 상징되는 의식의 최고센터로 육체의 두뇌 안에 위치한다. 이 사하스라라에 쿤달리니가 이르르면 만트라의 도움이 필요없는 니르비자(nirbija)의 사마디 즉 니르비칼파 (nirvikalpa)가 이루어진다. 상스카라 즉 미래의 행동근원은 최고의 지식 앞에서 승화되고 수행자는 행탈을 위한 길을 걷게 된다.

\paragraph{사마디 (Samadhi)} :요가의 8단계중마지막 단계로써 초의식의 상태를 말한다. 사마디의 종류 중 사비자 (sabija) 나 사 비칼파(savikalfa)와 같이 미리의 대상을 정한 사마디의 경우는 수행자 자신이 초의식의 마음은 그 명상의 대상을 경험하게 되며 비자(bija), 즉 선정된 대상은 정확히 그 진실된 본성으로 나타나게 된다. 그것이 아닌 또 다른 차원의 사마디에 이르게 되면 명상의 대상이 더 이상 필요 없게 되고, 수행자는 자기 자아 스스로에 의한 지식을 깨닫게 되고 또 그 자아는 우주적자아(브라만)와 일치된다. 이 단계에서는 아는 자와 아는 대상의 구분이 없어지게 된다 (비칼파) 다만 완벽한 지식 즉 안다는 자체만 존재할 뿐이  다. 사마디는 파탄잘리의 요가수트라에 첫번째 장의 주제이다.

\paragraph{사마나(Samana)} :프라나의 다섯 흐름중의 하나. 사마나는 말은 깨달음을 얻기 위한 수행의 특정한 길을 성실히 따르는 수행자에게 산이는 말이다 이것은 요가 수트라의 두번째 장(후)인 「사드하나 파다」의 주제이다.

\paragraph{사하스라라(Sahasrara)} : 「수천개의 잎을 가진 연꽃」으로 상징 되는 의식의 최 고 센터 로 육체의 두뇌 안에 위치한다. 이 사하스라라에 쿤달리니가 이르르면 만트라의 도움이 필  요없는 니르비자(nirbija)의 사마디 즉 니르비칼파 (nirvikatpa)가 이루어진다. 삼스카라 즉 미래의 행동 근   원은 최고의 지식 앞에서 승화되고 수행자는 행탈을 위한 길을 걷게 된다.

\paragraph{사트빅 (satvik)} :상키야 철학에서 적용되는 물질적인 존재의 세 가지 속성 중의 하나로서 그 조화됨과 순수함과 균형감에 사로잡히는 것.

\paragraph{사트야(Satya)} :야마의 다섯 가지 중 두번째의 금기. 자신과 남에게 정직하다는 것.

\paragraph{사비 칼팍 사마디 (Savikalpa Samadhi)} : 사마디의 낮은 단계로 수행자가 수행 대상이나 비자Ibija), 즉 미리 주어진 명상 대상 둥과 직접 하나가 되어 실제적으로 결험해보는것. 파탄잘리의 요가 수트라의 첫번째 장에 들장하는 사비칼파 사마디에는 여텁 가지가 있다.

\paragraph{세투-반다사나(Setu-bandhasana)} : 「교각 자세」

\paragraph{삭티(Shakti)} :산스크리트어의 「가능한」이란 단어 삭(sak)에서 유래된 말로써 뜻은 힘,에너지, 능력이다. 흔히 활동 적인 신적인 능력이나 에너지 원리를 일컫는데 쓰인다.  예를 들어 쿤달리니는 치트 삭리 즉 의식의 힘 또는 힘을 가진 의식으로 불리운다.

\paragraph{사라바사나(Shalabhasana)} : 「메뚜기 자세」

\paragraph{사우차(Shaucha)} · 니야마의 첫번째 덕목. 도덕적 수행이며 몸과 마음과 영적인 정화의 과정으로 개인의 모든 부분의 기능을 을바로 나아가게 한다.

\paragraph{사바사나(Shavasana)} :송장 자세. 이완의 자세

\paragraph{쉬르시 아사나 (Shrishasana)} : 머리로 몸을 지탱하고 꺼꾸로 서는 것.

\paragraph{시바(Shiva)}
힌두교의 세 가지 속성을의미하는세 신 중의 하나로 시바는 세번째의 신이다. 
해체를 의미하며 그를 통하여 모든 것은 본질적인 속성의 것으로 환원된다  (tamas). 
또 시바는 쿤달리니 즉 치트 삭티가 그 힘을 활성화시키는 의식의 원칙도 지배한다. 
요가 수행의 탄트릭 체계의 목적은 생명에너지를 「수천의 꽃잎이 달린,연꽃모양」으로 형성화되는 최고의 에너지 센터까지 끌어올려 말 그대로 최고의 깨달음에 자신을 융화시키는 것이다.

\paragraph{시디스(Siddhis)} : 「도달된」 또는 「힘」으로 수행자가 높은 수준히 요가 수행을 했을 때 나타나는 능력이다. 이것은 좀 위험한 유혹이 될 수도 있고 수행의 장애가 될수도 있다. 그것들은 때로 「비부티스(vibhtis),라고도 불리며 요가 수트라의 세번째 장 「비부티 파다(Vibhilti-pada).에 기술되어 있다.

\paragraph{시다사나(siddhasana)} :완성의 자세. 앉은 자세로써 호흡수행이나 명상수행시에 사용

\paragraph{수카사나(Sukhasana)} : 좀 쉬운 자세로써 호흡수행이나 명상수행시 사용

\paragraph{수슘나(Sushumna)} :의식의 에너지 (cit)나 생명 에너지(jiva)가 흐르는 중앙의 통로 척추를 따라 흐르는 에너지 통로의 3개 중 하나로써 수행 전에 예비로 호흡수행을 하는것은 바로 이 수슘나의 중앙 통로를 개방해서 양쪽 코로 같이 호흠을 균둥히 하고 마음이 자연스럽게 명상할 수 있는 기쁨의 상태로 진입하게끔 하는 것이다.

\paragraph{수트라(Sutra)} : 「실」온의미하며 수트라는철학자들이 그의 세계의 주된 사상이 진보해나가는 것을 기록해둔 짤은 금언같은 것이 다. 즉 모든 필요 없는 문구를 삭제시킨 간결한 문구이다 수트라 중에서도 특히 구전되어 내려오는 것은 주석을 달지 않는 한 쉽게 이해하기 힘들다.

\paragraph{스와디스타나 차크라(Swadisthana Chakra)} :의식의 중심부위로 「의식의 거주지」라고 불리며 생식기 바로 위쪽에 위치한 신경계와 동일하다. 이곳의 에너지의 원칙은 액성으로 물로써 상징된다. 맛의 지각적 감각과 양손의 활동적감각을 통제하고 초승달 형상으로 도식화되며 색깔은 우유빛이다. 이 스와디스타나 차크라의 대상이 되는 음절은 브암(vam) 이 다.


\paragraph{와드야야(Swadhyaya)} : 니야마스(도덕적 금기)의 다섯가지 중네번째 것. 이 스와디야야를통해서 스스로의 연구를 수행하게 된다. 여기서는 경전에 대한 공부도 포함이 되고 자신의 마음을 명상에 대 한 추가를 할 수 있게 해 주는 영적인 대상에 대한 연구도 포함되며, 수행자 스스로의 일상생활에 대한 검토나 수행자의 만트라에 대한 끝없는 반복도 포함된다. 그러 한 후에 도 이 성 적 으로 용인되는 진리를 시험해 볼수있게 되고 직관으로 향하는 이지적인 이성의 상장도 시험해보게 된다(즈나나 요가 참조).

\paragraph{스와스티카사나(Swastikasana)} :행운의 자세, 호흡수행이나 명상수행시 적용되는 앉는 자세

\newpage
\section{탈}

\paragraph{탈루(Talu)} :인간의 두개골의 바닥 부분과 일치하며 미세한 육체의 구개 부위를 말함.

\paragraph{타마식 (Tamasic)} :타마스 즉 타성적인 속성으로써 물질적인 것의 세 가지 창조된 특질 중의 하나

\paragraph{타파스(Tapas)} : 불 또는 뜨거움을 의미하는 단어로 나야마 즉도덕률의 하나로써 육체나 마음, 감각에 대해 엄격해야 하고 깨달음에 대한 강한 결심을 북돋아 주는 것.

\paragraph{타트바(Tattva)} : 「존재하고 있다」는 의미로써 사물의 원초적이고 본능적인 속성. 상키야 철학에서는 타트바스를 최고의 프라크리티, 특히 사물의 다섯 가지 물질적 속성 즉 지구의 흙의 성질, 물의 액성 등과 같은 것들이 방출된것이 라 설명한다.

\paragraph{테자스(Tejas)} :마음으로'부터 발산되는 찬란한 지혜

\paragraph{투리야(Turiya)} : 「제4의 것,으로 깨어 있고, 잠자고, 꿈꾸는 세 가지 속성에서 진보한 초의식의 상태

\paragraph{트라타카(Trataka)} :집중력을 강화시키기 위해 한 곳을 응시하는 수행

································································································································································································································
\newpage
\section{우}

\paragraph{우다나(Udana)} :프라나의 다섯 가지 큰 흐름 중의 하나. 후두북 윗쪽의 육체 를 통제하며 우리 의 감각 (시각, 청각, 후각 미각)과 기침, 재채기, 연동운동 등의 기능을 다스린다.

\paragraph{우파니샤드(Upanishad)} :베다의 가장 철학적인 부분이며, 베다 저서 중에 가장 최초의 철학. 베다의 가르침의 정 수는 이 글에 담겨져 있다.

································································································································································································································

\newpage
\section{비}

\paragraph{바유(Vayu)} :공기,기체 즉 공기의 에너지 원칙을 의미하는 산스크리트어. 때로는 프라나라고도 쓰인다.

\paragraph{베단틴(Vedantin)} .베단타 철학의 사도들의 용어이며, 흔히 아드바이타-배단타 철학의 지지자들을 의미하는데 아드바이틴스라고도 불린다.

\paragraph{비부(Vibhu)} :글자그대로 「모든곳에 충만한」이란뜻. 바로 이 「모든곳에 충만한」 자아의 속성을 통해 수행자는 자신의 환경을 통제하고 그가 기적적이라고 생각하는 능력,  즉 비부티스(시디 참조)를 지배하게 된다. 또 비부라는 이 말은 모든 인류가 궁극적으로 우주의 마음에 연결되어 있다는 이론을 의미하기도 한다.

\paragraph{비슈다 차크라(Vishuddha Chakra)} : 「정화된」의 뜻. 이 센터는 척추를 따라서 목구멍의 밑부분의 위치와 일치한다.  이 차크라는 공간의 원리 를 지배하고, 원으로 도식 화되며 ,  상징의 색깔은 푸른잭이다. 이것을 상징하기 위해서 바치는 것은 봉오리가 열리 고 닫히는 꽃이 다. 이 차크라가 통제하는 지각적 감각은 청각이고 활동적 감각은 입의 감각이며 깨우침을 주는 음절은 밤(ham)이다.

\paragraph{브리스치카사나(vrischikasana)} 
		전갈 자세

\paragraph{브리티 (Vritti)} : 「파도」 혹은 「순화」를 의미한다. 깊은 잠재의식의 저장손인 치타로부터 발생한 「파도」 즉'브리티스에 의해 만들어 져 이리저리 마음속에서 표류하는 생각을 통제  하는 훈련을 말한다. 요가수트라에 정의된 요가는 이런 치타 브리티스를 수행하는 것.

\paragraph{브야나(Vyana)} '프라나의 다섯 가지 주된 흐름 중의 하나. 브야나는 전체 두뇌부와 근육,신경,순환적인 육체에 모두 퍼져 있으며 긴장과 이완을 규칙적으로 만들어준다.


\newpage
\section{야}

\paragraph{야마(Yama)} :금기. 요가에는 라자 요가의 여덟 가지 단계 중 첫번째 단계 에 다섯 가지 금기가 등장한다. 이런 금기의 목적은 영적 진보에 지장을 주는 버릇이나 습관을 점차적    으로 없애고 수행자가 평온한 의식 을 지니게 하는 것이다.

   이 다섯가지 금기는 다음과 같다.

   ①생각과 언행과 행동에 있어 난폭해지지말라(아힘사 ahimsa)

   ② 참될것이며 거짓을 하지 말라(사트야 satya)

   ③ 남의 것을 탐하거나 훔치지 말라(아스데야 asteya)

   ④ 자신의 것에 집착하거나 독점하지 말라(아파리그라하 aparigraha)

   ⑤ 행동과 지각적 감각과 마음에 있어 항상 순결하라 (브라마차리아 brahmacharya) 

       (더 자세한 설명을 위해 다섯가지 야마스가 따로 각기 설명되어져 있다. )

\paragraph{요가무드라(Yoga Mvdra)} :요가의 상징. 특별한 자세

\paragraph{요가(Yoga)} :인도철학의 학과. 상키야 철학과 가깝게 연관되어 철학적인 교의에 철학적 진리를 수행자 자신이 육체적 ·정신적 ·영적으로 완전하게 확실히 체득을 하게 하는 실천적인 체계를 첨가하는 특징이 있다. 보편적으로 인간의 잠재력을 개발하는 정확한 과학이며 수많은 수행자들을 통하여 5천년 동안에 임상실험 을 통하여 확인된 바가 있는 것이다. 이것을 성문화한 것이 파탄잘리 성인이었으며, 기원전 2세기경이었다. 요가 단어는 산스크리트어의 유즈 (yuj)가 어원이며 뜻은 결합하다 또는 적용하다는 것이다.

요가의 의 미는 하나로 된 다이며 체계적으로 실천을 적용하고 결과와 이익 을 점검하는 것이다. 
이 요가의 적용에 대한 감각을 훈련이란 뜻하며 즈나나 요가는 자식의 훈련이라  한다.

\paragraph{요가 수트라(Yoga Sutra)}
196개의 경구로 되어 있는 라자요가 즉 왕도(포볼)이며, 기원전 2세기경에 파탄잘리로부터 구성되어 있으며, 요가 철학의 수행적 요구와 철학의 근원이 모두 담겨져 있는 기본적인 형식의 경전이다.



출처: http://l2j.co.kr/2485 [L2J with DuraBoys]



%	================================================================== Part			산스크리트어
	\addtocontents{toc}{\protect\newpage}
	\part{관련 인물}
	\noptcrule
	\parttoc				

% ========================================================================================= chapter
\chapter{인도}
\minitoc



% -----------------------------------------------------------------------------
%
% -----------------------------------------------------------------------------
	\section{크리쉬나 마차리아 }


% -----------------------------------------------------------------------------
%
% -----------------------------------------------------------------------------
	\section{파티비 조이스}



% -----------------------------------------------------------------------------
%
% -----------------------------------------------------------------------------
	\section{B.K.S. 아헹가 }

15세에 요가에 입문하고 17세부터 요가를 가르치며 치열한 수행을 통해 현대 하타요가의 살아 있는 전설이 된 요가의 대가. 자신의 수행 체험에 의해 파탄잘리의 『요가 수트라』를 해석하여 독특한 방식으로 70년 이상 세계 곳곳에서 제자들을 가르쳐 왔다. 그가 현대화 하고 정련시킨 ‘요가’를 아헹가 웨이(Iyengar way)로 불리며, 정확한 자세를 바탕으로 온몸을 각성시키고 뼛속 깊은 곳으로부터 긴장을 풀어줄 뿐 아니라 보조기구를 사용하여 질병을 치료하는 데 있어 탁월한 효과를 가지는 수행 체계로 현재 세계에서 가장 널리 수련되고 있다. 인도를 비롯한 세계 각 국에 아헹가 요가 수련 센터가 있으며, 고전적 요가 입문서로 전무후무한 요가서의 베스트셀러인 『요가 디피카』외에 『아헹가요가』,『요가 호흡 디피카』,『요가 수행 디피카』등 10여 권이 넘는 주옥같은 요가 관련 저서들을 저술하여 요가의 길을 가는 수행자들에게 올바른 이정표를 제시해 주고 있다. 
이러한 업적으로 「유엔 평화 헌장의 과학 박사」, 「미국 전기 협회」의「올해의 요가 교육자상」,「세계 연합 전인 치유 의학회」의「Purna Swasthya상」을 비롯한 많은 상을 수상했다. 2004년 타임지에 의해 세계에서 가장 영향력있는 100인중 한사람으로 선정되다.


% -----------------------------------------------------------------------------
%
% -----------------------------------------------------------------------------
	\section{스리 K 파타비 조이스 }


% ========================================================================================= chapter
	\chapter{한국}
	\minitoc

% -----------------------------------------------------------------------------
%
% -----------------------------------------------------------------------------
	\section{현천玄天 스님}

대학 시절 요가에 입문하였으며, 백양사 승가 대학 사집과를 수료하고, 동국대학교 불교대학원[禪學]과 불교 서울 전문 강당(2년)을 졸업했다. 선방을 다니다가 해탈 도구로 육신의 중요성을 느끼고 히말라야 리시케쉬 및 인도의 여러 수행처에서 요가를 배웠으며, 특히 하타 요가 세계 제일의 도장인「아헹가 요가 연구소」에서 최고급 과정을 20년 동안 10여 차례 수료하고 상급 자격증(Advanced Level) 취득.
이후 백담사 무문관 3년 결사 회향을 비롯하여 제방에서 참선하면서 요가가 禪 수행에 지대한 도움이 됨을 체험하고 현재는 아헹가 선생의 저서를 우리말로 옮기는 작업을 하고 있으며 또 학교에 요가를 보급하는데 힘쓰고 있다.
제9교구 본사 동화사 교무국장을 역임했으며, 아헹가 요가센터와 청량 선원을 운영하면서 사단법인 아헹가 요가 협회장을 맡고 있다.

\paragraph{역서}
역서로 요가의 고전이라 불리는 『요가 디피카』와 『아헹가 행법 요가』,『요가 호흡 디피카 (공역)』,『요가 수행 디피카』,『초급 아헹가 요가(공역)』가 있다. 





% -----------------------------------------------------------------------------
%
% -----------------------------------------------------------------------------
	\section{홍승준}


% -----------------------------------------------------------------------------
%
% -----------------------------------------------------------------------------
	\section{김윤}



% -----------------------------------------------------------------------------
%
% -----------------------------------------------------------------------------
	\section{권익선}


		spine자연치유센터원장

		스파인 자연치유 8단계 요가

		\href{http://cafe.daum.net/NHyoga}{다음 카페}



			스파인(Spine)자연치유요가센터 원장, 세계자연치유운동협회 이사 겸 카이로프랙틱 연구위원이다. 
			병약한 어린 시절을 보낸 저자는 강한 사람이 되어야 잘 살 수 있고 행복해진다는 생각으로 태권도, 합기도, 검도, 국궁 등 다양한 무술을 섭렵했다. 
			특히 전통무예에 매료되어 택견과 조선검을 다년간 수련하면서 전국 대회에 참여하기도 했으나 스파링 도중 부상을 당하며 무예를 통해 강한 몸과 정신을 유지하는 것에 한계를 느꼈다. 
			다친 몸과 약해진 체력을 극복하기 위해 몸 수련법과 정신수련법에 관심을 가지게 되면서 요가, 명상, 단전호흡 등 다양한 치료법을 공부했다. 
			사이비 명상수련단체와 엉터리 요가 지도에 회의를 느낄 무렵, 올바른 요가호흡에 대해 알게 되어 온갖 방법으로도 치유되지 않던 비염을 스스로 극복했다. 
			요가와 카이로프랙틱을 통해 다양한 수련법들의 공통적인 원리를 정리하고 통증 치유는 물론 몸과 삶을 변화시킬 수 있도록 ‘스파인 자연치유 8단계 요가’와 척추강화요가 ‘스파인 시퀀스’를 정리했다. 
			저자는 요가와 호흡이 단지 다이어트를 하거나 유연한 몸을 만드는 데에만 유용한 것이 아니라 아프고 힘든 사람들, 특히 중년(그리고 노년)이 되어 건강에 자신이 없는 모든 사람들에게 효과적이라고 말한다. 
			아울러 신체 부위별 통증을 잡기 위한 몸풀기를 비롯하여 코어 근육을 강화하는 확장호흡, 몸을 강하게 만드는 자세연결 시스템 요가를 소개한다.


		\href{https://www.youtube.com/channel/UCmvG5Zebt11jTuEj55ItkSg/featured}{유튜버}



% ----------------------------------------------------------------------------- 정두화 
%
% -----------------------------------------------------------------------------
	\section{정두화 바유 Vayu}

\paragraph{1996년}
전직 약사인 바유는 약뿐만 아니라 삶의 문제를 해결하는 처방에 대한 다양한 모색과 실험 끝에 1996년부터 요가여행을 시작했고 1999년부터는 매일 연습하는 삶을 선택하였다.
\paragraph{2001년}
2001년 부터는 요가의 뿌리를 찾아 인도를 여행하다가 마침내 2005-2007년 인도 마이솔에서 구루지, 쉬리 파티비 조이스를 통해 아쉬탕가 요가의 전통과 연결되었다.
\paragraph{2008년}
2008년 북경에서 존 스콧 선생님을 만나 빈야사의 참의미에 눈뜨고 아쉬탕가 요가를 번역 출간 하였고, 
\paragraph{2009년}
2009년 뉴질랜드 넬슨의 스틸포인트 요가센터에서 아쉬탕가 요가 지도자 과정을 졸업하였다.
\paragraph{2007년}
2007년부터 고향인 부산에서 정통 마이솔 수업을 열고 
\paragraph{2010년}
2010년 해운대 요가 VnA를 설립하면서 아쉬탕가 빈야사 요가 전통의 과학적인 현대 해석을 위해 노력하고 있다.

\paragraph{2012년}
2012년부터 요가쿨라에서, 2014년부터 존 스콧 차이나 요가 지도자 과정에서 요가 해부학 강의를 담당하고 있으며, 
요가 해부학 워크샵의 연구 주제들을 10년간의 소그룹 수업의 체험에 비추어 자세 타입에 따른 요가 교정의 이론과 방법을 정립하고 있다.


\paragraph{이메일} 이메일	\href{8jiyoga@live.co.kr}{8jiyoga@live.co.kr}
\paragraph{블로그} 블로그 	\href{https://blog.naver.com/jdh3653} {blog.naver.com/jdh3653}
\paragraph{홈페이지} 홈페이지	\href{http://www.yogakula.co.kr/}{요가 쿨라 www.yogakula.co.kr}





% ========================================================================================= chapter
	\chapter{외국인}
	\minitoc

	\section{존 스콧 John Scott}





% ========================================================================================= chapter
	\chapter{루마니아}
	\minitoc


	\section{미르체아 엘리아데}



% -----------------------------------------------------------------------------
%
% -----------------------------------------------------------------------------
	\section{키노 맥그리거}

\paragraph{}
키노 맥그레거는 아쉬탕가 요가를 정립한 스리 K 파타비 조이스에게서 아쉬탕가 요가를 가르칠 수 있는 자격을 인증받은 14명의 미국인 중의 한명이며, 그 가운데 가장 젊은 여성이다.

\paragraph{}
그녀는 남편 팀 펠트만과 함께 마이애미 라이프 센터를 설립하였고, 그곳에서 아쉬탕가 요가의 매일 수련, 위크샵, 집중 수련을 지도하고 있다.
세계 최대의 요가 잡지인 요가 저널은 요가의 떠오르는 최고의 별 21명 중 한 명으로 소개하였다.

\paragraph{}
그녀는 매년 15개 이상의 나라를 돌아다니면서 요가를 지도하고 있고, 허핑턴 포스트에 글을 기고하며 The power of Astanga Yoga 등 네권의 책을 저술하였다.



\paragraph{홈페이지}
	\href{http://www.kinoyoga.com}{홈페이지}



\paragraph{유튜브 채널}
	\href{http://www.youtube.com/kinoyoga}{유튜브}






%	================================================================== Part			산스크리트어
	\addtocontents{toc}{\protect\newpage}
	\part{관련단체}
	\noptcrule
	\parttoc				



% ========================================================================================= chapter
	\chapter{관련단체}
	\minitoc

% -----------------------------------------------------------------------------
%
%
%
% -----------------------------------------------------------------------------
	\newpage
	\section{사단법인 한국아헹가요가협회 }



% -----------------------------------------------------------------------------
%
%
%
% -----------------------------------------------------------------------------
	\newpage
	\section{부산요가명상원}


최진태  \\ 
부산요가지도자교육센터 \\
(부산요가명상원) 원장




% ====== ====== ====== ====== ====== ====== ====== ====== ====== ====== ====== ====== ====== ====== ====== ====== ====== ====== ====== ====== ====== ====== ====== ====== ====== ====== 요가잡지
	\chapter{요가 잡지}
	\minitoc



	% -----------------------------------------------------------------------------
	%
	%
	%
	% -----------------------------------------------------------------------------
		\section{요가 잡지}
	
	
		\section{요가 저널}




% ====== ====== ====== ====== ====== ====== ====== ====== ====== ====== ====== ====== ====== ====== ====== ====== ====== ====== ====== ====== ====== ====== ====== ====== ====== ====== 요가강습원
	\chapter{요가 강습원}
	\minitoc


	% -----------------------------------------------------------------------------
	%
	% -----------------------------------------------------------------------------
		\section{요가 강습원}
	
		\section{요가 강습원 : 부산 지역 }

		\subsection{부루 요가 명상 센타 : 동구 초량동 }

부산 동래구 아시아드대로 256 (우)47837
지번온천동 1421-1
연락처
051-502-2786 대표번호

		\clearpage
		\begin{figure}		
		\centering			
		\caption{	부루 요가 명상 센타}				
		\includegraphics[width=0.9\textwidth] {./fig/20190710_114002.jpg}			
		\end{figure}		
		\clearpage





		\subsection{구루 요가 : 연제구 거제동 }

	
		\clearpage
		\subsection{샨티 요가 센타 : 동구 초량동 }

		\clearpage
		\begin{figure}		
		\centering			
		\caption{	샨티요가 센타}				
		\includegraphics[width=0.9\textwidth] {./fig/20190701_164832.jpg}			
		\end{figure}		
		\clearpage


	% -----------------------------------------------------------------------------
	%
	% -----------------------------------------------------------------------------
		\section{요가 강습원 : 김해 지역 }
		\subsection{김해 지역}
	

		\section{요가 강습원 : 울산지역}
		\subsection{울산 지역}
	
		\section{요가 강습원 : 양산지역}
		\subsection{양산 지역}
	





% -----------------------------------------------------------------------------
%
%
%
% -----------------------------------------------------------------------------
\newpage
\section{요가 용품}


% -----------------------------------------------------------------------------
%
%
%
% -----------------------------------------------------------------------------
\newpage
\section{허거머거}

\subsection{허거머거}



	\href{http://www.huggermugger.co.kr/}{허거머거}

\subsection{붓다팬츠코리아}

	\href{http://it-store.co.kr/main/index.php}{붓다팬}



% -----------------------------------------------------------------------------
%
%
%
% -----------------------------------------------------------------------------
\newpage
\section{EVA YOGA}


\subsection{[EVA YOGA]}


[EVA YOGA]10분 카디오 CARDIO (유산소+근력+전신) 요가플로우 2
에바요가EVAYOGA
조회수 5.4천회


9:53
[EVA YOGA]요가타올 사용설명서
에바요가EVAYOGA
조회수 5.5천회


9:35
[EVA YOGA]다이어트 무기 소환 3탄(신의 선물/디톡스/지방분해/레몬 비타민C의 30배/깔라만시)
에바요가EVAYOGA
조회수 2회
새 동영상


20:21
[EVA YOGA]4주완성 2주차 도전! 다리 찢기(박쥐자세)
에바요가EVAYOGA
조회수 1.7천회
새 동영상


13:15
[EVA YOGA]10분 카디오 CARDIO (유산소+근력+전신) 요가플로우 3
에바요가EVAYOGA
조회수 4.6천회


10:18
[EVA YOGA]EVA의 요가매트
에바요가EVAYOGA
조회수 1.2만회

[EVA YOGA]30분 디톡스 빈야사 요가 30min detox vinyasa yoga
에바요가EVAYOGA
조회수 18만회

[EVA YOGA]10분 카디오 CARDIO (유산소+근력+전신) 요가플로우 1
에바요가EVAYOGA
조회수 9.1천회

[EVA YOGA]24분 젠틀플로우 요가(손목통증 없이)
에바요가EVAYOGA
조회수 8.6천회


\subsection{[EVA YOGA]}



구독! 좋아요! 많이 부탁드립니다. 구독은 클릭▶https://goo.gl/1nWbJv
반갑습니다. 요가 강사 EVA입니다. 이번 영상은 수리야나마스카라의 정확한 호흡법과 동작을 설명합니다. 수리야 나마스카라만으로도 충분한 수련이 되는건 익히 알고 계시죠? 빈야사 요가 역시 수리야나마스카라를 기본으로 합니다.
자! 준비 되셨나요?  지금 시작합니다.









%	================================================================== Part			힌두교
	\addtocontents{toc}{\protect\newpage}
	\part{힌두교}
	\noptcrule
	\parttoc				



% ------ ------ ------ ------ ------ ------ ------ ------ ------ ------ ------ ------
	\chapter{힌두교}
	\minitoc				



% ----------------------------------------------------------------------------- section
%
% -----------------------------------------------------------------------------
	\section{힌두교 개요}


옴(AUM): 힌두교에서 부르는 깨닫는 소리 \\

\paragraph{}
힌두교(산스크리트어: Sanātana Dharma 사나타나 다르마, 영어: Hinduism )는 남아시아에서 발생한 종교로 인도를 비롯한 남아시아에서 널리 믿어지고 있는 종교이다. 
힌두교 신자들은 자신의 종교를 칭할 때 영원한 다르마라는 의미의 \textbf{사나타나 다르마(Sanātana Dharma)}라고 부르는 것이 일반적이다.

\paragraph{인도교}
인도교(印度敎)라고도 한다. 
힌두교를 범인도교라 함은 힌두(Hindū)는 인더스강의 산스크리트 명칭 ‘신두(Sindhu:大河)’에서 유래한 것으로, 인도와 동일한 어원을 갖기 때문이다. 

\paragraph{}
힌두교라고 할 때는, \textbf{민간 힌두교 전통}과 \textbf{베다 힌두교 전통}으로부터 비슈누파와 같은 \textbf{박티 전통}에 이르기까지 다양하고 복잡한 전통들 전체를 뜻한다. 
또한 \textbf{요가 전통}과 \textbf{카르마 개념}에 기초한 매일의 도덕적 삶과 힌두 결혼 풍습과 같은 사회적 일반 규범도 포함된다.

\paragraph{}
힌두교는 기독교와 이슬람교 다음 가는 세계의 큰 신앙이다(신자수는 2005년 9억 4천만명). 
힌두교의 발생은 고대 인도의 종교 사상인 베다에서 비롯되며, 베다의 사상은 기원전 1500년 이전으로 거슬러 올라간다. 
베다는 종교로서 그리고 글로서 오늘날 남겨진 문학 가운데 가장 오래된 것으로 여겨지고 있다. 
힌두교는 여러 신들의 존재를 부정하지 않는 다신교적 일신교(택일신교 또는 일신숭배)로서, 교주(敎主) 즉 특정한 종교적 창시자가 없는 것이 특징이다.


\section{힌두교의 개념}



\section{힌두교의 명칭}

\paragraph{영원한 법칙 - 사나타나 다르마}
힌두교로 번역되는 영어 힌두이즘(Hinduism)이라는 명칭은 영국이 인도를 식민지로 삼으면서 인도의 종교에 자의적으로 붙인 이름이다. 
원래 힌두교 신자들은 자신들의 종교를 힌두교라고 부르지는 않았으며, 영원한 다르마(법칙)라는 의미의 사나타나 다르마(산스크리트어: Sanātana Dharma)라고 불렀다. 
산스크리트어 사나타나 다르마는 힌두교의 기본 교의를 지칭하는 말이다. 
사나타나(Sanātana)는 영원하다는 뜻이며 다르마(Dharma)는 법(法) 또는 법칙으로 번역된다.

\paragraph{힌두이즘}
힌두이즘이란 말은 서양에서 종교를 포함한 인도 문화의 특질을 총칭한 표현이다.
따라서 가장 넓은 뜻의 힌두교는 인도에서 행해지는 모든 종교가 포함된다.
그러나 일반적으로 힌두교라고 말할 경우에는 외래 종교인 이슬람교 · 조로아스터교(특히, 파시교) · 기독교 등이 제외되며 또한 인도 고유의 종교 중에서도 불교 · 자이나교 등이 제외된다.
가장 좁은 뜻의 힌두교는 불교와 자이나교의 진출로 한때 후퇴하였던 베다 힌두교 전통, 즉 브라만교가 8세기경에 새로운 종교적 형태로 대두한 것을 의미한다.

\paragraph{}
일반적으로 힌두교라고 할 때는, 민간 힌두교 전통과 베다 힌두교 전통으로부터 비슈누파와 같은 박티 전통에 이르기까지, 위의 타 종교들을 제외한 여러 다양하고 복잡한 전통들 전체를 뜻한다. 
이 일반적인 의미의 힌두교에는 또한 요가 전통이 포함되며, 카르마 개념에 기초한 매일의 도덕적 삶과 힌두 결혼 풍습과 같은 사회적 일반 규범도 포함된다.




% ----------------------------------------------------------------------------- section
%
% -----------------------------------------------------------------------------
	\section{힌두교의 역사}


\subsection{발생}
힌두교는 고대 인도에서 발생하였다. 힌두교의 주요 경전인 베다는 기원전 1500년 경에 성립되어 베다 산스크리트어로 기록되었다.

\subsection{전파}
근대 이전에 인도 부근의 네팔, 인도네시아 지역에 전파되었으며 근대에 이르러 인도인들의 이주에 따라 세계 각지로 전파되었다. \\
힌두교의 많은 신들과 주요 사상은 불교에 큰 영향을 주었으며 불교의 전파와 함께 힌두교의 신화와 전설이 전파되었다.

현재 힌두교를 국교로 하는 나라는 네팔이다. 
인도는 종교의 자유를 인정하나 많은 사람들이 힌두교를 믿는다.



\section{힌두교의 경전}

\paragraph{}
힌두교의 근본 경전은 베다 · 우파니샤드이며 그 외에도 브라마나 · 수트라 등의 문헌이 있는데, 이 모든 것들은 인도의 종교적 · 사회적 이념의 원천이 되고 있다. 
\paragraph{}
또한 경전에 준하는 것으로 《마하바라타》와 《라마야나》(라마의 기행)의 2대 서사시가 유명한데, 특히 전자의 일부인 《바가바드 기타》는 널리 애창되고 있다. 
이 외에 푸라나 · 탄트라 · 아가마 · 상히타 등이 힌두교 각 파에서 존중되고 있다.





\section{힌두교의 종파}


조금 부유한 북쪽지방과 브라만계급과 크샤트리아계급중에서는 비슈누파가 많고 남쪽의 가난한 지방과 수드라계급과 불가촉천민인 하리잔에게는 시바파가 많다.





\section{힌두교의 교의}



\subsection{기본 교리}
힌두교의 기본 교의는 \textbf{우주의 법칙}과 \textbf{인간의 윤회}를 근간으로 한다. 
힌두교의 교의를 이루는 기본 개념들은 대부분 이후 불교, 자이나교, 시크교 등 인도에서 발원한 다른 종교에 도입되었다.


			\begin{itemize}[topsep=0.0em, parsep=0.0em, itemsep=0em, leftmargin=12.0em, labelwidth=3em, labelsep=3em] 
			\item 우주론
			\item 윤회 (산스크리트어: Samsara)
			\item 브라만 (산스크리트어: Brahman)
			\item 신 (산스크리트어: Brahman)
			\item 아트만 (산스크리트어: Atman)
			\item 카르마 (산스크리트어: Karma)
			\item 다르마 (산스크리트어: Dharma)
			\item 모크샤 (산스크리트어: Moksa)
			\end{itemize}


\subsection{우주론}
힌두교에 따르면 우주는 생성, 발전, 소멸을 반복한다. 
\textbf{브라흐마신}이 우주를 생성하고 \textbf{비슈누신}이 우주를 유지, 발전시키며 \textbf{시바신}이 우주를 소멸시킨다. 
마치 계절이 바뀌듯 우주의 시간도 바뀌어 결국에는 낡은 우주가 소멸되고 새로운 우주가 만들어진다.

			\begin{itemize}[topsep=0.0em, parsep=0.0em, itemsep=0em, leftmargin=12.0em, labelwidth=3em, labelsep=3em] 
			\item 브라흐마신
			\item 비슈누신
			\item 시바신
			\end{itemize}



\subsection{윤회}
	\paragraph{}
윤회(산스크리트어: Samsara)는 돌아간다는 뜻이다. 흔히 수레바퀴로 상징된다. 
수레바퀴가 한바퀴 돌아 제자리로 돌아가듯 인간 역시 이 번 생애에서 다음 생애로 돌아간다는 뜻이다.

	\paragraph{}
힌두교의 교의에 의하면 인간 역시 우주 안에서 끊임없이 낳고 죽기를 반복하며 윤회한다. 
인간이라 하더라도 다음 생애에는 짐승으로 태어 나기도 하며 천신으로 태어나기도 한다. 
이러한 수없이 많은 윤회의 과정에서 사람은 한 인생을 살면서 쌓은 업(산스크리트어: Karma)에 의해 다음 생애가 결정된다. 
\textbf{다르마}에 따른 삶을 살다 죽으면 다음 생애에는 보다 고귀한 존재로 태어난다. 
\textbf{다르마}에 어긋나는 삶을 살면 고귀한 존재라 할지라도 다음 생애에는 비천한 존재로 태어난다.\\
불교와 비교하기 위해서는 윤회 문서를 참고하십시오.


\subsection{브라만}
브라만(산스크리트어: Brahman)은 우주의 진리 또는 우주 자체를 가리키는 말이기도 하다. 
힌두교의 최상위 목표는 '참된 나'인 아트만이 '우주의 진리'인 브라만과 하나가 되는 것이다. 
카스트 제도의 가장 상위계층을 브라만이라고 칭한 것은 이들이 스스로 우주의 지혜를 구하는 것이 자신들의 업이라고 생각했기 때문이다.\\
불교와 비교하기 위해서는 범천 문서를 참고하십시오.

\subsection{신}
힌두교의 신은 브라마(Brahma), 비슈누(Vishnu), 시바(Shiva)의 삼주신(Trimurti,三主神)과 여러 남신, 여신들로 이루어져 있다.

	\paragraph{삼주신}
	원래 신들의 기본구도가 3신(神)시스템이다. 
	삼위일체(三位一體)니 삼신불(三身佛)이니 삼존불(三尊佛)이니 삼성각(三聖閣)이니 하는 것도 사실 3이란 숫자의 신성(神聖)에서 비롯된 것이다. 
	한마디로 3은 신(神)의 숫자다. 
	3이 상징하는 삼각형은 가장 안정적이고 견고한 구조다. 
	실제로 건축을 할때도 삼각구조는 필수적이다. 
	그래서 그런지 인도의 신격(神格)역시 마찬가지다. 
	이른바 삼신일체(三神一體)라는 교리시스템을 갖고 있는 것이다. 

		대개 힌두교에서는 브라마신과 비슈누신 그리고 시바신을 삼신(三神)구도로 한다.
		신들의 대표주자인 셈이다. 
		브라마신이 우주를 창조하고 비슈누신이 이를 유지하며 시바신이 창조를 위한 파괴를 하는 것이다. 
		그렇게 보면 집단지도체제를 유지하는 셈이다. 
		그리스신화에서 보는 주신(主神)의 전횡과 독선은 보이지 않는다. 
		이것이 인도인의 평화와 관용의 바탕이 되었을 것이다.

	\paragraph{브라마}
	\paragraph{비슈누}
	\paragraph{시바}




\subsection{아트만}
힌두교에서 생명은 숨과 같은 의미로 쓰였으며 아트만의 원래 뜻은 숨쉰다는 뜻이다. 
한국에서 생명을 목숨으로 표현하는 것과 유사하다. 
숨쉬는 생명 아트만(산스크리트어: Atman)은 '나'를 말한다. 
그러나, 끊임없이 윤회하는 삶에서 지금의 나는 '참된 나'가 아니다. 
지금의 나는 과거의 윤회를 거치는 동안 내가 쌓은 업(산스크리트어: Karma, 業)에 의한 것이다.
'참된 나'는 윤회의 과정에서 계속하여 존재하여 없어지지도 않고 변하지도 않는 '나라는 생명의 본질'을 뜻한다.\\
불교와 비교하기 위해서는 무아론 문서를 참고하십시오.

\subsection{카르마}
카르마(산스크리트어: Karma)는 업(業)으로 한역(漢譯)된다. 
살아가면서 하는 모든 행위를 말한다. 
업에는 다르마(산스크리트어: Dharma)에 마땅한 업과 다르마를 거스르는 업이 있다. 
선업(善業)으로 한역(漢譯)되는 '선한' 카르마는 다르마에 마땅한 업을 쌓는 것을 의미하며 선업을 쌓으면 다음 윤회에 보다 존귀한 존재가 된다. 
다르마를 거스르는 악업(惡業)을 쌓으면 다음 윤회에 보다 미천한 존재가 된다.\\
불교와 비교하기 위해서는 업보, 연기 (불교) 문서를 참고하십시오.

\subsection{다르마}
다르마(산스크리트어: Dharma)는 법(法)으로 번역된다. 
이때 '법'이 의미하는 바는 우주에 존재하는 영원한 법칙이기도 하며 모든 생명이 마땅히 따라야할 본질을 말하기도 한다.
고대 인도에서 유래한 신분제도인 카스트 제도에 의한 각 신분의 다르마는 브라흐만의 지혜, 크샤트리아의 용맹, 바이샤의 근면이다. 
플라톤이 《국가》에서 제시한 삼계급설과 유사하다.\\
불교와 비교하기 위해서는 법보 문서를 참고하십시오.

\subsection{모크샤}
모크샤는 브라만과 합일되는 것 또는 깨달음을 얻는 것을 말한다.\\
불교와 비교하기 위해서는 해탈 문서를 참고하십시오.





% ----------------------------------------------------------------------------- section
%
% -----------------------------------------------------------------------------
	\section{힌두교의 일반적인 특징}


% ----------------------------------------------------------------------------- section
%
% -----------------------------------------------------------------------------
	\section{육파철학}

			% 표 : 육파 철학
			% ===== ===== ===== ===== ===== ===== ===== =====  table							.						
			\begin{table} [h]													
				\caption{		육파 철학 	}								
				\label{tab:title}													
																
				\tabulinesep=0.4em													
				\begin{tabu} to 1.0\linewidth { 													
						X[	r,	0.2	]	%	번호						
						X[	l,	1	]	%	학파명
						X[	c,	1	]	%	내용						
						X[	r,	0.2	]	%	비고						
						}											
					\tabucline [1pt,] {-}													
					번호	&명칭	&	내용	&	비고	 \\						
					\tabucline [0.1pt,] {-}													
					\tabucline [0.1pt,] {-}													
					1	&상키아		학파	&		&		 \\						
					2	&요가 		학파	&		&		 \\						
					3	&니야야  		학파	&		&		 \\						
					4	&바이세시카 	학파	&		&		 \\						
					5	&미맘사 		학파	&		&		 \\						
					6	&베단타 		학파	&		&		 \\						
					\tabucline [0.1pt,] {-}													
					\tabucline [0.1pt,] {-}													
				\end{tabu}													
			\end{table}													
																
			\clearpage													
			% -----------------------------------------------------------------------------							.						





% ------ ------ ------ ------ ------ ------ ------ ------ ------ ------ ------ ------
	\chapter{베다 Veda}


% ----------------------------------------------------------------------------- section
%
% -----------------------------------------------------------------------------
	\section{베다}



	베다 · 베다서(Vedas, véda, 지식, 앎, Knowledge, Knowing), 베다 문헌(Vedic texts) 또는 베다 산스크리트 전집(Vedic Sanskrit corpus)은 모두 고대 인도를 기원으로 하는 대량의 신화적 · 종교적 · 철학적 성전이자 문헌들을 가리키는 낱말이다. 
베다 문헌들은 베다 산스크리트어로 기록된 것으로 산스크리트 문학(Sanskrit literature)에서 가장 오래된 층에 해당하며 또한 힌두교의 가장 오래된 성전(聖典)들을 이루고 있다.

	베다 문헌은 삼히타 · 브라마나 · 아란야카 · 우파니샤드 · 수트라의 다섯 부문으로 분류된다. 이 중에서 삼히타는 《리그베다》·《야주르베다》·《사마베다》·《아타르바베다》의 4종의 삼히타를 의미한다. 이 4종의 삼히타는 힌두교의 정전(正典)을 이루며, 특별히, 4종의 베다 정전이라는 의미의 투리야라고 부른다. 베다 문헌 전체는 투리야를 중심으로 형성되어 있다. 그리고 투리야는 그 중에서 《리그베다》를 기초로 하여 형성되었다.[5] 흔히 "베다"라고 할 때는 이들 투리야만을 의미할 경우가 많다.

베다 문헌이 인도의 각 종교와 철학에서 차지하는 위상은 해당 종교와 철학에 따라 다르다. 
이러한 점은 힌두교 종파들 내에서도 마찬가지이다. 
힌두교 전통에서, 베다 문헌을 권위있는 정전(正典)으로 인정하는 인도 종교 및 철학 학파들을 아스티카(āstika) 즉 정통파라고 한다. 
반면, 그렇지 않은 인도의 종교나 철학을 나스티카(nāstika) 즉 이단 또는 비정통파라고 한다. 
힌두교의 입장에서 볼 때, 대표적인 나스티카는 불교와 자이나교인데, 이 두 종교는 베다 문헌의 권위를 인정하지 않는다.
힌두교에 속한 종파이면서도 베다의 권위를 인정하지 않는 종파로는 시크교[8][9] 와 브라모이즘(Brahmoism) 이 있다.




\section{베다의 분류}


\subsection{다섯 부문}
베다 문헌은 크게 삼히타(Samhitas) · 브라마나(Brahmanas, 제의서, 祭儀書) · 아란야카(Aranyakas) · 우파니샤드(Upanishads, 오의서, 奧義書) · 수트라(Sutras)의 5개의 부문으로 분류된다. 이들 베다 문헌들은 힌두교의 정전(正典)을 이루고 있는 투리야라고 부르는 "4종의 삼히타(네 가지 베다)"를 중심으로 형성되어 있다.

삼히타는 "집성" 또는 "컬렉션"을 의미하는데, 4종의 삼히타는 리그베다(Rigveda) · 야주르베다(Yajurveda) · 사마베다(Samaveda) · 아타르바베다(Atharvaveda)이다. "베다" 또는 "베다서"라고 할 때 이들 4종의 삼히타들 즉 투리야만을 지칭하는 경우가 흔하다. 
이들 4종의 삼히타들 중에서 아타르바 베다를 제외한 3종의 삼히타를 트라야라고 한다. 
트라야는 인도 철기시대(1200-272 BC)의 베다 시대 종교(1500-500 BC)에서 희생제식(犧牲祭式)을 뜻하는 야즈나(yajna)의 수행과 관련되어 있다. 
반면, 트라야에 속하지 않는 아타르바베다는 희생제식과는 무관하게 필요에 따라 사용된다.

\subsection{슈루티와 스므리티}
힌두 전통에 따르면, 베다 문헌들은 크게 슈루티와 스므리티의 두 부류로 구분된다. 
슈루티는 "들은 것"을 뜻하고 스므리티는 "기억된 것"을 뜻한다. 
힌두 전통에서 슈루티는 현자들이 명상 속에서 신으로부터 직접 들은 것 또는 신이 직접 드러내어 보여준 것을 기록한 문헌이고 반면 스므리티는 그 기원이 인간에 있는 것으로 스승에서 제자로 전승되어 내려온 지혜를 기록한 문헌이다. 
때문에 슈루티를 다른 명칭으로는 아파우루세야라고도 하는데 "인간에게서 기원하지 않은 것"이라는 의미이다. 
힌두 전통에 따르면, 삼히타 · 브라마나 · 아란야카 · 우파니샤드 · 수트라의 5개의 영역의 베다 문헌들 중에서 수트라를 제외한 나머지 4개 영역의 문헌들은 모두 슈루티에 해당한다.

\subsection{만트라}
투리야 즉 리그베다 · 야주르베다 · 사마베다 · 아타르바베다의 4종의 삼히타는 야주르베다에 산문체의 주해가 삽입되어 있다는 것을 제외하고는 모두 운문체로 되어 있다. 그리고 이들 삼히타들에 들어있는 개별 시구를 만트라(眞言 · mantra)라고 한다. 일부 엄선된 베다 만트라들은 오늘날의 힌두교에서도 기도 · 종교 행사 · 기타 상서로운 의식 등에서 낭송되고 있다.



\section{삼하타}


삼히타는 다음의 네 가지 베다를 말한다.

리그베다: 신들을 희생제식으로 불러들이는 찬가들을 담은 가장 오래된 문헌
야주르베다: 희생제식의 진행과 관련된 만트라들과 정해진 문구들을 담은 문헌
사마베다: 희생제식에서 사용하는 가곡(歌曲) 또는 가창(歌唱)을 담은 문헌
아타르바베다: 재앙을 털어버리고 복을 빌 때 사용되는 내용을 담은 문헌

\subsection{리그베다}
리그베다(Rigveda)의 성립 연대는 기원전 1700-1100년이다. 
힌두교의 여러 신들을 희생제식을 행하는 장소로 불러들이는 찬가들을 담고 있다.
힌두교 사제 계급(Vedic priesthood) 중 호트리(hotr)라고 불리는 주사제가 낭송한다.

\subsection{야주르베다}
야주르베다(Yajurveda)의 성립 연대는 기원전 1400-1000년이다.
희생제식의 진행과 관련된 만트라들과 정해진 문구들을 담고 있다.
힌두교 사제 계급 중 아드바르유(adhvaryu)라 불리는 제식 진행 사제가 낭송한다.

\subsection{사마베다}
사마베다(Samaveda)의 성립 연대는 기원전 1000년이다.
희생제식에서 사용하는 가곡(歌曲) 또는 가창(歌唱)을 담고 있다.
힌두교 사제 계급 중 우드가트리()라고 불리는 찬송가를 맡은 사제가 찬송한다.

\subsection{아타르바베다}
아타르바베다(Atharvaveda)의 성립 연대는 기원전 1200-1000년이다.
주문(呪文), 이야기, 예언, 액막이 부적, 조복(調伏: 원수나 악마를 굴복시킴), 소량의 사색적인 찬가들 등과 같이 주로 재앙을 제거하고 복을 불러오는 주술 관련 내용을 담고 있다.

\subsection{베다의 낭송}
4가지 베다 중에서 3가지 베다만 낭송된다. 
즉 야주르베다는 1음(音)만으로 낭송되고, 리그베다는 가사의 악센트에 따라 중음(中音) · 저음(低音) · 고음(高音)으로 낭송된다. 
사마베다는 큰 의식에서 좀 더 넓은 음넓이로 간음(間音)을 더 넣으면서 낭송된다.





\section{브라마나}


\section{아란야마}


\section{우파니샤드}


\section{수트라}



% ------ ------ ------ ------ ------ ------ ------ ------ ------ ------ ------ ------
	\chapter{윤홍식 홍익학당}



% ----------------------------------------------------------------------------- section
%
% -----------------------------------------------------------------------------
	\section{윤홍식의 홍익학당}




			% 표 : 육파 철학
			% ===== ===== ===== ===== ===== ===== ===== =====  table							.						
			\begin{table} [h]													
				\caption{		육파 철학 	}								
				\label{tab:title}													
																
				\tabulinesep=0.4em													
				\begin{tabu} to 1.0\linewidth { 													
						X[	r,	0.2	]	%	번호						
						X[	l,	1	]	%	학파명
						X[	c,	1	]	%	내용						
						X[	r,	0.2	]	%	비고						
						}											
					\tabucline [1pt,] {-}													
					번호	&명칭	&	내용	&	비고	 \\						
					\tabucline [0.1pt,] {-}													
					\tabucline [0.1pt,] {-}													
					1	&56분  몰입의 기술 			&		&		 \\						
					2	&초보자를 위한 단학			&		&		 \\						
					3	&양심이 답이다				&			&		 \\						
					4	&선문답에서 배우는 선의 지혜	&		&		 \\						
					5	&내안의 창조성을 깨우는 몰입	&		&		 \\						
					\tabucline [0.1pt,] {-}													
					\tabucline [0.1pt,] {-}													
				\end{tabu}													
			\end{table}													
																
			\clearpage													
			% -----------------------------------------------------------------------------							.						


%	================================================================== Part			조르아스트교
	\addtocontents{toc}{\protect\newpage}
	\part{조르아스트교}
	\noptcrule
	\parttoc				


% ========================================================================================= chapter
	\chapter{조르아스트교}

	% -----------------------------------------------------------------------------
	%
	% -----------------------------------------------------------------------------
		\section{조르아스트교}	

		조르 아스트교 
		마쯔다교
		배화교
		


 	 	 	% ===== ===== ===== ===== ===== ===== ===== ===== ===== ===== ===== ===== ===== =====											
 	 	 	\begin{itemize}[											
 	 	 				topsep=		0	em,					
	 					parsep=		0	em,					
						itemsep=		0	em,					
						leftmargin=		6	em,					
						labelwidth= 	3	em,					
						labelsep=		3	em					
						]								
			% ----- ----- ----- ----- ----- ----- ----- ----- ----- ----- ----- ----- ----- ----- ----- ----- ----- ----- ----- ----- ----- ----- ----- ----- 											
			\item 중동의 박트리아지방에서 자라슈트라에 의해 세워진 종교
			\item 조르아스트라는 이름은 ''차라수트라''가 그리스어에서 변한것이다.
			\item 조로아스교의 창조신 '아후라 마즈타''를 중심으로 선과 악의 질서 및 세계를 구분하는 것이 특징이다.
			% ----- ----- ----- ----- ----- ----- ----- ----- ----- ----- ----- ----- ----- ----- ----- ----- ----- ----- ----- ----- ----- ----- ----- ----- 											
			\end{itemize}											
			% ===== ===== ===== ===== ===== ===== ===== ===== ===== ===== ===== ===== ===== =====											





%	================================================================== Part			불교
	\addtocontents{toc}{\protect\newpage}
	\part{불교}
	\noptcrule
	\parttoc				


% ========================================================================================= chapter
	\chapter{불교}

	% -----------------------------------------------------------------------------
	%
	% -----------------------------------------------------------------------------
	\section{불교}










%	================================================================== Part			요가 개요
	\addtocontents{toc}{\protect\newpage}
	\part{인도}
	\noptcrule
	\parttoc				


% ========================================================================================= chapter
	\chapter{인도의 역사}

% -----------------------------------------------------------------------------
%
% -----------------------------------------------------------------------------
	\section{인도의 역사}


% ========================================================================================= chapter
	\chapter{인도의 왕조}

% -----------------------------------------------------------------------------
%
% -----------------------------------------------------------------------------
	\section{인도의 왕조}


			% 표 : 인도의 왕조													
			% ===== ===== ===== ===== ===== ===== ===== =====  table							.						
			\begin{table} [h]													
			\caption{		인도의 왕조			}								
			\label{tab:title}													
			\tabulinesep=1.0em													
			\begin{longtabu} to 1.0\linewidth { 													
					X[	r,	1.0	]	%	기간						
					X[	l,	1	]	%	왕조명						
					X[	c,	0.2	]	%	내용						
					X[	r,	0.2	]	%	비고						
					}											
			\tabucline [1pt,] {-}													
			기간	&	왕조명	&	내용	&	비고	 \\						
			\tabucline [0.1pt,] {-}													
			\tabucline [1.1pt,] {-}													
			\endfirsthead													
			\endhead													
						&16국 시대			&		&		 \\						
						&4대 국가 시대		&		&		 \\						
						&마가다 왕국		&		&		 \\						
						&난다 왕국			&		&		 \\						
						&마우리아 왕조		&		&		 \\						
			\tabucline [0.1pt,] {-}													
			\tabucline [0.1pt,] {-}													
			\end{longtabu}													
			\end{table}													
																
%			\clearpage													
			% -----------------------------------------------------------------------------							.						



			% 표 : 인도의 왕조													
			% ===== ===== ===== ===== ===== ===== ===== =====  table							.						
			\begin{table} [h]													
			\caption{		인도의 왕조			}								
			\label{tab:title}													
			\tabulinesep=1.0em													
			\begin{longtabu} to 1.0\linewidth { 													
					X[	r,	1.0	]	%	기간						
					X[	l,	1	]	%	왕조명						
					X[	c,	0.2	]	%	내용						
					X[	r,	0.2	]	%	비고						
					}											
			\tabucline [1pt,] {-}													
			기간	&	왕조명	&	내용	&	비고	 \\						
			\tabucline [0.1pt,] {-}													
			\tabucline [0.1pt,] {-}													
			\endfirsthead													
			\endhead													
						&북인도			&		&		 \\						
						&남인도			&		&		 \\						
			\tabucline [0.1pt,] {-}													
			\tabucline [0.1pt,] {-}													
			\end{longtabu}													
			\end{table}													



																
			\clearpage													
			% -----------------------------------------------------------------------------							.						


			% 표 : 인도의 왕조													
			% ===== ===== ===== ===== ===== ===== ===== =====  table							.						
			\begin{table} [h]													
			\caption{		인도의 왕조			}								
			\label{tab:title}													
			\end{table}													
																
			\begin{center}													
																
			\tabulinesep=0.0em													
			\begin{longtabu} to 1.0\linewidth { 													
					X[	r,	1.0	]	%	기간						
					X[	l,	1	]	%	왕조명						
					X[	c,	0.2	]	%	내용						
					X[	r,	0.2	]	%	비고						
					}											
			\tabucline [1pt,] {-}													
			기간	&	왕조명	&	내용	&	비고	 \\						
			\tabucline [0.1pt,] {-}													
			\tabucline [0.1pt,] {-}													
			\endfirsthead													
			\endhead													
																
			BC 321 - BC 184	&마우리아 제국		&		&		 \\						
						&쿠산왕조, 굽타왕조	&		&		 \\						
			1206 - 1526		&인도 이슬람 제국시대	&		&		 \\						
			1526 - 1858		&무굴 제국			&		&		 \\						
			1848 - 1947		&영국령 인도		&		&		 \\						
						&인도 공화국		&		&		 \\						
			\tabucline [0.1pt,] {-}													
			\tabucline [0.1pt,] {-}													
			\end{longtabu}													
			\end{center}													
																
			\clearpage													
			% -----------------------------------------------------------------------------							.						
																
																
																
																


	




% -----------------------------------------------------------------------------
%
% -----------------------------------------------------------------------------
	\section{굽타 왕조‎}


% -----------------------------------------------------------------------------
%
% -----------------------------------------------------------------------------
	\section{마우리아 왕조‎}

% -----------------------------------------------------------------------------
%
% -----------------------------------------------------------------------------
	\section{무굴 제국‎}

% -----------------------------------------------------------------------------
%
% -----------------------------------------------------------------------------
	\section{벵골의 왕조‎ }

% -----------------------------------------------------------------------------
%
% -----------------------------------------------------------------------------
	\section{십육대국‎}

% -----------------------------------------------------------------------------
%
% -----------------------------------------------------------------------------
	\section{출라 왕조‎}

% -----------------------------------------------------------------------------
%
% -----------------------------------------------------------------------------
	\section{쿠샨 왕조‎ }


% ========================================================================================= chapter
	\chapter{인도의 종교}

% -----------------------------------------------------------------------------
%
%
%
% -----------------------------------------------------------------------------
\newpage
\section{인도의 종교}

\section{힌두이즘}

\section{불교}

\section{자이나교}


\section{시크교}


% ========================================================================================= chapter
\chapter{아리아인}

\section{아리아인}



\chapter{인도의 지리}


\newpage
\section{갠지즈 강}

\section{야무나 강}



% ========================================================================================= chapter
	\chapter{힌디어}


	\section{데바나가리 문자}

힌디어는 인도유럽어오족의 인도-이란어파에 속하며 데브나가리문자를 쓰는 언어입니다.
데바나가리문자는 모두 46자로 모음 11개와 자음 35개로 이루어진 표음문자입니다.



데바나가리 문자()는 고대 인도에서 생겨나 발달한 문자이다. 
아부기다로서, 음절문자와 알파벳의 특징을 두루 갖췄다. 
산스크리트어를 비롯하여, 팔리어, 프라크리트어(en:Prakrit) 등을 옮겨적는데 쓰였다. 
현대 인도 언어 가운데 힌디어, 카슈미르어, 마라티어, 네팔어 등 인도의 여러 대언어의 기록, 인쇄 등에 쓰이고 있다.



\subsection{기원}
데바나가리는 아람 문자에서 갈라져 나온 브라흐미 문자에서 그 기원을 찾을 수 있다. 
기원전 3세기 무렵부터 사용되었던 브라흐미 문자는 6세기 무렵에 여러 문자로 분화되기 시작하였다. 
그 가운데 하나가 7세기 무렵에 생겨난 나가리 문자로 이는 지금의 데바나가리 문자의 원형이다.

'나가리() 는 '나가라'() 즉 '도시의 문자'라는 뜻이다. 
여기에 '신'이란 뜻인 '데바' (deva; = 신)가 접두어로 붙어 신성시된 것이 데바나가리란 이름의 유래이다.



	\subsection{특징}
	\paragraph{}

데바나가리 문자의 주된 특징으로 각 문자가 문자 윗부분의 일직선에 이어진채로 쓰인다는 점을 꼽을 수 있다. 현대어에서는 이 수평선이 낱말 단위로 그어져 있으나, 산스크리트어 전통 서법에서는 모든 문장에 이 수평선을 그었다.

데바나가리는 닿소리를 나타내는 글자가 특정 홀소리(일반적으로는 음가 /a/)를 수반하는 음절로 읽혀지는 아부기다 문자이다. 이것을 딸림 모음이라고 하는데, 인도의 언어에는 -a라는 홀소리가 매우 많이 쓰이기 때문에 기본형이 되었다. a소리 이외의 모음을 나타내려 할 때는 닿소리 문자에 별도의 홀소리기호를 붙인다.

데바나가리는 33개의 닿소리 문자와 10 종류의 홀소리 문자로 이루어지며, 여기에 아홉 종류의 모음 부호와 비음, 무성기음 기호에 생략을 나타내는 몇가지 기호, 숫자 10개로 구성되어 있다. 쓰는 순서는 왼쪽에서 오른쪽이다.



	\href{https://www.youtube.com/channel/UCoJOZqb7VzOv-nL0r9-EaTg}{한국을 사랑한 인도인}

	\href{https://www.youtube.com/watch?v=N7zYgG6JxPg}{자음 쓰기 1}
	\href{https://www.youtube.com/watch?v=yvLmgVnGaOs}{자음 쓰기 2}
	\href{https://www.youtube.com/watch?v=TohXp9qSgeQ&t=2s}{자음 쓰기 3}
	\href{https://www.youtube.com/watch?v=yhL2FyehmGo}{자음 쓰기 4}
	\href{https://www.youtube.com/watch?v=mlDiVqy-_KU}{자음 쓰기 5}


	\section{모음}


	\section{자음}




	\section{유니코드}



% ========================================================================================= chapter
	\chapter{참고 문헌}

% -----------------------------------------------------------------------------
%
% -----------------------------------------------------------------------------
	\section{참고 문헌}

% -----------------------------------------------------------------------------
%
% -----------------------------------------------------------------------------
	\section{아쉬탕가 요가의 힘 키노의 프라이머리 시리즈 완벽 가이드}


	\paragraph{지은이} 
	\paragraph{목차} 





%	================================================================== Part			요가강습
	\addtocontents{toc}{\protect\newpage}
	\part{요가 강습}
	\noptcrule
	\parttoc				


% ========================================================================================= chapter
	\chapter{요가 강습}

% -----------------------------------------------------------------------------
%
% ----------------------------------------------------------------------------- 동구문화센타
	\section{동구 국민 체육 문예 센타}


% -----------------------------------------------------------------------------
%
% ----------------------------------------------------------------------------- 동구문화센타
	\section{요가 오승아}

		요가 오승아 
		강사사진
		대한요가 요가지도, 제임스짐 휘트니...

% ----------------------------------------------------------------------------- 요가장예령
%
% ----------------------------------------------------------------------------- 
	\section{요가 장예령}

		요가 장예령 
		강사사진
		아나하타 요가원 빈야사요가 지도, 라...


	\subsection{2020년 07월}


			% 표 : 2020년 7월 출석표 7월출석
			% ===== ===== ===== ===== ===== ===== ===== =====  table							.						
			\begin{table} [h]													
			\caption{		2020년 7월 출석부 }
			\label{tab:title}													
			\tabulinesep=0.2em													
			\begin{longtabu} to 1.0\linewidth { 													
					X[	r,	0.1	]	%	번호						
					X[	l,	1.0	]	%	일자						
					X[	l,	0.2	]	%	출석						
					X[	r,	1.0	]	%	비고						
					}											
			\tabucline [1pt,] {-}													
			번호	&일자	&출석	&	비고	 \\						
			\tabucline [0.1pt,] {-}													
			\tabucline [1.1pt,] {-}													
			\endfirsthead													
			\endhead													

1	&	2020년 7월 1일 수요일	&출		&첫날		\\ 
2	&	2020년 7월 2일 목요일	&출		&		\\
3	&	2020년 7월 3일 금요일	&출	&		\\
4	&	2020년 7월 4일 토요일	&	휴무	&		\\
5	&	2020년 7월 5일 일요일	&	휴무	&		\\
			\tabucline [0.1pt,] {-}													
6	&	2020년 7월 6일 월요일	&휴무	&		\\
7	&	2020년 7월 7일 화요일	&결		&		\\
8	&	2020년 7월 8일 수요일	&출		&		\\
9	&	2020년 7월 9일 목요일	&결석	&운동복을 안챙겨감		\\
10	&	2020년 7월 10일 금요일	&		&		\\
11	&	2020년 7월 11일 토요일	&	휴무	&		\\
12	&	2020년 7월 12일 일요일	&	휴무	&		\\
			\tabucline [0.1pt,] {-}													

13	&	2020년 7월 13일 월요일	&		&		\\
14	&	2020년 7월 14일 화요일	&		&		\\
15	&	2020년 7월 15일 수요일	&		&		\\
16	&	2020년 7월 16일 목요일	&		&		\\
17	&	2020년 7월 17일 금요일	&		&		\\
18	&	2020년 7월 18일 토요일	&	휴무	&		\\
19	&	2020년 7월 19일 일요일	&	휴무	&		\\
			\tabucline [0.1pt,] {-}													

20	&	2020년 7월 20일 월요일	&		&		\\
21	&	2020년 7월 21일 화요일	&		&		\\
22	&	2020년 7월 22일 수요일	&		&		\\
23	&	2020년 7월 23일 목요일	&		&		\\
24	&	2020년 7월 24일 금요일	&		&		\\
25	&	2020년 7월 25일 토요일	&	휴무	&		\\
26	&	2020년 7월 26일 일요일	&	휴무	&		\\
			\tabucline [0.1pt,] {-}													

27	&	2020년 7월 27일 월요일	&		&		\\
28	&	2020년 7월 28일 화요일	&		&		\\
29	&	2020년 7월 29일 수요일	&		&		\\
30	&	2020년 7월 30일 목요일	&		&		\\
31	&	2020년 7월 31일 금요일	&		&		\\
			\tabucline [0.1pt,] {-}													
			\tabucline [0.1pt,] {-}													
			\end{longtabu}													
			\end{table}													






% ------------------------------------------------------------------------------
% End document
% ------------------------------------------------------------------------------
\end{document}


		\begin{itemize}[topsep=0.0em, parsep=0.0em, itemsep=0em, leftmargin=6.0em, labelwidth=3em, labelsep=3em] 
			\item 	[①]
		\end{itemize}




% =================================================================================================== Part 혼화 재료

% ========================================================================================= chapter

%	-----------------------------------------------------------  section  

	%	------------------------------------------------------------------------------  table

			\begin{table} [h]
	
			\caption{잔 골재의 표준입도}  
			\label{tab:title} 
	
			\begin{center}
			\tabulinesep=0.4em
			\begin{tabu} to 0.8\linewidth { X[r] X[l] X[c]  }
			\tabucline [1pt,] {-}
			체의 호칭 치수 (mm)		& 체를 통화한 것의 질량 백분율(\%) \\
			\multicolumn{4}{c} {단위량} \\
			\tabucline [0.1pt,] {-}
			2.5	&100\\
			1.2	& 99 $\sim$ 100 \\
			0.6	& 60 $\sim$ 80 \\
			0.3	& 20 $\sim$ 50 \\
			0.15	&  5 $\sim$ 30 \\
			\tabucline [0.1pt,] {-}
			\end{tabu} 
			\end{center}
			\end{table}



	%	------------------------------------------------------------------------------  문제

		\clearpage
		\begin{small}	
		\begin{question}
		레디 믹스트 콘크리트에서 \textbf{회수수}를 혼합수로 사용할 경우 주의할 사항 중 틀린것은 ?
		\begin{enumerate}[label=\arabic*), topsep=0.0em, itemsep=-1.0em ]
			\item [①] 고강도 콘크리트의 경우 회수수를 사용하여서는 안된다. 
			\item [②] 슬러지수의 사용 시 \textbf{단위 슬러지 고형분}은 콘크리트 질량의 3\% 이하로 한다. 
			\item [③] 회수수의 품질 시험 항목은 4가지로 염소 이온량, 시멘트 응결 시간의 차, 모르타르 압축강도의 비, 단위 슬러지 고형분율 이다. 
			\item [④] 콘크리트를 배합할 때, 회수수 중에 함유된 슬러지 고형분은 물의 질량에는 포함되지 않는다. 
		\end{enumerate}
		\end{question}



		\begin{solution}
		해설
		\end{solution}
		\end{small}	
		\hrulefill


		%	----------------------------------------------------------------------------- 수식
			\begin{equation}
			\begin{aligned}
			T_2 = T_1 - 0.15 ( T_1 - T_0 ) \times t
			\end{aligned}
			\end{equation}


			\begin{description}[style=sameline, leftmargin=2cm, topsep=0.0em, itemsep=0.0em]
			\item[$T_0$] 		주의의 온도(${}^\circ C$)
			\item[$T_1$] 		비볐을때의 콘크리트 온도 (${}^\circ C$)
			\item[$t$] 		비빈후 부터 타설이 끝났을때 까지의 시간 (h)
			\end{description}


			\begin{description}[style=sameline, leftmargin=2cm, topsep=0.0em, itemsep=0.0em]
			\item[$T_0$] 		주의의 온도(${}^\circ C$)
			\item[$T_1$] 		비볐을때의 콘크리트 온도 (${}^\circ C$)
			\item[$t$] 		비빈후 부터 타설이 끝났을때 까지의 시간 (h)
			\end{description}



			\begin{itemize}		[
							topsep=0.0em, 
							parsep=0.0em, 
							itemsep=0em, 
							leftmargin=12.0em, 
							labelwidth=3em, 
							labelsep=3em
							] 
			\item []
			\end{itemize}


			\begin{itemize}[topsep=0.0em, parsep=0.0em, itemsep=0em, leftmargin=12.0em, labelwidth=3em, labelsep=3em] 
			\item [1.] 	따다 아사나
			\item [2.] 	우드르바 하스타 아사나
			\item [3.] 	웃따나 아사나
			\item [4.] 	아르다 웃따나 아사나
			\item [5.] 	차뚜랑가 단다 아사나
			\item [6.] 	우르드바 우카 스바나 아사나
			\item [7.] 	아도무카 스바나 아사나
			\item [8.] 	아르다 웃따나 아사나
			\item [9.] 	웃따나 아사나
			\item [10.] 	우드르바 하스타 아사나
			\item [11.] 	따다 아사나
			\end{itemize}


	\href{https://www.youtube.com/watch?v=SpqKCQZQBcc}{태양경배자세A}
	\href{https://www.youtube.com/watch?v=CL3czAIUDFY}{태양경배자세A}


